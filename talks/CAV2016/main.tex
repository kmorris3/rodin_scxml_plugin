

%-----------------------------------------------------------
\documentclass[10pt,xcolor=svgnames,compress,blackandwhite]{beamer}
\usepackage{etex}
\mode<presentation>
%----------------------------------------------------------
%	colors and appearance
\usetheme{Warsaw}

\setbeamercolor{structure}{fg=SteelBlue!60!Black}
%{fg=SteelBlue!65!black}
\useoutertheme[subsection=false]{miniframes} 
%\useoutertheme{infolines} 
%------------------------------
%
%	block and items
\setbeamertemplate{blocks}[rounded][shadow=true] 
\setbeamertemplate{navigation symbols}{} 
\useinnertheme{circles} 
\setbeamercovered{transparent}
%
%
%\Large size for frame titles
\setbeamercolor{title}{fg=white, bg=SteelBlue!50!black}
\setbeamerfont{frametitle}{size={\large}} 
%\setbeamercolor{frametitle}{fg=black}
\setbeamercolor{frametitle}{bg=SlateGray}
%-----------------------------------------

%------------------------------------------
%	to have the background black.... 
%\setbeamercolor{structure}{fg=red!55!black}
%\beamersetaveragebackground{gray!60!black}
%\setbeamercolor{normal text}{fg=white}
%\setbeamercolor{itemize item}{fg=red!60!black}
%\setbeamercolor{enumerate item}{fg=black}
%------------------------------------------


\usepackage[english]{babel}
\usepackage[latin1]{inputenc}
\usepackage{tikz}
\usepackage{pgf}
\usepackage{verbatim}
\usetikzlibrary{arrows,shapes,backgrounds}
\usepackage{times}
\usepackage[T1]{fontenc}
\usepackage{multimedia}
\usepackage{hyperref}
\usepackage{xkeyval}
\usepackage{grffile} % To parse file names properly in \includegraphics
\usepackage{graphicx} 
\usepackage{subfigure}
\usepackage{xcolor}
\usepackage{rotating}
\usepackage{amsmath,amsfonts,amssymb,amsthm,amsbsy}
% \usepackage{footbib}
\usepackage{bbding}
\usepackage{animate}
\usepackage{multirow}
\usepackage{multicol}
\usepackage{booktabs} 
\usepackage{bm}
\usepackage{xspace}
\usepackage{mathtools}
%\usepackage[font=Times,timeinterval=20, timeduration=45, timewarningfirst=19, colorwarningfirst=green,
%timewarningsecond=68, colorwarningsecond=yellow,timedeath=0]{tdclock}
\usepackage{cite}
\RequirePackage{cite}
\usepackage{anyfontsize}
\usepackage{t1enc}

\usepackage[font=Times,timeinterval=15, timeduration=25, 
timewarningfirst=35, colorwarningfirst=green,
timewarningsecond=70,colorwarningsecond=yellow,
timedeath=0,font=Helv]{tdclock} 
% timewarningfirst=26, colorwarningfirst=green,			
% % end of introduction
% timewarningsecond=74,colorwarningsecond=yellow,
% % end of nanopore
% timedeath=0,font=Helv]{tdclock}


\newcommand{\cgr}[1]{{\color{green} #1}}
\newcommand{\ccy}[1]{{\color{cyan} #1}}
\newcommand{\cye}[1]{{\color{yellow} #1}}
\newcommand{\ssk}{\smallskip}
\newcommand{\bbk}{\bigskip}
\newcommand{\mmk}{\medskip}
%\usepackage{natbib}
%\bibpunct{(}{)}{;}{a}{,}{,} 

\setbeamercolor{lowercol}{fg=black,bg=SteelBlue!22}







 

\input{def}

\usepackage{soul}


\graphicspath{{./figs/}}

% compiles for final build all stuff needed 
% if undefined, then build more rapidly
\def\finalbuild{1}


%%%%%%%%%%%%%%%%%%%%%%%%%%%%%%%%%%%%%%%%%%%%%%%%%%%%%%%%%%%%%%%%%%%%%%%%%%
\title[Resilient Approach for PDE \ \qquad 
Min:~\factorclockfont{1.45} \cronominutes] 
{\fontsize{0.415cm}{0.5cm}\selectfont 
ULFM-MPI Implementation of a Resilient Task-Based Partial Differential Equations Preconditioner}

\author[F.~Rizzi]
% {\small Francesco Rizzi\\ ($8953$)
% \vspace*{-2.5ex}
% }
{\small {\it F.Rizzi}$^\dag$, K.Morris$^\dag$, 
K.Sargsyan$^\dag$, P.Mycek$^\ddag$, C.Safta$^\dag$,\\
O.LeMaitre$^\ddag$, O.Knio$^\ddag$, 
B.Debusschere$^\dag$ 
\vspace*{-2ex}
}

\institute[] % (optional, but mostly needed)
{%{\texttt{\bf fnrizzi@sandia.gov}}\\
$^\dag$Sandia National Laboratories, Livermore, CA\\
$^\ddag$Duke University, Durham, NC
\vspace*{-1ex}
}
% - Use the \inst command only if there are several affiliations.

\date{\scriptsize {\bf FTXS16 Workshop - HPDC16} \\ 
-- May 2016 --
\vspace{-0.25cm} }

% \logo{\includegraphics[height=0.035\textwidth]
% {figs/SNL_Stacked_Black_Blue.pdf}}
%%%%%%%%%%%%%%%%%%%%%%%%%%%%%%%%%%%%%%%%%%%%%%%%%%%%%%%%%%%%%%%%%%%%%%%%%



%%%%%%%%%%%%%%%%%%%%%%%%%%%%%%%%%%%%%%%%%%%%%%%%%%%%%%%%%%%%%%%%%%%%%%%%
\begin{document}


%\tikzstyle{na} = [baseline=-.5ex]
% \tikzstyle{every picture}+=[remember picture]
%
\begin{frame}
\vspace{-0.5cm}
\titlepage

\vspace*{-5mm}
\begin{center}
{\scriptsize
Supported by the US Department of Energy (DOE)\\ 
Advanced Scientific Computing Research (ASCR)\par
}
\end{center}

%\vspace*{1ex}

\vspace{-0.4cm}
\begin{columns}%[c,totalwidth=\textwidth]
\begin{column}{0.7\textwidth}
{\center {\tiny
Sandia National Laboratories is a multi-program 
laboratory managed and operated by
Sandia Corporation, a wholly owned subsidiary 
of Lockheed Martin Corporation, for the
U.S. Department of Energy's National Nuclear 
Security Administration under contract
DE-AC04-94AL85000.\par}}
\end{column}
\end{columns}


\end{frame}


%%%%%%%%%%%%%%%%%%%%%%%%%%%%%%%%%%%%%%%%%%%%%%%%%%%%%%%%%%%%%%%%%%%%%%%%%%%
% \begin{frame}{Outline}
% % \begin{multicols}{2}
% %  \tableofcontents [hideallsubsections]      % [pausesections]
%   \tableofcontents
% %   \end{multicols}
% \end{frame}
% \section{Outline}
% \begin{frame}
% \tableofcontents
% \end{frame}
% %%%%%%%%%%%%%%%%%%%%%%%%%%%%%%%%%%%%%%%%%%%%%%%%%%%%%%%%%%%%%%%%%%%%%%%


\section{Motivation}
\graphicspath{{./figs/}}
% !TEX root = ../SCXMLREF.tex

\section{Introduction}
\label{sec:introduction}
This is the introduction...


\section{Algorithm}
\graphicspath{{./figsAlgo/}}
\input{algorithm.tex}

\section{Implementation}
\graphicspath{{./figsImpl/}}
\input{impl.tex}

\section{Results}
\graphicspath{{./figsResults/}}
% !TEX root = ../main.tex
\section{Results}

Text of paper \ldots

\section{Conclusions}


\begin{frame}
\frametitle{Conclusions and Ongoing Work}
%
\begin{columns}
% \hspace{-0.7cm}
\begin{column}{1.1\textwidth}
\bi
\item Application is resilient to:
  \bi
  \item Silent Data Corruptions during sampling.
  \item Missing data due to communication issues or node failures.
  \ei
  \mmk
\item Sampling/decomposition approach provides concurrency/parallelism.
\mmk
\item Convergence is achieved in all cases.
\mmk
\item Scalability is excellent.
\mmk
\item Ongoing work/outlook:
  \bi
  \item Dimensionality reduction.
  \ssk
  \item Extension to other types of PDE.
  % \ssk
  % \item Other faults?  
  \ei
\ei
%
\end{column}
\end{columns}
%
\end{frame}



%------------------------------



% \begin{frame}
% \frametitle{Overview of Past/Current Research Interests}
% %
% \begin{columns}
% \hspace{-1.6cm}
% \begin{column}{1\textwidth}
% \begin{figure}
% \includegraphics[width=1.15\textwidth]
% {researchOverview.pdf}
% \end{figure}
% \end{column}
% \end{columns}

% %
% % \begin{columns}
% % \begin{column}{1.1\textwidth}
% % \bi
% % \item Uncertainty Quantification: sparse grids, 
% % high-dimensionality, adaptive methods, forward/inverse problems, 
% % Bayesian inference, sensitivity analysis.
% % \mmk
% % \item Reactive nano-materials: diffusion, inverse problems.
% % \mmk
% % \item Molecular Dynamics: heterogenous systems, nanopores, 
% % LAMMPS (Sandia).
% % \mmk
% % \item UQ for Combustion: sensitivity to reactions rates uncertainties.
% % \mmk
% % \item CFD: fluid, mixing, vortex dynamics, Lagrangian methods, 
% % gravity currents, stratified flows.
% % \ei
% % %
% % \end{column}
% % \end{columns}
% %
% \end{frame}



% %------------------------------



% \begin{frame}
% \vspace{0.5cm}
% \begin{center}
% \begin{block}{   }
% \centering
% {\LARGE{{Thanks for your attention!}}}
% \end{block}
% \end{center}
% %
% \begin{columns}
% \begin{column}{0.33\paperwidth}
% \begin{figure}
% \includegraphics[width=0.93\textwidth]
% {./figsAlgo/2dAlgo/2dalgo.pdf}
% \end{figure}
% \end{column}
% %
% \hspace{-0.05cm}
% %
% \begin{column}{0.33\paperwidth}
% \begin{figure}
% \includegraphics[width=0.85\textwidth]
% {./figsImplementation/TMschematic}
% \end{figure}
% \end{column}
% %
% \hspace{-0.05cm}
% %
% \begin{column}{0.33\paperwidth}
% \begin{figure}
% \includegraphics[width=1\textwidth]
% {./figs2D/HF_rmsResidual_vs_numFaults.eps}
% \end{figure}
% \end{column}
% \end{columns}
% %
% % \bbk
% % \mmk
% % \begin{center}
% % \centering
% % Research supported by: \\ 
% % US, DOE - Office of Advanced Scientific Computing Research
% % \end{center}
% \end{frame}
% %----------------------------------



\begin{frame}
\frametitle{Acknowledgments}
%
\begin{columns}
\begin{column}{0.95\textwidth}
\centering
This material is based upon work supported by the 
U.S. Department of Energy, Office of Science, Office of 
Advanced Scientific Computing Research, 
under Award Number 13-016717.\par 
\end{column}
\end{columns}
%
\bbk
\mmk

\begin{columns}
\begin{column}{0.95\textwidth}
\centering
Sandia National Laboratories is a multi-program 
laboratory managed and operated by Sandia Corporation, 
a wholly owned subsidiary of Lockheed Martin Corporation, 
for the U.S. Department of Energy's 
National Nuclear Security Administration under contract 
DE-AC04-94AL85000.\par
\end{column}
\end{columns}
%
\bbk
\mmk

\pause
\begin{center}
{\Large \bf THANK YOU!}
\end{center}

\end{frame}


%%%%%%%%%%%%%%%%%%%%%%%%%%%%%%%%%%%%%%%%%%%%%%%%%%%%%%%%%%%%%%%%%%%%%%%

\end{document}

