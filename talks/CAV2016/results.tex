


\begin{frame}

\vspace{0.5cm}
\centering
\begin{block}{}
\centering
\textbf{\Large{{\it {Resilience Results}}}}
\end{block}
%
\end{frame}



\begin{frame}
\frametitle{Test Problem}
%
\vspace{-0.05cm}

\begin{columns}
\begin{column}{0.65\textwidth}
\bi
\item 2D linear elliptic equation.
\item $201^2$ grid, $3x3$ subdomains.
\item Nominally: $3249$ sampling and $2136$ regression tasks. 
\item 1 {\bf \textcolor{green}{server}}, 
14 {\bf \textcolor{pink}{clients}} size 2. 
\item Faults affect clients only.
\ei
\end{column}
\begin{column}{0.4\textwidth}
\vspace{-0.35cm}
\begin{center}
\includegraphics[width=1\textwidth]{pdeSC15Solution.png}\\
{\small True Solution}
\end{center}
\end{column}
\end{columns}
%
\vspace{0.2cm}
%
\begin{columns}
\begin{column}{0.55\textwidth}
\begin{center}
\vspace{-0.4cm}
\includegraphics[width=0.6\textwidth]{/layout/layout.png}\\
{\small Partitioning}
\end{center}
\end{column}
\begin{column}{0.55\textwidth}
\begin{center}
\vspace{-0.5cm}
\includegraphics[width=0.5\textwidth]{SCresil.pdf}\\
{\small SC configuration}
\end{center}
\end{column}
\end{columns}
%
\end{frame}




\begin{frame}
\frametitle{Injecting SDC and Hard Faults}
%
\begin{columns}
\begin{column}{1.1\textwidth}

\begin{block}{Selective reliability}
Inject/perturb applications at target points and evaluate how it behaves. 
Some parts of the algorithm are assumed to be handled in a 
more reliable manner than others. \footnotesize{M.Hoemmen,M.Heroux,2012}
\end{block}

\pause
\begin{block}{Silent Data Corruptions (SDC)}
\bi
% \item SDC: transient and do not cause the termination of the application.
\item Selective reliability: only affect sampling stage.
\item Injection at random; $\# faults = 0.25, 0.5, 1 \%$ of tasks.
\item Corrupt all boundary conditions data of a task.
\item Bit-flip model: random bit-flip in binary representation. 
\ei
\end{block}
%
\pause
\begin{block}{Hard Faults}
\bi
% \item Permanent, cause the termination of the application.
\item Selective reliability: can affect sampling and regression.
\item Injection at random; $2, 4, 6$ clients crashing.
\item Actually kill the processes associated with those ranks.
\ei
\end{block}
\end{column}
\end{columns}
%
\end{frame}






\begin{frame}
\frametitle{Details}
%
\begin{columns}
\begin{column}{1.1\textwidth}
\begin{block}{Silent Data Corruptions (SDC)}
\bi
\item Resilience condition: out of the samples used in the regression, 
the number of uncorrupted samples has to be greater than 
the minimum set needed to have a well-posed 
regression problem. 
% \item Filter $(-100,100)$ to eliminate outrageous data (expert opinion).
\ei
\end{block}
%
\begin{block}{Hard Faults}
\bi
\item Server continues the execution using 
only the clients that are alive.
\item No need for ULFM collectives to rebuild broken communicators.
\ei
\end{block}
\begin{block}{Oversampling}
\bi
\item Oversampling: $\rho >1$, such that $N = \rho N_{nom}^{s}$.
\item $N_{nom}^{s}$: number of samples for the fault-free scenario.
\ei
\end{block}

\end{column}
\end{columns}
%

\begin{columns}
\begin{column}{1.1\textwidth}
\bi
\item Analyze hard faults only.
\item Hard and soft faults together.
\ei
\end{column}
\end{columns}
%
\end{frame}






\begin{frame}
\frametitle{Hard Faults Only}
%
\begin{columns}
\begin{column}{0.6\textwidth}
\bi
% \item clients workload
% \item Hard faults randomly happening during the sampling. 
\item Angular direction $=$ client name. 
\item Data $=$ total number of tasks being handled during the simulation.
\item No-fault case: \\workload is fairly uniform 
% \item Asymmetry becomes increasingly more evident. 
\item As expected, increasing the number of faults causes 
the clients that are alive to handle more and more 
tasks to compensate for those that are dead.
\ei
\end{column}
%
\begin{column}{0.5\textwidth}
\begin{center}
\includegraphics[width=1\textwidth]{clients_load-crop.pdf}\\
{Clients Workload.}
\end{center}
\end{column}
\end{columns}
%
\end{frame}






\begin{frame}
\frametitle{Hard Faults (HF) Only}
%
\begin{columns}
\hspace{-1cm}
\begin{column}{0.33\textwidth}
\begin{center}
{\bf HF Sampling Only}\\
\vspace{0.15cm}
\includegraphics[width=1.28\textwidth]{H_sampling-crop.pdf}
\end{center}
\end{column}
%
\begin{column}{0.33\textwidth}
\begin{center}
{\bf HF Regression Only }\\
\vspace{0.15cm}
\includegraphics[width=1.3\textwidth]{H_regression-crop.pdf}
\end{center}
\end{column}
%
\begin{column}{0.33\textwidth}
\begin{center}
{\bf HF Samp/Regress}\\
\vspace{0.15cm}
\includegraphics[width=1.3\textwidth]{H_both-crop.pdf}
\end{center}
\end{column}
\end{columns}
%
\vspace{0.1cm}
%
\begin{columns}
\begin{column}{1.1\textwidth}
\bi
\item the best case scenario is when all faults affect regression 
because full computational power is available for a longer part of the simulation
\item HF for both: losing $14 \%$, $28 \%$ and $42 \%$ 
of the clients yields, respectively, a total overhead 
of $8 \%$, $19 \%$ and $30 \%$. 
\ei
\end{column}
\end{columns}
%
\end{frame}




\begin{frame}
\frametitle{Hard Faults and SDC}
%
\begin{columns}
\hspace{-1cm}

\begin{column}{0.33\textwidth}
\begin{center}
{\bf 0.25 \%~SDC}\\
\vspace{0.15cm}
\includegraphics[width=1.26\textwidth]{H+Ssoft025-crop.pdf}
\end{center}
\end{column}
%
\begin{column}{0.33\textwidth}
\begin{center}
{\bf 0.5 \%~SDC}\\
\vspace{0.15cm}
\includegraphics[width=1.26\textwidth]{H+Ssoft050-crop.pdf}
\end{center}
\end{column}
%
\begin{column}{0.33\textwidth}
\begin{center}
{\bf 1.0~\%~SDC}\\
\vspace{0.15cm}
\includegraphics[width=1.26\textwidth]{H+Ssoft100-crop.pdf}
\end{center}
\end{column}
\end{columns}
%
\vspace{0.1cm}
%
\begin{columns}
\begin{column}{1.1\textwidth}
\bi
\item Consider $4$ hard faults; \textit{four-fold} increase in 
SDC from $9$ to $33$ causes 
the sampling overhead to only increase from $9 \%$ to about $15 \%$. 
\item Regression overhead only increases from $30 \%$ to about $38 \%$. 
\item This yields the total overhead to 
only increase from $21 \%$ to $28 \%$. 
\ei
\end{column}
\end{columns}
%
\end{frame}





% \begin{frame}
% \frametitle{Results: $7 \%$ Oversampling}
% %
% \begin{columns}
% \begin{column}{0.55\textwidth}
% \begin{center}
% {\bf All-Data Corruption}\\
% \vspace{0.15cm}
% \includegraphics[width=1\textwidth]
% {SamplingFaultsOnly_corruptAllDataSmallRegTol_7percentOversampling-crop.png}
% \end{center}
% \end{column}
% \begin{column}{0.55\textwidth}
% \begin{center}
% {\bf Single Corruption}\\
% \vspace{0.15cm}
% \includegraphics[width=1\textwidth]
% {SamplingFaultsOnly_corruptSingleDataSmallRegTol_7percentOversampling-crop.png}
% \end{center}
% \end{column}
% \end{columns}
% %
% \vspace{0.1cm}
% %
% \begin{columns}
% \begin{column}{1.1\textwidth}
% \bi
% \item Less oversampling yields smaller overehead. 
% \item Overhead is smaller when all the data in a sampling task is corrupted. 
% \item Note: four-fold faults increase $\Rightarrow$ minimal change in the overhead.
% \ei
% \end{column}
% \end{columns}
% %
% % %
% \end{frame}







% \begin{frame}
% \frametitle{Scaling}
% %
% \begin{columns}
% \begin{column}{1.0\textwidth}
% \bi
% \item Edison (NERSC), using native Cray-MPICH.
% \item Elliptic PDE on unit square. 
% \item Fix the number of clients per server and 
% the amount of data owned by each server, while 
% increasing the problem size 
% by adding increasingly more clusters. 

% %N subdomains: $12^2,24^2,48^2,96^2$.
% % \item N cores: $144$, $576$, $2304$, $9216$.
% % \item Grid/subdomain: $\sim 100^2$.
% \ei
% \end{column}
% \end{columns}
% %
% \begin{columns}
% \begin{column}{0.55\textwidth}
% \bi
% \item Subdomains: $12^2,18^2,24^2,30^2,36^2,42^2$.
% \item Sub grid size: $180^2$.
% \item N servers: $16,36,64,100,144,196$.
% \item Num clients/server: $64$.
% \item Size of client: $4$ MPI ranks.
% \ei
% % \begin{center}
% % \includegraphics[width=1\textwidth]{weak}\\
% % {\bf Weak Scaling}
% % \end{center}
% \end{column}
% %
% \begin{column}{0.55\textwidth}
% % \bi
% % \item N subdomains: $12^2,24^2,48^2,96^2$.
% % \item N cores: $144$, $576$, $2304$, $9216$.
% % \item Full Grid: $2501^2$.
% % \ei
% \begin{center}
% \includegraphics[width=1\textwidth]{weak-eps-converted-to.pdf}\\
% {\bf Weak Scaling}
% \end{center}
% \end{column}
% \end{columns}
% %
% \end{frame}


