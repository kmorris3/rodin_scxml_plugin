\documentclass[runningheads,a4paper]{llncs}

% Set TOC depth to 3
%\setcounter{tocdepth}{3}

% Package for change tracking
%\usepackage[disabled]{chgtrk}
\usepackage{chgtrk}
\newCTcontributor{Karla}
\newCTcontributor{Colin}
\newCTcontributor{Rob}
\newCTcontributor{Son}
\newCTcontributor{Michael}

% Package for typesetting AMS Symbols
\usepackage{amssymb}

% Package for typesetting abbreviations
\usepackage{abbrev}


% Package for typesetting diagrams using TikZ
\usepackage{tikz}
\usetikzlibrary{positioning}
%\usepackage{pgf-picture}

% Package for typesetting requirements
%\usepackage[compact]{reqdoc}

% Package for typesetting Event-B mathematical symbols
\usepackage{bsymb}

% Package for typesetting SCXMLREF example models in Event-B
%\usepackage{eventB-SCXMLREF}

% Package for AMS Math
\usepackage{amsmath}

% Package for typesetting URLs
\usepackage{url}
\urldef{\mailsa}\path|{cfs, mjb}@ecs.soton.ac.uk|
\urldef{\mailsa}\path|{knmorri, rob}@sandia.gov|

% package for fancy tables
\usepackage{booktabs}

% Package for highlight TODOs
%\usepackage[disable]{hltodonotes}
%\usepackage[]{hltodonotes}

\newcommand{\keywords}[1]{\par\addvspace\baselineskip
\noindent\keywordname\enspace\ignorespaces#1}

% Package for fancy references
\usepackage{varioref}

% Package for including figures
\usepackage{graphicx}
\usepackage{subcaption}

% Package for sub-floats (e.g. figures)
% \usepackage{subfig}

% Package for including standalone source files
\usepackage{standalone}

% For floating listings
\usepackage{float}
\newfloat{lstfloat}{htbp}{lop}
\floatname{lstfloat}{Listing}
\def\lstfloatautorefname{Listing} % needed for hyperref/auroref

% Package for listings (e.g. JavaScript)
%\usepackage{eventBlistings}

\usepackage{hyperref}
\hypersetup{
  colorlinks=true,
  linkcolor = blue,
  urlcolor=cyan!50!black,
  citecolor=cyan,
}

\usepackage{examplep}
% Package for typesetting Event-B, load this package after all other packages
\usepackage[colour]{lstEventB}

% Listing for XML
\definecolor{dkgreen}{rgb}{0,0.6,0}
\definecolor{gray}{rgb}{0.5,0.5,0.5}
\definecolor{mauve}{rgb}{0.58,0,0.82}
\definecolor{gray}{rgb}{0.4,0.4,0.4}
\definecolor{darkblue}{rgb}{0.0,0.0,0.6}
\definecolor{lightblue}{rgb}{0.0,0.0,0.9}
\definecolor{cyan}{rgb}{0.0,0.6,0.6}
\definecolor{darkred}{rgb}{0.6,0.0,0.0}

\lstset{
  basicstyle=\ttfamily\scriptsize,
%  basicstyle=\ttfamily\footnotesize,
  columns=fullflexible,
  showstringspaces=false,
  numbers=left,                   % where to put the line-numbers
  numberstyle=\tiny\color{gray},  % the style that is used for the line-numbers
  stepnumber=1,
  numbersep=5pt,                  % how far the line-numbers are from the code
  backgroundcolor=\color{white},      % choose the background color. You must add \usepackage{color}
  showspaces=false,               % show spaces adding particular underscores
  showstringspaces=false,         % underline spaces within strings
  showtabs=false,                 % show tabs within strings adding particular underscores
  frame=none,                   % adds a frame around the code
  rulecolor=\color{black},        % if not set, the frame-color may be changed on line-breaks within not-black text (e.g. commens (green here))
  tabsize=2,                      % sets default tabsize to 2 spaces
  captionpos=b,                   % sets the caption-position to bottom
  breaklines=true,                % sets automatic line breaking
  breakatwhitespace=false,        % sets if automatic breaks should only happen at whitespace
  title=\lstname,                   % show the filename of files included with \lstinputlisting;
                                  % also try caption instead of title  
  commentstyle=\color{gray}\upshape
}

\lstdefinelanguage{XML}
{
  morestring=[s][\color{mauve}]{"}{"},
  morestring=[s][\color{black}]{>}{<},
  morecomment=[s]{<?}{?>},
  morecomment=[s][\color{dkgreen}]{<!--}{-->},
  stringstyle=\color{black},
  identifierstyle=\color{lightblue},
  keywordstyle=\color{red},
  morekeywords={xmlns,xsi,noNamespaceSchemaLocation,type,id,x,y,source,target,version,tool,transRef,roleRef,objective,eventually}% list your attributes here
}
% Listing for XML


\begin{document}

\mainmatter  % start of an individual contribution

% first the title is needed
\title{Refinement of Statecharts with Run-to-Completion Semantics}

% a short form should be given in case it is too long for the running head
\titlerunning{Refinement of Statecharts}

% the name(s) of the author(s) follow(s) next
%
\author{ 
K. Morris \inst{1} 
\and C. Snook \inst{2}%\textsuperscript{https://orcid.org/0000-0002-0210-0983} 
\and T.S. Hoang \inst{2}
\and R. Armstrong \inst{1}
\and M. Butler \inst{2} 
}

%  \authorrunning{} has to be set for the shorter version of the authors' names;
% otherwise a warning will be rendered in the running heads. When processed by
% EasyChair, this command is mandatory: a document without \authorrunning
% will be rejected by EasyChair

\authorrunning{K.Morris, C. Snook et al.}

% Institutes for affiliations are also joined by \and,
\institute{
	Sandia National Laboratories, 
	Livermore, California, U.S.A.\\
	\email{\{knmorri,rob\}@sandia.gov}
	\and
	University of Southampton,
	Southampton, United Kingdom\\
	\email{\{cfs,t.s.hoang,mjb\}@soton.ac.uk}\\
}

\maketitle

% Reset all abbreviations
%\resetabbrev

% !TEX root = ../SCXMLREF.tex

\section{Introduction}
\label{sec:introduction}

Formal verification of high-consequence systems requires the analysis
of formal models that capture the properties and functionality of the
system of interest. Although high-consequence controls and systems are
designed to limit complexity, the requirements and consequent proof
obligations tend to increase the complexity of the formal verification.  
Proof obligations for such requirements can be made more tractable using
abstraction/refinement, providing a natural divide and conquer
strategy for controlling complexity.

\Statecharts~\cite{Harel} are often used for high-consequence controls
and other critical systems to provide an unambiguous, executable way
of specifying functional as well as safety, security, and reliability
properties.  While functional properties (usually) can be tested,
safety, security and reliability properties (usually) must be proved
formally.  Here we give a binding from \Statecharts to \EventB~\cite{abrial10:_model_event_b} so that
this type of reasoning can be carried out.  Moreover, hierarchical
encapsulation maps well onto \Statecharts in a way that is not very
different from previous work in \iUMLB~\cite{snook14:_b_statem,Snook2006,Snook12:FMCO}, a diagrammatic modelling notation for \EventB.
Binding \iUMLB to a UML~\cite{Rumbaugh2004} version of \Statecharts is natural and the
addition of run-to-completion semantics, expected by \Statechart
designers, is much of the contribution of this work.  Another
contribution is the augmentation of the textual and parse-able format
for \Statecharts, \SCXML to accommodate elements necessary to support formal
analysis. 

There are many formal semantics that can be bound to 
 the \Statechart graphical language~\cite{Eshuis_2009}. While \Statecharts and various semantic interpretations of
\Statecharts admit refinement reified as both hierarchical or parallel
composition (e.g. see Argos~\cite{Maraninchi91theargos}), here, as
previously~\cite{snook14:_b_statem}, we focus only on hierarchical
refinement, the form that \EventB natively admits.  Here we define
hierarchical composition to mean nesting new transition systems inside
previously pure states, and parallel composition to be the combination
in one machine of formerly separate transition systems.
A hierarchical development of a system model uses refinement
concepts to link the different levels of abstraction. Each subsequent
level increases model complexity by adding details in the form of
functionality and implementation method. As the model complexity
increases in each refinement level, tractability of the detailed model
can be improved by the use of a graphical representation, with rich
semantics that can support an infrastructure for formal verification.


The semantics adopted here adheres closely to UML \Statecharts~\cite{Alexandre} and is implemented in \iUMLB.
Models described in \Statecharts are expressed in \SCXML and translated into \EventB logic which uses the \Rodin~\cite{abrial10:_rodin} for machine proofs.
With suitable restrictions, \Statecharts already provide a sound, intuitive, visual metaphor for refinement. 
Outfitted with a formal semantics, this work borrows from well-used \Statechart practices in digital design.  
We previously reported~\cite{Morris_2016} our early attempts to relate \Statecharts to \EventB. 
\ColinAdd{
	At that stage (and similarly in\mbox{~\cite{Snook12:FMCO}}) we suggested the necessary extensions and basic mechanism of translation but, through restrictions and abstraction, avoided the more challenging problem of refinement with run to completion semantics. 
} 
The goal of the present work is to provide usable, well-founded tools that are familiar to designers of high-consequence systems and yet provide the currently lacking formal guarantees needed to ensure safety, security, and reliability.

 % UML \Statechart semantics are not the only formal
 % semantics that can be bound to the \Statechart graphical
 % language~\cite{Eshuis_2009}.  In Statecharts every triggering signal
 % can cause transitions that emit other triggers in a cascade.
 % Different semantic interpretations of Statecharts resolve these
 % cascades differently.  
 % Argos, for
 % example, views cascading transitions as instantaneous and
 % simultaneous rather than the queue-based semantics adopted here.
%
% The \EventB modelling method provides the logic and refinement
% theory required to formally analyse a system model.  The open-source
% \Rodin provides support for \EventB
% including automatic theorem provers.  \iUMLB
% augments the \EventB language with a graphical interface including
% state-machines.  

% With suitable restrictions, \Statecharts already provide a sound,
% intuitive, visual metaphor for refinement. Outfitted with a formal
% semantics, this work borrows from well-used \Statechart practices in
% digital design.  We previously reported~\cite{Morris_2016} our early 
% attempts to relate \Statecharts to \EventB. The goal of the present 
% work is to provide usable,well-founded tools that are familiar to 
% designers of high-consequence systems and yet provide the currently 
% lacking formal guarantees needed to ensure safety, security, and reliability.

% We previously reported~\cite{Morris_2016} our early attempts to relate \Statecharts to \EventB. At that stage we had tried some aspects of the translation by using simplifications and we were beginning to gain insights into the problem, but had not arrived at the translation we now use.

The rest of the paper is structured as follows.  Section~\ref{sec:background} provides background information on \SCXML, \EventB, and \iUMLB.  Section~\ref{sec:secbot} presents our running example.  Section~\ref{sec:discussion} discusses the various challenges for introducing a refinement notion into \SCXML and demonstrates our approach.  Section~\ref{sec:extensions} shows our extensions to \SCXML which are necessary for reasoning about properties of \SCXML models.  In Section~\ref{sec:translation}, we illustrate our translation of \SCXML models into \EventB using the example introduced in Section~\ref{sec:secbot}.  Section~\ref{sec:example} shows how properties of the \SCXML models can be specified as invariants and verified in \EventB.  We summarise our contribution and conclude in Section~\ref{sec:conclusion}.

%%% Local Variables: 
%%% mode: latex
%%% TeX-master: "../SCXMLREF.tex"
%%% End: 



%\end{scriptsize}

\end{document}

%%% Local Variables: 
%%% mode: latex
%%% TeX-master: t
%%% End: 
