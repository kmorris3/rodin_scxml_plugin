% !TEX root = ../SCXMLREF.tex

\section{Introduction}
\label{sec:introduction}

\subsection{Signature}
A (UML-B) state-machine is a tuple |SM, S, T, r, o, c, i| where
\SonInlineComment{This is just the ``syntax'' of state-machines. It is (necessarily) incomplete and missing several constraints, e.g., the owner and contain relationship should form tree-like relationship between state-machines}
\begin{itemize}
\item |SM| is a set of state-machines,

\item |S| is the set of states,

\item |T !: S <-> S| is the set of transitions between states,

\item |r !: SM| is the root state machine,

\item |o !: S --> SM| is the ownership relationship between states and state-machines,

\item |c !: SM \ {r} --> S| is the containment relationship between nested state-machines (i.e., except the root machine) and their containing state.

\item |i !: !POW(S)| is the initial states.
\SonInlineComment{Are these different from the states that reach from the INITIALISATION}
\end{itemize}

\subsection{Semantics}
Each UML-B state-machine |SM, S, T, r, o, c, i| corresponding to a transition system |States, trns|.
\begin{itemize}
\item |States <: SM +-> S|: The state is an evaluation of statemachine to the current state of that state-machine.
  \SonInlineComment{There are some well-definedness condition here}
  \begin{itemize}
  \item |!sm, State . State !: States & sm !: dom(State) => o(State(sm)) !: sm| 
  \end{itemize}

\item |trns : States ** T +-> States|: Label transition from states to states.
\end{itemize}
\SonInlineComment{We should define how to map transitions \EventBinline{T} to label transition \EventBinline{trns}}.

%%% Local Variables: 
%%% mode: latex
%%% TeX-master: "../SCXMLREF.tex"
%%% End: 
