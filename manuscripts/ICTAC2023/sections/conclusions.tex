\section{Conclusions}
\label{sec:conclusions}

In this paper, we formalize the semantics of SCXML run-to-completion statecharts using \EventB. We formalize the syntactic elements using \EventB contexts and using \EventB machines to model the dynamic semantics.  Moreover the semantics model is built in a compositional fashion: the semantics of (untriggered) statechart and run-to-completion schedule is developed independently and composed to create the SCXML statechart's semantics model. This approach allows us to reduce the complexity on the consistency reasoning by focusing on the different parts of the models. The combined model inherits the consistency of the sub-models by construction.

In order to ensure the consistency of the semantics, several well-definedness conditions on the syntactic elements have been identified. They are encoded as axioms in the formal models. These well-definedness conditions can be used as the specification of a validation tool to ensure the consistency of SCXML models. Given the semantic model are consistent, any instantiation will inherit this consistency without the need for reproving. For instance, the model of the turnstile example can have consistency about the active states and the triggering mechanism. Often, these consistency checks make up the majority of the proof  obligations, and only a small number are related to the specific properties of the model.

% Future work
We plan to extend this work to  accommodate for the semantics of refinement as described in~\cite{Morris2018,Morris2020}. In particular, we will formalise the syntactic constraints which allow the consistent refinement of SCXML run-to-completion statecharts, proving the consistency of the refinement rules, e.g., in \cite{DBLP:journals/isse/MorrisSHHAB22}.  The consistency of the semantic models focuses on safety properties, expressed as invariants. Furthermore, we model some of the syntactic elements in our formal models at a fairly abstract level, e.g., the notion of |enabling|, |exiting|, |entering| states for transformation.  This means that the model can be applied to different statechart notations, e.g. UML-B~\cite{DBLP:conf/sefm/SnookBHFD22}.
%At the moment, our semantics focuses on the statemachine structure and the run-to-completion mechanism. We will extend this semantics model to cover other aspects of the notation such as data items.