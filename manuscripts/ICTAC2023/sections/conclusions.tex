\section{Conclusions}
\label{sec:conclusions}

We formalize the semantics of SCXML run-to-completion statecharts using \EventB. We formalize the syntactic elements using \EventB contexts and dynamic semantics using \EventB machines.  The semantics model is built in a compositional fashion: the semantics of (untriggered) statecharts and run-to-completion schedule is developed independently and composed to create the SCXML statechart's semantics model. This approach allows us to reduce the complexity of the consistency reasoning by focusing on different parts of the models. The combined model inherits the consistency of the sub-models by construction.

In order to ensure the consistency of the semantics, several well-definedness conditions on the syntactic elements have been identified. They are encoded as axioms in the formal models. These well-definedness conditions can be used as the specification of a validation tool to ensure the consistency of SCXML models. Given the semantic models are consistent, any instantiation will inherit this consistency without the need for reproving. For instance, the model of the turnstile example can have consistency about the active states and the triggering mechanism. Often, these consistency checks make up the majority of the proof  obligations, only a small number are related to the specific model properties.

% Future work
We plan to extend this work to formalize the semantics of refinement as described in~\cite{Morris2018,Morris2020}. In particular, we will formalise the syntactic constraints that ensure consistent refinement of SCXML statecharts, proving the consistency of the refinement rules, e.g., in \cite{DBLP:journals/isse/MorrisSHHAB22}.  The consistency of the semantic models focuses on safety properties, expressed as invariants. Furthermore, we model some of the syntactic elements in our formal models at a fairly abstract level, e.g., the notion of |enabling|, |exiting|, |entering| states for transformation. 
Our abstract semantics supports the majority of typical statechart features such as transitions, hierarchical structure, clustering, concurency, start and stop states. 
We do not cover history or timeout mechanisms. 
Our composition approach means that our untriggered statechart semantics, which is common to most statechart notations, can be reused regardless of their triggering semantics (or lack thereof).
E.g. UML-B~\cite{DBLP:conf/sefm/SnookBHFD22} is untriggered.
Furthermore, since SCXML is based on the widely used Harel statechart semantics~\cite{HAREL1987231}, our run to completion semantics can also be used for such notations and where notations deviate in their run semantics (e.g.~\cite{Eshuis_2009}), we at least encapsulate the extent of re-work required.  
%At the moment, our semantics focuses on the statemachine structure and the run-to-completion mechanism. We will extend this semantics model to cover other aspects of the notation such as data items.