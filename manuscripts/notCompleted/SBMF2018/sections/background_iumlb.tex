% !TEX root = ../SCXMLREF.tex

\subsection{iUML-B State-machines}
\label{sec:iumlb}

\iUMLB provides a diagrammatic modelling notation for \EventB in the form of state-machines and class diagrams. 
The diagrammatic models are contained within an \EventB machine and generate or contribute to parts of it. 
For example a state-machine will automatically generate the \EventB data elements (sets, constants, axioms, variables, and invariants) to implement the states while \EventB events are expected to already exist to represent the transitions. 
Transitions contribute further guards and actions representing their state change, to the events that they elaborate.  
State-machines are typically refined by adding nested state-machines to states.
Figure~\ref{fig:iumlb-sm} shows an example of a simple state-machine with two states.
\begin{figure}[!htbp]
	\centering
	\includegraphics[width=0.6\textwidth]{figures/iumlb-SM}
	\caption{An example \iUMLB state-machine}
	\label{fig:iumlb-sm}
\end{figure}

Each state is encoded as a boolean variable and the current state is indicated by one of the boolean variables being set to |TRUE|. 
An invariant ensures that only one state is set to |TRUE| at a time.
%The state-machine, is initialised by setting one state variable to |TRUE| and all others to |FALSE|.
Events change the values of state variables to move the |TRUE| value according to the transitions in the state-machine.  
The \EventB translation%
%
\footnote{%
  Here, $\mathrm{partition(S, T1, T2, \ldots)}$ means the set $S$ is partitioned into disjoint (sub-)sets $T1, T2, \ldots$.
that cover $S$} %
of the state-machine in Figure~\ref{fig:iumlb-sm} can be seen in Listing~\ref{lst:eventb-sm}.%
\begin{lstlisting}[caption={Translation of the state-machine in Fig.~\ref{fig:iumlb-sm}},label={lst:eventb-sm}, language=Event-B, escapechar=|, frame=single, float=t]
 variables S1 S2
invariants 
	TRUE !: {S1, S2} => partition({TRUE}, {S1}/\{TRUE}, {S2}/\{TRUE})
events
    INITIALISATION: begin S1, S2 := TRUE, FALSE end
    e: when S1 = TRUE then S1, S2 ≔ FALSE, TRUE  end
    f: when S2 = TRUE then S2 := FALSE end
end
\end{lstlisting}	
	
%%% Local Variables:
%%% mode: latex
%%% TeX-master: "../SCXMLREF"
%%% End:
