\section{The semantics of refining SCXML models}
\ColinInlineComment{This is some old text.. needs updating for our latest approach}

To model the complete \SCXML refinement semantics we now combine the previous two models. 
That is, using the inclusion mechanism built into CamilleX, we combine our models of scxml execution with our models for state-machines.

\subsection{M1 - refinement of combined scxml and state-machine }
For this stage we include the corresponding M1 stages for SCXML and state-machine.

The gluing context defines the needed refinement relationship between concrete and abstract parts of the model.
\begin{enumerate}
	\item \label{item:triggeredness} For transitions that refine abstract transitions, the transition is triggered if and only if its corresponding abstract transition is triggered.
	\item For \emph{triggered} transitions that refine abstract transitions, the trigger is the same as that of the corresponding abstract transition.
	\item For \emph{finalised} transitions that refine abstract \emph{finalised} transitions, the transition source is the same as that of the corresponding abstract transition. I.e. you cannot strengthen the source of an already finalised transition.
	\item the abstract \emph{finalised} transitions are a subset of the abstract transitions that correspond to concrete finalised transitions. I.e. once transitions are finalised they stay finalised in refinements. 
\end{enumerate}
Initially, for item \ref{item:triggeredness} we defined a single axiom giving the equality of the domains of the abstract and concrete transition-trigger relationship.
However, this was insufficient to prove the guard strengthening of the untriggered refined transitions because the |transition_link| relationship is not injective.
(I.e. in general, an abstract transition could be refined by more than one concrete transition)
Instead, we defined two axioms, universally quantifying over the set of refined transitions that are/are not triggered, with the consequent that the corresponding abstract transitions are/are not triggered.
These axioms were sufficient to automatically discharge the guard strengthening POs for both triggered and untriggered refined transitions.

\subsection{M2 - merging old and new events}
We found a problem when we tried to merge the refined and new  transitions of the combined scxml and statechart semantics. 
The problem occurs for merging both triggered and untriggered transitions.
The problem is that we have not considered the case of merging a new state-machine transition with an old refined scxml triggered transition.
Probably this is nonsensical and we should add guards to prevent it.
Ok for triggered but for untriggered our way of distinguishing refined scxml untriggered transitions is via the disappearing variable dqaux
