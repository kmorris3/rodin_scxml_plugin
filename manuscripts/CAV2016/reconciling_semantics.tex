\section{Reconciling SCXML and iUML-B Semantics}
\label{sect:recon}

The SCXML basic structure of states with nested 
statemachines and transitions is the same as that 
in iUML-B, but there are several semantic 
differences that make translation into iUML-B difficult. 
This sections provides a description of some of the
main differences, as well as possible solutions for their
reconciliation.

\begin{description}
\item [Refinement:]
Refinement is a central concept in iUML-B, detail is 
added in refinements by progressive hierarchical 
nesting. There is no refinement in SCXML. Features
in the diagrams have to be restricted to guarantee
that the models are correct refinements. An example,
of such a restriction is not allowing for actions in 
both parent and nested states to change the same varaible.

\item [Events:]
The meaning of event is very different between iUML-B 
and SCXML. In iUML-B transitions are sub-parts of 
events. In order for an event to be enabled for firing, 
all of its sub-parts (transitions) must be 
simultaneously enabled. This means that two different 
transitions with the same event can only fire at the 
same time and hence will never fire if they are sourced 
from different states of the same parent state-machine. 
In SCXML, events are triggers that enable transitions 
to fire. If two different transitions from different 
source states are both triggered by the same event, one 
may fire without the other if one source state is not 
active.

\item [Entry and Exit Actions:]
SCXML includes the concept of entry and exit actions 
which are executed whenever a transition enters and 
exits respectively, the containing state. 
iUML-B has been extended to support the construction 
of entry/exit actions.

The addition of these feature is straight forward,
however, the lack of sequential composition in 
Event-B (hence iUML-B) means that the semantics of entry/
exit actions will differ in some scenarios. That is, in 
SCXML the source state’s exit actions are taken before the 
transition’s actions, which are before the target state’s 
entry actions. In iUML-B all the actions are taken in 
parallel, as there is no concept of execution order within 
an event. 

We restrict SCXML so that the actions are parallelisable. 
Effectively this means that the same variable cannot be 
assigned more than once in any set of actions that will be 
taken when a transition fires. The Event-B static checker 
would then raise an error if the same variable is assigned 
in for example, the source states exit actions and the 
target states entry actions. 

% One difficulty arises when a transition exits a parent 
% state without specifying a particular nested sub-state. 
% Strictly, only the exit actions of the currently active sub-
% state should be executed. However, this would be difficult 
% in iUML-B due to the lack of any conditional execution.


\item [Transition firing:]
The hierarchical state constructs of both SCXML and
iUML-B are equivalent, but their transition 
firing mechanism differs significantly. In iUML-B 
transitions fire spontaneously when their guard and source 
state are true. In SCXML, there are two kinds of transitions, 
‘When’ transitions can fire spontaneously as soon as their 
guard becomes true. This is the same as iUML-B transitions 
and we have no problem translating these. Other SCXML
transitions are triggered by the 
occurrence of some other event, which may be external 
or internal (induced by the actions of another transition). 
In iUML-B if several transitions are simultaneously 
enabled one of the enabled transitions is non-deterministically 
chosen for firing whereas SCXML has ordering rules to 
determine which transitions to fire next.

Thigger events could be simulated by generating a flag to 
represent the trigger and adding a guard on the trigger 
flag to the transitions that are triggered by it. The flag 
should then be reset by whichever transition is triggered 
by it in order to ‘consume’ that trigger event. A special 
interface event that sets the flag would be generated to 
represent the external interface receiving a trigger.

Transitions may also trigger each other. This could be 
modelled by a similar mechanism except that the interface 
event is not needed since the flag is set directly by 
another transition.

% \item [Transition execution:]
% When a particular SCXML transition fires it carries out 
% a sequence of actions in particular order. For example, 
% a hierarchy of nested source states exit,
% performing exit actions starting from the innermost 
% one and working outwards. The order of the actions 
% is significant in SCXML as these actions could 
% write to the same variable. In Event-B all actions of a 
% transition are executed simultaneouslly within the elaborated 
% event. It is not possible (i.e. not well-formed) for two of these 
% actions to write to the same variable.
% SCXML transitions can be designated ‘internal’, which 
% prevents exiting and re-entering its source state in 
% some cases. In SCXML target state can be omitted which 
% results in a transition that does not change state (
% this is different from a transition that exits a state 
% and then re-enters the same state).
% Neither of these features is supported in iUML-B since 

\item [Run to completion semantics:] SCXML has a run-to-completion 
(or big-step/little-step) semantics. This means that an external 
trigger is only consumed when no transition can be taken without 
doing so. This is quite cumbersome to implement in iUML-B since 
it requires constructing the conjunction of the negated guards of
all the transitions that are internally triggered (including when 
transitions) and adding this to all externally triggered transitions.

\item [Final States:]
The concept of a final state differs between iUML-B and 
SCXML. In SCXML a state machine (or parent state) may 
reside in a final state indicating that it is done and 
waiting for another transition to exit the parent 
state.  In iUML-B a final state is not a proper state 
of the parent state-machine. It is merely a notation 
for indicating that the state-machine is becoming non-
active. I.e. that the parent state is exiting. Hence 
any transitions that target a final state are part of a 
transition that leaves the parent state. For a ‘root’ 
state-machine the final state means that the state-
machine has been left completely and no state is active.

\item [Initial States:]
Initial states are similar to iUML-B. The transition 
from the initial state forms part of the actions to 
enter the parent state. However, the correspondence 
between incoming transitions to the parent state and 
initial transitions is more explicit in iUML-B. 
SCXML has another way to specify an initial state using 
an attribute of the state. In this case there is no way 
to add extra transition actions.
If no initial state is specified, the default is the 
first one in the document.
iUML-B allows different intial states for different 
incoming transitions. In SCXML this would be done by 
extending the transition into the substate.



\end{description}



