% !TEX root = main.tex

\section{Translation Tool}

An EMF meta-model for SCXML is available from the Sirius 
project. It is used by the Sirius project as a test sample 
so has not been released to users but is available in the 
github source code repository and will be copied and built 
for release with the iUML-B-SCXML tools. It supports SCXML 
functionality including data modelling and action/
expression language. 
Running this code in a debug environment enables us to 
produce the following SCXML model using the EMF sample 
editor. The serialised XMI is shown to the left. 

The serialisation shown above is the default produced for a 
newly created model. It contains the prefix tag ‘scxml:’ on 
each element, which is not standard scxml syntax. However, 
the serialisation is flexible and is able to load/update 
models without this prefix being introduced.

Hierarchical nested state charts are translated to similar 
corresponding state-machine structures in iUML-B according 
to the rules described in 5.2.1. There are two alternative 
styles of Event-B representation for iUML-B state-machines. 
Currently the state-variables style is adopted because it 
is simpler to translate from the Scxml model. However, the 
alternative state-enumeration style has benefits in user-
readability and may be supported in future. This would 
require conventions regarding the name of the state-machine 
to be adopted and used by the modeller in order to 
construct guards that refer to the current value of the 
state-machine.

\subsection{Refinement Levels}
For scxml:State the refinement level refers to any state 
machines generated from the children nested in the state (i.
e. generated iUML-B states must be all based on the 
refinement level of the containing statemachine).
For iumlb:invariants the generated invariant is only 
generated at the specified refinement level, not in 
subsequent refinements.

\subsection{Constructing event elaborated by transitions}
The Event-B events that are elaborated by an iUML-B 
transition are constructed as follows:

The event names are obtained (cumulatively) by the 
following methods:
a)  the transition has iumlb:label attributes,
b) the transition's source is an initial state (see below), 
or,
c) if none of the above provide any labels, a default 
'source\_target' format is used.

Note that trigger events are deliberately not used for 
transition events because we want to keep them as a 
separate concept from transition firing in line with scxml 
semantics.

If the transition is in an initial state at the outer state 
chart level the name is INITIALISATION. 
If the transition is in an initial state of a nested state 
chart the names of all the events that are associated with 
incoming transitions to the parent state are used.

\subsection{Data elements}
Data elements (which are collated in Datamodel elements) 
are used to model ancillary variables as part of the model 
in the way that SCXML intends them to be used.

Data elements are interpreted as a variable. The type of 
the variable is given in an iumlb:type attribute. 
The id attribute is interpreted as the name of the 
variable. The value is used as the right hand side of an 
assignment action to initialise the variable.  (Some syntax 
conversion is performed to convert the predicate from SCXML 
format into Event-B mathematical language). The variable is 
introduced at the same refinement level as the parent 
element that contains it.


%------------------------------------------------------------------------------

%%% Local Variables: 
%%% mode: latex
%%% TeX-master: "HDMachine"
%%% End: 
