% easychair.tex,v 3.4 2015/12/10

\documentclass{easychair}
%\documentclass[EPiC]{easychair}
%\documentclass[debug]{easychair}
%\documentclass[verbose]{easychair}
%\documentclass[notimes]{easychair}
%\documentclass[withtimes]{easychair}
%\documentclass[a4paper]{easychair}
%\documentclass[letterpaper]{easychair}
\newcommand{\specialcell}[2][c]{%
  \begin{tabular}[#1]{@{}c@{}}#2\end{tabular}}

\usepackage{doc}

%use packages added by karla
\usepackage[utf8]{inputenc}
\usepackage[english]{babel}
\usepackage{color}
\usepackage{amsmath}
\usepackage{setspace}
\usepackage{fancyvrb,relsize}
\usepackage{aeguill}
\usepackage{rotating}
\usepackage{amssymb}
\usepackage{color}
\usepackage{amsbsy}
\usepackage{psfrag}
\usepackage{soul}
\usepackage{relsize}
\usepackage{verbatim}
%use packages added by karla

%use packages added by Colin
 \usepackage{multirow}
 \usepackage{graphicx}
 %use packages added by Colin

% use this if you have a long article and want to create an index
% \usepackage{makeidx}

% In order to save space or manage large tables or figures in a
% landcape-like text, you can use the rotating and pdflscape
% packages. Uncomment the desired from the below.
%
% \usepackage{rotating}
% \usepackage{pdflscape}

% Some of our commands for this guide.
%
\newcommand{\easychair}{\textsf{easychair}}
\newcommand{\miktex}{MiK{\TeX}}
\newcommand{\texniccenter}{{\TeX}nicCenter}
\newcommand{\makefile}{\texttt{Makefile}}
\newcommand{\latexeditor}{LEd}

%\makeindex

%% Front Matter
%%
% Regular title as in the article class.
%
% \title{The {\easychair} Class File\\
%        Documentation and Guide for Authors%
% \thanks{Other people who contributed to this document include Maria Voronkov
%   (Imperial College and EasyChair) and Graham Gough (The University of
%   Manchester).}}

% \title{Reconciling SCXML and Event-B Semantics}
% \title{Generating Event-B from a Statechart Representation}
% \title{The use of SCXML to bridge Event-B and Higher level represetations}
\title{Reconciling SCXML Statechart Representations and Event-B Lower Level Semantics}
% \title{Statechart representation based on SCXML}

% Authors are joined by \and. Their affiliations are given by \inst, which indexes
% into the list defined using \institute
%
% \author{
% Serguei A. Mokhov\inst{1}\thanks{Designed and implemented the class style}
% \and
%     Geoff Sutcliffe\inst{2}\thanks{Did numerous tests and provided a lot of suggestions}
% \and
%    Andrei Voronkov\inst{3}\inst{4}\inst{5}\thanks{Masterminded EasyChair and created versions
%      3.0--3.4 of the class style}
% }

\author{
Karla Morris\inst{1}
\and
Colin Snook\inst{2}
}

% Institutes for affiliations are also joined by \and,
\institute{
  Sandia National Laboratories, 
  Livermore, California, U.S.A.\\
  \email{knmorri@sandia.gov}
\and
   University of Southampton,
   Southampton, United Kingdom\\
   \email{cfs@ecs.soton.ac.uk}\\
 }

%  \authorrunning{} has to be set for the shorter version of the authors' names;
% otherwise a warning will be rendered in the running heads. When processed by
% EasyChair, this command is mandatory: a document without \authorrunning
% will be rejected by EasyChair

\authorrunning{}

% \titlerunning{} has to be set to either the main title or its shorter
% version for the running heads. When processed by
% EasyChair, this command is mandatory: a document without \titlerunning
% will be rejected by EasyChair

\titlerunning{}

\begin{document}

\maketitle

\begin{abstract}
  BLA BLA 
\end{abstract}

% The table of contents below is added for your convenience. Please do not use
% the table of contents if you are preparing your paper for publication in the
% EPiC series

% \setcounter{tocdepth}{2}
% {\small
% \tableofcontents}

%\section{To mention}
%
%Processing in EasyChair - number of pages.
%
%Examples of how EasyChair processes papers. Caveats (replacement of EC
%class, errors).

\pagestyle{empty}

%------------------------------------------------------------------------------
\section{Introduction}
\label{sect:introduction}


% \begin{figure}[tb]
% 	\begin{centering}
% 	\includegraphics[width=0.5\textwidth]{logoEC}
% 	\caption{EasyChair logo}
% 	\label{fig:easychair-logo}
% 	\end{centering}
% \end{figure}

% \textcolor{red}{This section should focus on the motivation behind pursuing a scxml 
% representation of our models, what are the benefits?}

The formal verification of high consequence systems 
requires the analysis of formal models that capture 
the properties and functionality of the system of 
interest. Discharging proof obligations for systems' 
properties or requirements can be made more tracktable 
depending on the abstraction use to create the model, 
as properties are expressed in term of variables that 
are relevant at different abstraction levels.  

A herarchical development of a system model makes 
use of refinement concepts to link the different levels
of abstraction. Each subsequent level increases model 
complexity by adding implementation details to the 
model in the form of functionality, capabilities and 
finner requirements. As the model complexity increases 
in each refinement level tractability of the model 
can be improve by the use of a graphical representation, 
with rich semantics that can support an infrastructure 
for formal verification.

The Event-B language provides the logic, and refinement
theory require to formaly analyze a system model. The 
open-source Rodin tool auments the Event-B language by 
providing a graphical interphace in the form of
iUML-B. The goal of this work is to create a unified model 
representation capable of leveraging the structure and 
herarchy that is inherently part of a statechart 
diagram, which will serve to enable the formal verification
of requirements from two different  perspective. First, 
the translation of the unified representation to Event-B. Second,
the analysis of requirements related to the structure of 
the statechart it self, which is a higher level representation 
of the model. 

We base this unified statechart model representaiton 
on SCXML.  This is a general-purpose event-based state machine 
language that combines concepts from CCXML and Harel 
State Tables. Harel State Tables are included in UML. 
The concrete syntax for SCXML\footnote{http://www.w3.org/TR/scxml/} 
is based on XML. Hence, SCXML is an XML notation for 
UML style state-machines extended with an action 
language that is intended for call control features 
in voice applications.

An example of SCXML syntax is shown in Figure \ref{fig:scxml}. 
A graphical representation of this example is shown in Figure  \ref{fig:StatemachineSCXML}. 
We use this example to illustrate points throughout the paper.

%------------------------------------------------------------------------------

% \section{Reconciling SCXML and iUML-B Semantics}
\label{sect:recon}

The SCXML basic structure of states with nested 
statemachines and transitions is the same as that 
in iUML-B, but there are several semantic 
differences that make translation into iUML-B difficult. 
This sections provides a description of some of the
main differences, as well as possible solutions for their
reconciliation.

\begin{description}
\item [Refinement:]
Refinement is a central concept in iUML-B, detail is 
added in refinements by progressive hierarchical 
nesting. There is no refinement in SCXML. Features
in the diagrams have to be restricted to guarantee
that the models are correct refinements. An example,
of such a restriction is not allowing for actions in 
both parent and nested states to change the same varaible.

\item [Events:]
The meaning of event is very different between iUML-B 
and SCXML. In iUML-B transitions are sub-parts of 
events. In order for an event to be enabled for firing, 
all of its sub-parts (transitions) must be 
simultaneously enabled. This means that two different 
transitions with the same event can only fire at the 
same time and hence will never fire if they are sourced 
from different states of the same parent state-machine. 
In SCXML, events are triggers that enable transitions 
to fire. If two different transitions from different 
source states are both triggered by the same event, one 
may fire without the other if one source state is not 
active.

\item [Entry and Exit Actions:]
SCXML includes the concept of entry and exit actions 
which are executed whenever a transition enters and 
exits respectively, the containing state. 
iUML-B has been extended to support the construction 
of entry/exit actions.

The addition of these feature is straight forward,
however, the lack of sequential composition in 
Event-B (hence iUML-B) means that the semantics of entry/
exit actions will differ in some scenarios. That is, in 
SCXML the source state’s exit actions are taken before the 
transition’s actions, which are before the target state’s 
entry actions. In iUML-B all the actions are taken in 
parallel, as there is no concept of execution order within 
an event. 

We restrict SCXML so that the actions are parallelisable. 
Effectively this means that the same variable cannot be 
assigned more than once in any set of actions that will be 
taken when a transition fires. The Event-B static checker 
would then raise an error if the same variable is assigned 
in for example, the source states exit actions and the 
target states entry actions. 

% One difficulty arises when a transition exits a parent 
% state without specifying a particular nested sub-state. 
% Strictly, only the exit actions of the currently active sub-
% state should be executed. However, this would be difficult 
% in iUML-B due to the lack of any conditional execution.


\item [Transition firing:]
The hierarchical state constructs of both SCXML and
iUML-B are equivalent, but their transition 
firing mechanism differs significantly. In iUML-B 
transitions fire spontaneously when their guard and source 
state are true. In SCXML, there are two kinds of transitions, 
‘When’ transitions can fire spontaneously as soon as their 
guard becomes true. This is the same as iUML-B transitions 
and we have no problem translating these. Other SCXML
transitions are triggered by the 
occurrence of some other event, which may be external 
or internal (induced by the actions of another transition). 
In iUML-B if several transitions are simultaneously 
enabled one of the enabled transitions is non-deterministically 
chosen for firing whereas SCXML has ordering rules to 
determine which transitions to fire next.

Thigger events could be simulated by generating a flag to 
represent the trigger and adding a guard on the trigger 
flag to the transitions that are triggered by it. The flag 
should then be reset by whichever transition is triggered 
by it in order to ‘consume’ that trigger event. A special 
interface event that sets the flag would be generated to 
represent the external interface receiving a trigger.

Transitions may also trigger each other. This could be 
modelled by a similar mechanism except that the interface 
event is not needed since the flag is set directly by 
another transition.

% \item [Transition execution:]
% When a particular SCXML transition fires it carries out 
% a sequence of actions in particular order. For example, 
% a hierarchy of nested source states exit,
% performing exit actions starting from the innermost 
% one and working outwards. The order of the actions 
% is significant in SCXML as these actions could 
% write to the same variable. In Event-B all actions of a 
% transition are executed simultaneouslly within the elaborated 
% event. It is not possible (i.e. not well-formed) for two of these 
% actions to write to the same variable.
% SCXML transitions can be designated ‘internal’, which 
% prevents exiting and re-entering its source state in 
% some cases. In SCXML target state can be omitted which 
% results in a transition that does not change state (
% this is different from a transition that exits a state 
% and then re-enters the same state).
% Neither of these features is supported in iUML-B since 

\item [Run to completion semantics:] SCXML has a run-to-completion 
(or big-step/little-step) semantics. This means that an external 
trigger is only consumed when no transition can be taken without 
doing so. This is quite cumbersome to implement in iUML-B since 
it requires constructing the conjunction of the negated guards of
all the transitions that are internally triggered (including when 
transitions) and adding this to all externally triggered transitions.

\item [Final States:]
The concept of a final state differs between iUML-B and 
SCXML. In SCXML a state machine (or parent state) may 
reside in a final state indicating that it is done and 
waiting for another transition to exit the parent 
state.  In iUML-B a final state is not a proper state 
of the parent state-machine. It is merely a notation 
for indicating that the state-machine is becoming non-
active. I.e. that the parent state is exiting. Hence 
any transitions that target a final state are part of a 
transition that leaves the parent state. For a ‘root’ 
state-machine the final state means that the state-
machine has been left completely and no state is active.

\item [Initial States:]
Initial states are similar to iUML-B. The transition 
from the initial state forms part of the actions to 
enter the parent state. However, the correspondence 
between incoming transitions to the parent state and 
initial transitions is more explicit in iUML-B. 
SCXML has another way to specify an initial state using 
an attribute of the state. In this case there is no way 
to add extra transition actions.
If no initial state is specified, the default is the 
first one in the document.
iUML-B allows different intial states for different 
incoming transitions. In SCXML this would be done by 
extending the transition into the substate.



\end{description}




% !TEX root = main.tex

\section{Semantic Differences and their Reconciliation}
\label{sect:diff}

\textcolor{red}{Differences in semantics between scxml and event-b}
\textcolor{red}{Elude to the reason for ignoring some of the scxml feature 
when it comes to the translation.}

\textcolor{red}{List the difference in the syntax of each representation (Short section)}

SCXML and iUML-B have syntactic similarities in their use of a state-transition notation with conditional transitions applying actions upon ancillary variables.
However, their semantics have significant differences. SCXML is based on Harel state-charts with so called, `run to completion' semantics, whereas iUML-B is designed upon the `guarded action' semantics of Event-B.
In this section we discuss the implications of this semantic difference with respect to translation between the 2 notations.

\begin{description}

\item [Transition firing:]

There are three methods of initiating transitions in SCXML:
\begin{itemize}
\item `When' transitions are considered for execution if their source state is active and their \emph{cond} attribute evaluates to true. 
This is similar to iUML-B transitions, which fire spontaneously when their guard (including source state) is true.
In iUML-B if several transitions are simultaneously enabled one of the enabled transitions is non-deterministically chosen for firing whereas 
SCXML has ordering rules to determine which transitions to fire next.
\item A transition may be `Triggered' by the occurrence an external interface event. 
This could be simulated in iUML-B by generating a flag to represent the trigger and adding a guard on the trigger flag to the transitions that are triggered by it. 
The flag should then be reset by whichever transition is triggered by it in order to `consume' that trigger event.
 A special interface event that sets the flag would be generated to represent the external interface receiving a trigger.
\item Transitions may also trigger each other within their actions. 
This could be modelled by a similar mechanism to the trigger events except that the interface event is not needed since the flag is set directly by another transition.
\end{itemize}

\item [Run to completion semantics:] 
SCXML has a run-to-completion (aka big-step/little-step) semantics.
 This means that an external trigger is only consumed when no transition can be taken without doing so. 
This is quite cumbersome to implement in iUML-B since it requires constructing the conjunction of the negated guards of all the transitions that are internally triggered (including when transitions) and adding this to all externally triggered transitions.

\item [Composition of execution actions:]
When a particular SCXML transition fires it carries out a sequence of actions in a well-defined predictable order. 
For example, a hierarchy of nested source states are exited (performing their exit actions sequentially) starting from the innermost one and working outwards.
The order of execution is significant when some of these actions write to, or use the value of, a previously written variable. 
In Event-B, all actions of a transition are executed simultaneously in parallel by the elaborated event. 
It  is not possible (i.e. not well-formed) for two of these actions to write to the same variable.
If any actions use the value of a variable that is being written, the value is the value before the transition started being executed.

SCXML transitions can be designated `internal' which prevents exiting and re-entering its source state in some cases.
 In SCXML, target state can be omitted which results in a transition that does not change state (this is different from a transition that exits a state and then re-enters the same state).
Neither of these features are supported in iUML-B. A transition must have a target state and if it is the same as the source state the transition performs any exit and entry actions of its source/target state when it fires.

\item [Events:]
The meaning of event is very different between iUML-B and SCXML.
 In iUML-B transitions are sub-parts of events. 
 In order for an event to be enabled for firing, all of its sub-parts (transitions) must be simultaneously enabled. 
 This means that two different transitions with the same event can only fire at the same time and hence will never fire if they are sourced from different states of the same parent state-machine. 
In SCXML, events are triggers that enable transitions to fire. 
If two different transitions from different source states are both triggered by the same event, one may fire without the other if one source state is not active.

\item [Final States:]
The concept of a final state differs between iUML-B and SCXML. 
In SCXML a state machine (or parent state) may reside in a final state indicating that it is done and waiting for another transition to exit the parent state. 
 In iUML-B a final state is not a proper state of the parent state-machine. 
 It is merely a notation for indicating that the state-machine is becoming non-active. I.e. that the parent state is exiting. 
 Hence any transitions that target a final state are part of a transition that leaves the parent state. 
 For a `root'  state-machine, the final state means that the state-machine has been left completely and no state is active.

\item [Initial States:]
Initial states are similar in both notations. 
The transition from the initial state forms part of the actions to enter the parent state. 
However, the correspondence between incoming transitions to the parent state and initial transitions is more explicit in iUML-B. 
SCXML has another way to specify an initial state using an attribute of the state.
 In this case there is no way to add extra transition actions.
If no initial state is specified, the default is the first state in the document.
iUML-B allows different initial states for different incoming transitions. 
In SCXML this would be done by extending the transition into the substate which, in iUML-B is also an optional alternative to the multiple initial states method .

\item [Entry/Exit Actions:]
SCXML and iUML-B both include the concept of entry and exit actions which are executed whenever a transition enters, resp. exits, the containing state. 
However, their use in iUML-B is restricted by the lack of sequential composition in Event-B. 
For example, if an exit action of a state, s, assigns a to a variable, v, then no transitions from s are allowed to assign to v either directly or via entry actions of their target state.
We restrict the SCXML models that can be translated so that executing the actions in parallel is equivalent to executing them in sequence.
Effectively this means that the same variable cannot be assigned more than once in any set of actions that will be taken when a transition fires. 
The Event-B static checker will raise an error if this restriction is violated.

Another difficulty arises when a transition exits a parent state without specifying any particular nested sub-state. 
Strictly, only the exit actions of the currently active sub-state should be executed. However, this would be difficult in iUML-B due to the lack of any conditional execution.
\textcolor{red} {How do we deal with this problem? }

\item [Refinement:]
Refinement is a central concept of Event-B where detail is built up in stages facilitating validation of abstract concepts before introducing complexity. 
In iUML-B state-machines, refinement is achieved by adding nested state-machines to existing states.
There is no refinement in SCXML. The entire system is introduced in one hierarchical state-chart. We provide an extension to SCXML (Section \ref{sect:extension}) so that the target refinement level of an element can be specified.

\end{description}

%------------------------------------------------------------------------------

%\section{Reconciling Semantics}
%\label{sect:recon}
%
%\textcolor{red}{Reconciling scxml semantics for event-b code generation.  
%(Important, people will be interested)}
%
%In this section we describe how we have reconcile the disparity between SCXML and iUML-B semantics.
%%While the basic structure of states with nested Statemachines and transitions is the same as that in iUML-B, there are several semantic differences that make translation into iUML-B difficult. We note some features with possible solutions to the problem.

%\begin{description}
%
%\item [Transition triggers:] There are two kinds of transitions in SCXML. 
%`When' transitions can fire spontaneously as soon as their guard becomes true. 
%This is the same as iUML-B transitions  and we have no problem translating these.
%
%The other kind of transition is `Triggered' by an interface event. 
%This could be simulated by generating a flag to represent the trigger and adding a guard on the trigger flag to the transitions that are triggered by it. 
%The flag should then be reset by whichever transition is triggered by it in order to `consume' that trigger event.
% A special interface event that sets the flag would be generated to represent the external interface receiving a trigger.
%
%Transitions may also trigger each other. This could be modelled by a similar mechanism except that the interface event is not needed since the flag is set directly by another transition.
%
%\item [Run to completion semantics:] 
%SCXML has a run-to-completion (aka big-step/little-step) semantics.
% This means that an external trigger is only consumed when no transition can be taken without doing so. 
%This is quite cumbersome to implement in iUML-B since it requires constructing the conjunction of the negated guards of all the transitions that are internally triggered (including when transitions) and adding this to all externally triggered transitions.

%\item [Transition firing:]
%
%\item [Composition of execution actions:]
%
%\item [Events:]
%
%\item [Final States:]
%
%\item [Initial States:]
%
%\item [Entry and Exit Actions] 
%These can be added to iUML-B 
%quite easily however, the lack of sequential composition in 
%Event-B (hence iUML-B) means that the semantics of entry/
%exit actions will differ in some scenarios. That is, in 
%SCXML the source state’s exit actions are taken before the 
%transition’s actions, which are before the target state’s 
%entry actions. In iUML-B all the actions are taken in 
%parallel, as there is no concept of execution order within 
%an event. 
%We restrict the SCXML models that can be translated so that executing the actions in parallel is equivalent to executing them in sequence.
%Effectively this means that the same variable cannot be assigned more than once in any set of actions that will be taken when a transition fires. 
%The Event-B static checker will raise an error if this restriction is violated.
%
%\textcolor{red} {How do we deal with the other problem -  nested states with exit actions and a non-specific exit transition.}

%
%\item [Refinement:]
%We provide an extension to SCXML (Section \ref{sect:extension}) so that the refinement level of an element can be specified.

%\end{description}


% \begin{figure}[tbp!]
%   \VerbatimInput[samepage=false]{caseStudy/BlockState.scxml}
%   \begin{center}(a)\end{center}
%   \VerbatimInput[samepage=false]{caseStudy/Invariant.scxml}
%   \begin{center}(b)\end{center}
%   \caption{The RCA abstract classes: (a) SCXML representation (b) Event-B translation} 
%   \label{fig:scxml_event-b}
% \end{figure}

\fvset{frame=single,numbers=left,fontsize=\relsize{-2},numbersep=8pt}
\begin{figure}[tbp!]
  \VerbatimInput[samepage=false]{caseStudy/Invariant.scxml}
  \begin{center}(a)\end{center}
  \VerbatimInput[samepage=false]{caseStudy/BlockState.scxml}
  \begin{center}(b)\end{center}
  \caption{SCXML model representation: (a) invariant declaration (b) state and related transitions } 
  \label{fig:scxml}
\end{figure}

\fvset{frame=single,numbers=left,fontsize=\relsize{-2},numbersep=8pt}
\begin{figure}[tbp!]
  \VerbatimInput[samepage=false]{caseStudy/BlockState.txt}
  \caption{Event-B translation} 
  \label{fig:event-b}
\end{figure}
%------------------------------------------------------------------------------

% !TEX root = main.tex

\section{Extending SCXML}
\label{sect:extension}


To facilitate Event-B formal verification, extensions to the SCXML modelling notation are necessary so that additional modelling features required by Event-B can be integrated with the SCXML model.
The SCXML schema allows extension elements and attributes belonging to a different namespace to be added. The tooling (both XML and EMF) provides fallback mechanisms so that these extensions are supported without the need for syntactic definition. We define a new namespace,  \emph{iumlb} and add two new elements, \emph{iumlb:invariant} and \emph{iumlb:guard} (Table \ref{iumlb_elements_table}) as well as a number of new attributes which are shown in Table \ref{iumlb_attributes_table}.
 Invariants are not supported in SCXML but are needed to describe verifiable properties of a model. SCXML Transitions only have a single \emph{cond} attribute whereas we need to introduce conjuncts of a transition condition at various refinement steps and may also need to designate some invariants or guards as theorems that can be derived from the preceding conjuncts. 
New attributes are introduced to support the predicate (string) and the derived (boolean) theorem property of invariants and guards. The concept of refinement is not supported in SCXML. We introduce a new integer valued attribute, \emph{iumlb:refinement}, which may be attached to any element of either namespace in order to specify the refinement level of that element. 


% Please add the following required packages to your document preamble:
% \usepackage{multirow}
% \usepackage{graphicx}
\begin{table}[tbp]
\centering
\resizebox{\textwidth}{!}{%
\begin{tabular}{|l|l|l|}
\hline
\textbf{iumlb Element}           & \textbf{Meaning}                                             & \textbf{Legal Attributes} \\ \hline
\multirow{5}{*}{iumlb:invariant} & \multirow{5}{*}{generates an invariant in Event-B or iUML-B} & iumlb:name,               \\
                                 &                                                              & iumlb:derived,            \\
                                 &                                                              & iumlb:predicate,          \\
                                 &                                                              & iumlb:comment,            \\
                                 &                                                              & iuml:refinement           \\ \hline
\multirow{5}{*}{iumlb:guard}     & \multirow{5}{*}{generates a guard in Event-B or iUML-B}      & iumlb:name,               \\
                                 &                                                              & iumlb:derived,            \\
                                 &                                                              & iumlb:predicate,          \\
                                 &                                                              & iumlb:comment,            \\
                                 &                                                              & iuml:refinement           \\ \hline
\end{tabular}%
}
\caption{New Elements}
\label{iumlb_elements_table}
\end{table}

% Please add the following required packages to your document preamble:
% \usepackage{multirow}
% \usepackage{graphicx}
\begin{table}[tbp]
\centering
\resizebox{\textwidth}{!}{%
\begin{tabular}{|l|l|l|}
\hline
\textbf{iumlb Attribute}           & \textbf{Meaning}                                                                                                           & \textbf{Legal Parents}   \\ \hline
iumlb:label                        & string used as the name of an Event-B event elaborated by the generated i-UML-B transition                                 & scxml:transition         \\ \hline
\multirow{11}{*}{iumlb:refinement} & \multirow{11}{*}{non-negative integer representing the refinement level at which the parent element should be introduced.} & scxml:scxml,             \\
                                   &                                                                                                                            & scxml:datamodel,         \\
                                   &                                                                                                                            & scxml:data,              \\
                                   &                                                                                                                            & scxml:state,             \\
                                   &                                                                                                                            & scxml:parallel,          \\
                                   &                                                                                                                            & scxml:transition,        \\
                                   &                                                                                                                            & scxml:onEntry,           \\
                                   &                                                                                                                            & scxml:onExit,            \\
                                   &                                                                                                                            & scxml:assign             \\
                                   &                                                                                                                            & iumlb:invariant,         \\
                                   &                                                                                                                            & iumlb:guard              \\ \hline
\multirow{3}{*}{iumlb:comment}     & \multirow{3}{*}{string used as a comment on the generated iUML-B element}                                                  & iumlb:invariant,         \\
                                   &                                                                                                                            & iumlb:guard,             \\
                                   &                                                                                                                            & (could be added to more) \\ \hline
iumlb:type                         & string used as the membership set for the Event-B variable generated from the parent data element                          & scxml:data               \\ \hline
\multirow{2}{*}{iumlb:name}        & \multirow{2}{*}{string used for the name or label of a generated iUML-B element}                                           & iumlb:invariant,         \\
                                   &                                                                                                                            & iumlb:guard              \\ \hline
\multirow{2}{*}{iumlb:predicate}   & \multirow{2}{*}{string used for the predicate of a guard or invariant}                                                     & iumlb:invariant,         \\
                                   &                                                                                                                            & iumlb:guard              \\ \hline
\multirow{2}{*}{iumlb:derived}     & \multirow{2}{*}{boolean indicating that the guard is a theorem (default to false)}                                         & iumlb:invariant,         \\
                                   &                                                                                                                            & iumlb:guard              \\ \hline
\end{tabular}%
}
\caption{New Attributes}
\label{iumlb_attributes_table}
\end{table}

%------------------------------------------------------------------------------


%------------------------------------------------------------------------------

% !TEX root = main.tex

\section{Translation Tool}

An EMF meta-model for SCXML is available from the Sirius 
project. It is used by the Sirius project as a test sample 
so has not been released to users but is available in the 
github source code repository and will be copied and built 
for release with the iUML-B-SCXML tools. It supports SCXML 
functionality including data modelling and action/
expression language. 
Running this code in a debug environment enables us to 
produce the following SCXML model using the EMF sample 
editor. The serialised XMI is shown to the left. 

The serialisation shown above is the default produced for a 
newly created model. It contains the prefix tag ‘scxml:’ on 
each element, which is not standard scxml syntax. However, 
the serialisation is flexible and is able to load/update 
models without this prefix being introduced.

Hierarchical nested state charts are translated to similar 
corresponding state-machine structures in iUML-B according 
to the rules described in 5.2.1. There are two alternative 
styles of Event-B representation for iUML-B state-machines. 
Currently the state-variables style is adopted because it 
is simpler to translate from the Scxml model. However, the 
alternative state-enumeration style has benefits in user-
readability and may be supported in future. This would 
require conventions regarding the name of the state-machine 
to be adopted and used by the modeller in order to 
construct guards that refer to the current value of the 
state-machine.

\subsection{Refinement Levels}
For scxml:State the refinement level refers to any state 
machines generated from the children nested in the state (i.
e. generated iUML-B states must be all based on the 
refinement level of the containing statemachine).
For iumlb:invariants the generated invariant is only 
generated at the specified refinement level, not in 
subsequent refinements.

\subsection{Constructing event elaborated by transitions}
The Event-B events that are elaborated by an iUML-B 
transition are constructed as follows:

The event names are obtained (cumulatively) by the 
following methods:
a)  the transition has iumlb:label attributes,
b) the transition's source is an initial state (see below), 
or,
c) if none of the above provide any labels, a default 
'source\_target' format is used.

Note that trigger events are deliberately not used for 
transition events because we want to keep them as a 
separate concept from transition firing in line with scxml 
semantics.

If the transition is in an initial state at the outer state 
chart level the name is INITIALISATION. 
If the transition is in an initial state of a nested state 
chart the names of all the events that are associated with 
incoming transitions to the parent state are used.

\subsection{Data elements}
Data elements (which are collated in Datamodel elements) 
are used to model ancillary variables as part of the model 
in the way that SCXML intends them to be used.

Data elements are interpreted as a variable. The type of 
the variable is given in an iumlb:type attribute. 
The id attribute is interpreted as the name of the 
variable. The value is used as the right hand side of an 
assignment action to initialise the variable.  (Some syntax 
conversion is performed to convert the predicate from SCXML 
format into Event-B mathematical language). The variable is 
introduced at the same refinement level as the parent 
element that contains it.


%------------------------------------------------------------------------------

%%% Local Variables: 
%%% mode: latex
%%% TeX-master: "HDMachine"
%%% End: 


%------------------------------------------------------------------------------


\section{Case Study}
\label{sect:caseS}

\textcolor{red}{Find a system that we can model and some how describe the benefits 
(e.g. model simplification) of using a specific syntax over the other.}
	\begin{itemize}
		\item What model behavior can you capture with each semantic?
		\item What properties of the model are easier to simulate?
		\item Where do we introduce unnecessary complexity?
	\end{itemize}	

\begin{figure}[]
  \begin{centering}
  \includegraphics[width=0.75\textwidth]{caseStudy/TurnstileSimpleModel}
  \caption{SCXML Statemachine diagram}
  \label{fig:StatemachineSCXML}
  \end{centering}
\end{figure} 

\begin{figure}[]
  \begin{centering}
  \includegraphics[width=0.75\textwidth]{caseStudy/TurnstileSimpleModel_iumlb}
  \caption{Part of State-machine diagram in iUML-B}
  \label{fig:StatemachineiUML-B}
  \end{centering}
\end{figure} 

%------------------------------------------------------------------------------
\section{Conclusion}
\label{sect:concl}

Just to have a reference ~\cite{texniccenter}

%------------------------------------------------------------------------------
\section{Future Work}
\label{sect:future-work}


% \begin{figure}[tb]
%   \begin{centering}
%     \includegraphics[width=0.5\textwidth]{throneEC.jpg}
%     \includegraphics[width=0.3\textwidth]{throneEC.jpg}
%     \includegraphics[width=0.15\textwidth]{throneEC.jpg}
%   \end{centering}
%   \caption{Why one should use EasyChair}
%   \label{fig:easythrone}
% \end{figure}

%------------------------------------------------------------------------------
\section{Acknowledgments}
\label{sect:acks}


\label{sect:bib}
\bibliographystyle{plain}
%\bibliographystyle{alpha}
%\bibliographystyle{unsrt}
%\bibliographystyle{abbrv}
\bibliography{easychair}

%------------------------------------------------------------------------------

%------------------------------------------------------------------------------
% Index
%\printindex

%------------------------------------------------------------------------------
\end{document}

