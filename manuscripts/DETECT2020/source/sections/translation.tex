% !TEX root = ../main.tex

\section{\SCXML Translation to \EventB}
\label{sec:translation}

The translation of a specific \SCXML model to \UMLB and subsequently to \EventB, comprises the following stages: 
\begin{itemize}
	\item 
Firstly, a basis machine and context are created to embody the semantics of the \SCXML language (Section~\ref{sec:run-completion}).
The basis provides variables and events to model the queue of triggers as well as abstract versions of events to model transitions firing.
The basis is independent of the particular \SCXML model which is added in subsequent refinements.
	\item 
Secondly, all possible combinations of transitions that can fire together are calculated and corresponding events are generated, at appropriate refinement levels, that refine the abstract basis events.  
If these transitions raise internal triggers, a guard, (e.g. |{i1, i2, ...} <: raisedTrigger|, where |i1, i2, ...| have been added to the internal triggers set), is introduced to define the raised triggers parameter. 
The subset allows more triggers to be raised in later refinements.
For triggered transitions, the trigger is specified by a guard that defines the value of the trigger parameter.\todo{COLIN: Explain how to elaborate all possible combinations of transitions} 
	\item 
Thirdly, the \SCXML state-chart is translated into a corresponding \UMLB state-machine whose transitions elaborate (i.e. add state change details to) the transition combination events that the transition may be involved in.
A transition may fire in parallel with transitions of parallel nested state-machines that have the same (possibly null) trigger.
	\item
Finally the \UMLB state-machine is translated into \EVENTB by programmatically invoking the \UMLB translator.
\end{itemize}
% Further details of the translation are given in~\cite{MoSn16,MoSnHo18,MoSnHo-ABZ2020}.

% New features of the translation added since~\cite{MoSnHo18} are as follows:
% \begin{description}
% \item[Trigger queues in basis:]
%   \begin{sloppypar}
%     The encoding of trigger queues in the abstract basis machine has been improved so that triggers are properly dequeued before potential use,
%     which allows triggers to be discarded if the controller cannot respond to them. 
%     This more accurately reflects the \SCXML semantics and was necessary in order to model the new drone case study properly.
%   \end{sloppypar}

% \item[Finalisation:] Transitions can be flagged as finalised which means their guards can not be strengthened in subsequent refinements. This allows them to `enforced' when they are enabled (i.e. completion cannot occur until they have fired) which is needed for verification. 

% \item[Restricted raising of internal triggers:] Once a trigger is introduced it must immediately be raised at that refinement level by any transitions that wish to do so. It cannot be raised in later refinements except by newly introduced transitions. This restriction was necessary to make simulation more useful by removing non-deterministic raising of triggers in anticipation of refinements.

% \item[Context instantiation:] The axioms of the basis context, that allow future triggers to be added, has been improved so that \PROB\footnote{ProB is an animator, constraint solver and model checker for the B-Method. https://www3.hhu.de/stups/prob} can automatically create an instantiation. 

% \end{description}

A tool to automatically translate \SCXML models into \UMLB has been produced. 
The tool is based on the \EMF and uses an \SCXML meta-model provided by Sirius~\cite{siriuswebsite} which has good support for extensibility. 
The \UMLB state-machine is subsequently translated into \EVENTB using the standard \UMLB translation~\cite{snook14:iumlbStatem} which provides variables to model the current state and guards and actions to model the state changes that transitions perform. Further details of the translation are given in~\cite{MoSn16,MoSnHo18,MoSnHo-ABZ2020}.

%%% Local Variables:
%%% mode: latex
%%% TeX-master: "../main"
%%% End: