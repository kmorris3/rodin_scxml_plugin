\documentclass{response}

\papertitle{Formal verification and validation of run-to-completion style state-charts using Event-B}
\paperauthor{K. Morris, C. Snook, T.S.Hoang, G. Hulette, R. Armstrong, and M. Butler}
\usepackage{chgtrk}
\newCTcontributor{Karla}
\newCTcontributor{Colin}
\newCTcontributor{Rob}
\newCTcontributor{Geoff}
\newCTcontributor{Son}
\newCTcontributor{Michael}

\begin{document}


\begin{comment}{Reviewer \#1}
    Presentation
    This paper presents an embedding of statecharts in Event-B. As such, it can
    be considered as a shallow embedding of statecharts in Event-B. A strong
    point of the paper is to adress through refinements the derivation of a 
    concrete application. The starting point being the basic semantics of the 
    execution model of statecharts and the last refinement being the considered
    application at a given level of detail. Moreover, the paper goes beyond the 
    usual safety properties usually addressed by such approaches : liveness 
    properties are here considered.
    From my point of view, the presented work is valuable. Actually, it
    can be considered as emerging from the synthesis of three conference papers
    [12,13,14].
\end{comment}

\begin{response}
  ...
  \begin{quote}
    `` ...
   ''
  \end{quote}
\end{response}


\begin{comment}{Reviewer \#1}
  Discussion
    I would appreciate if the authors could elaborate their work along the 
    following points:
    - Semantics. Statecharts have a long history. By now, they can be
    considered as belonging to the family of synchronous languages. It
    would be interesting if the authors could give some comments on the
    choice of statecharts and especially on the relevance of the considered
    semantics of run-to-completion. Otherwise stated could the authors
    highlight the benefits of such a choice in general and of stachecharts
    especially.
\end{comment}

\begin{response}
  ...
  \begin{quote}
    `` ...
   ''
  \end{quote}
\end{response}


\begin{comment}{Reviewer \#1}
- Proof of temporal properties. Different temporal properties are con-
sidered. Since the proof of such properties are outside (currently) the
Event-B method they have considered other tools and methods.
It would have been interesting to outline the frontier. As a matter of
fact, it is not clear for me if the Event-B properties can be tackled
in temporal logic, e.g., how sequences are embedded? Could you be
precise, comment your embedding?
\end{comment}

\begin{response}
  ...
  \begin{quote}
    `` ...
   ''
  \end{quote}
\end{response}

\begin{comment}{Reviewer \#1}
- Proof of fairness properties. In order to deal with fairness properties
you advocate a strong fairness assumption. From my point a view, this
is a strong operating assumption. May be you could comment on that?
Last, but may be I am wrong, I have the intuition that weak fairness
for the events handling dequeuing external triggers would be enough?
Please could you comment?
\end{comment}

\begin{response}
  ...
  \begin{quote}
    `` ...
   ''
  \end{quote}
\end{response}

\begin{comment}{Reviewer \#1}
• p.2 The three rules are not at the same level. The first ones are expressed 
explicitely in terms of Event-B refinement features, while the third one 
addresses statecharts.
\end{comment}

\begin{response}
  ...
  \begin{quote}
    `` ...
   ''
  \end{quote}
\end{response}


\begin{comment}{Reviewer \#1}
• p. 10 typo? listing 3 l. 10 \emptyset
\end{comment}

\begin{response}
  ...
  \begin{quote}
    `` ...
   ''
  \end{quote}
\end{response}

\begin{comment}{Reviewer \#1}
• p.13 Could you illustrate the sentence all possible combinations of each
set of transitions that can fire together are calculated and corresponding
events are generated, at appropriate refinement levels.
\end{comment}

\begin{response}
  ...
  \begin{quote}
    `` ...
   ''
  \end{quote}
\end{response}

\begin{comment}{Reviewer \#1}
• p.15 Fig. 5 is too small. One cannot read its text.
\end{comment}

\begin{response}
  ...
  \begin{quote}
    `` ...
   ''
  \end{quote}
\end{response}

\begin{comment}{Reviewer \#1}
• p.18 It would have been interesting to state the discussion of the first
paragraph of section 7: Verification of Safety Properties within the
context of the proof obligation generator you have at hand.
\end{comment}

\begin{response}
  ...
  \begin{quote}
    `` ...
   ''
  \end{quote}
\end{response}

\begin{comment}{Reviewer \#1}
• p.19 Could you precise your notion of run.
\end{comment}

\begin{response}
  ...
  \begin{quote}
    `` ...
   ''
  \end{quote}
\end{response}


\begin{comment}{Reviewer \#1}
• p. 20 typo. We are now present
\end{comment}

\begin{response}
  ...
  \begin{quote}
    `` ...
   ''
  \end{quote}
\end{response}

\begin{comment}{Reviewer \#1}
• p. 21 Could you give a formal definition or at least a reference of your 
strong fairness.
\end{comment}

\begin{response}
  ...
  \begin{quote}
    `` ...
   ''
  \end{quote}
\end{response}

\begin{comment}{Reviewer \#1}
• p.22 Could you comment on your definition of anticipated events. Why the set 
of convergent events is necessary to recall just before?
\end{comment}

\begin{response}
  ...
  \begin{quote}
    `` ...
   ''
  \end{quote}
\end{response}

\begin{comment}{Reviewer \#1}
• p.22 Proof of Convergence an Anticipation I wonder if this paragraph should 
not be before because you use such arguments before just after stating Theorem 
2.
\end{comment}

\begin{response}
  ...
  \begin{quote}
    `` ...
   ''
  \end{quote}
\end{response}

\begin{comment}{Reviewer \#1}
• p.23 typo. it will be dequeued.
\end{comment}

\begin{response}
  The text has been revised as follow:
  \begin{quote}
    `` If an external trigger is raised, then eventually, it will be dequeued.''
  \end{quote}
\end{response}

\begin{comment}{Reviewer \#1}
• p. 23 could you explain the square bracket notation, e.g. [externalTrigger.t]
\end{comment}

\begin{response}
  ...
  \begin{quote}
    `` ...
   ''
  \end{quote}
\end{response}


\begin{comment}{Reviewer \#1}
• typo. they do no hold a priori.
\end{comment}

\begin{response}
  The text has been revised as follow:
  \begin{quote}
    ``These proof will need to be done for each individual SCXML state-chart as they do not hold a priori.''
  \end{quote}
\end{response}


\begin{comment}{Reviewer \#1}
• typo. relying on lexicographic order . . .
\end{comment}

\begin{response}
  The text has been revised as follow:
  \begin{quote}
    ``We present a generic approach
    to reason about the proof of convergence and anticipation relying on
    lexicographic order as follow.''
  \end{quote}
\end{response}



\begin{comment}{Reviewer \#1}
• The paragraph Proof of Convergence and Anticipation needs to
be written again. There are many typos: this event removes, discards,
decreases, accroding, . . .
Moreover the sentence
The external events are anticipated accroding to the above variants triv-
ially since they only modify the external queue eQ. Note that we do not
attempt to prove the convergence of any future events here. Instead, we
assume that these future events will be prove to be convergence later.
seems to me problematic. In your definition of anticipated you did not
say that these events should be proven convergent later?
\end{comment}

\begin{response}
  ...
  \begin{quote}
    `` ...
   ''
  \end{quote}
\end{response}



\begin{comment}{Reviewer \#1}
• p.25 could you state explicitly your strong fairness property and the
interplay with the temporal properties you are concerned with.
\end{comment}

\begin{response}
  ...
  \begin{quote}
    `` ...
   ''
  \end{quote}
\end{response}


\begin{comment}{Reviewer \#1}
• p.25 A reference to the seminal Unless of Unity could be in order.
\end{comment}

\begin{response}
\KarlaInlineCommentTo{Son}{Please change in the manuscript}
  \begin{quote}
    `` ...
   ''
  \end{quote}
\end{response}


\begin{comment}{Reviewer \#1}
• p. 26 (Theorem 5) I think that the indexes in eQ should be first stated
as legal in both quantifications.
\end{comment}

\begin{response}
\KarlaInlineCommentTo{Son}{Please change in the manuscript}
  \begin{quote}
    `` ...
   ''
  \end{quote}
\end{response}


\begin{comment}{Reviewer \#1}
• p. 27 All the temporal proofs have been done in an adhoc way without
any tool support. It would be interesting to have a feedback about
this? To be provocative, if you are interested in temporal proofs why
did you choose this tool? Have you considered TLA which does support
temporal proofs (as well as refinements in a certain way)?
\end{comment}

\begin{response}
\KarlaInlineCommentTo{Son}{Add some comments of the complexity of the models and and translating them to TLA}
\KarlaInlineCommentTo{Son}{Add to manuscript in the related work section TLA}
  \begin{quote}
    `` ...
   ''
  \end{quote}
\end{response}


\begin{comment}{Reviewer \#1}
• It would be interesting to analyze if the proofs are specific to the ex-
ample or to the underlying semantics?
PS Could you put another zip version on the repository: I have had some
strange problems (missing characters) with some machine files? For such
files, I was not able to play again the proofs.
\end{comment}

\begin{response}
\KarlaInlineCommentTo{Son}{Rephrace in page 23 second bullet point the fact that the variant is model specific. Add a couple of sentences on the semantic nature of the other variants}
  \begin{quote}
    `` ...
   ''
  \end{quote}
\end{response}
 



\begin{comment}{Reviewer \#2}
The paper introduces a technique for the refinement of `run to completion' 
statechart modelling notation (using SCXML language) while preserving safety 
properties. The statechart specification is translated to event-B formalism, 
allowing for formal verification using a theorem prover. The proposed approach 
is demonstrated using a statechart specification of a drone.

Positive points:
+ Interesting topic
+ Technique well motivated
+ The paper is well written and easy to read.

Negative points:
- One single case study is not enough to validate the proposed approach. The 
statechart specification of the drone is rather small. More elaborated models 
are required to validate the proposed approach.
\end{comment}

\begin{response}
\KarlaInlineComment{Add references to previous case studies. Make a case that the case studies have grown in complexity and the thje drone in particular makes use of all the features for model construction and refinement. SecBot and Turnstile}
  \begin{quote}
    `` ...
   ''
  \end{quote}
\end{response}



\begin{comment}{Reviewer \#2}
General comments:
  - The three refinement rules listed in the introduction have not been 
  described explicitly in the rest of the paper. Please describe them (using 
  minimal examples) in section 3.
\end{comment}

\begin{response}
\KarlaInlineComment{Add section 3.2 for refinement rules}
  \begin{quote}
    `` ...
   ''
  \end{quote}
\end{response}


\begin{comment}{Reviewer \#2}
  - In the introduction, the paragraph before last "Page 3: lines 7 to 17" that
  compares the proposed approach to the work presented in [4] may be pushed to
  Section 3 or 4 since such comparison is meaningless before presenting the
  details of the approach and the example.
\end{comment}

\begin{response}
  ...
  \begin{quote}
    `` ...
   ''
  \end{quote}
\end{response}


\begin{comment}{Reviewer \#2}
 - The paper lacks a related work section. Please add one.
\end{comment}

\begin{response}
A related work section has been added to the manuscript. Please see section 
  \begin{quote}
    \KarlaInlineComment{I cannot remember what we agree on doing for this. It seems that there are sections of the intro that need to be move to related work}

    `` ...
   ''
  \end{quote}
\end{response}



\begin{comment}{Reviewer \#2}
Minor points:
  - After an introductory word or phrase, use a comma (this is a recurrent in
  the paper). For example, e.g. --> e.g., i.e. --> i.e., "To verify liveness 
  we outline" --> "To verify liveness,  we outline", etc.
\end{comment}

\begin{response}
  The requested changes have been completed.
\end{response}


\begin{comment}{Reviewer \#2}
- Abstract: "We introduce" --> "In this paper, we introduce".
\end{comment}

\begin{response}
  The text has been revised as follow:
  \begin{quote}
    `` In this paper, we introduce a notion of refinement into a ‘run to completion’ state-chart 
    modelling notation, and leverage Event-B’s tool support for theorem proving. ''
  \end{quote}
\end{response}


\begin{comment}{Reviewer \#2}
- Add "Even-B" to the list of keywords.
\end{comment}

\begin{response}
  The requested change has been completed.
\end{response}


\begin{comment}{Reviewer \#2}
- Page 2: line 5: "Particularly attractive is providing" --> "Particularly 
attractive in providing"
\end{comment}

\begin{response}
  The text has been revised as follow:
  \begin{quote}
    ``It is particularly attractive, to provide accessibility to abstraction/refinement via Rodin/Event-B 
    which has an intuitive metaphor in the Statechart semantics [12,14,13]. ''
  \end{quote}
\end{response}


\begin{comment}{Reviewer \#2}
- Page 2: line 10: "safety preservation" --> "safety properties preservation"
\end{comment}

\begin{response}
  The requested change has been completed.
\end{response}


\begin{comment}{Reviewer \#2}
- Page 2: line 20: "Preservation of safety" --> "Preservation of safety 
properties"
\end{comment}

\begin{response}
  The requested change has been completed.
\end{response}


\begin{comment}{Reviewer \#2}
- Page 2: line 45: "in the sense of [9]" --> "in the sense adopted by Lamport [9]"
\end{comment}

\begin{response}
  The requested change has been completed.
\end{response}


\end{document}
