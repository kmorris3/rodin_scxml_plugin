% !TEX root = ../main.tex
\section{Verification of Control Responses}
\label{sec:verificationResponses}

A model that has been proven to satisfy some safety (e.g., invariant)
properties, may still not behave in a useful way.  Therefore, as well
as verifying invariant properties, we would like to verify the
system's liveness (e.g., responsive) properties. That is, we want to
ensure that the controller responds to external triggers and makes
appropriate modifications to the system variables.  These kind of
liveness properties are difficult to prove via invariant preservation
since they are temporal properties.  In this section, we present our
approach to verify the responsive properties of the system.

We first start with some generic properties of our generated \EventB
model and the fairness assumption about the executions of the events.
We then discuss the proof for termination of responses for external
triggers in Section~\ref{sec:contr-rema-resp} and correctness of the
responses for external triggers in Section~\ref{sec:corr-resp-extern}.

\paragraph{Event Categories.} In our \EventB model, the events can
be separated into the following categories.
\begin{itemize}
\item \emph{External events}: These events raise external triggers.
  
\item \emph{System events}: Events other than external events are
  called system events. They are the events by which the system
  responds to the external triggers (by creating 
  different runs or simulation paths).
  These events can be seen in Figure~\ref{fig:basis} and are further
  categorised as follow.
  \begin{itemize}
  \item \emph{Future system events}: These events might raise internal
    triggers, i.e., |futureTriggeredTransitionSet|, and
    |futureUntriggeredTransitionSet|.  The purpose of these events is
    to enable future introduction of more system details via
    refinement.
    
  \item \emph{The dequeue external trigger event} (i.e.,
    |dequeueExternalTriggered|): These events dequeue the
    external trigger queue and will start a run.

  \item \emph{Internal system events}: These events belong to the
    internal behavior of the system to accommodate the runs for external
    triggers.  These events can be seen in different groups as in
    Figure~\ref{fig:basis}.
    \begin{itemize}
    \item \emph{The dequeue internal trigger event} (i.e.,
      |dequeueInternalTriggered|): These events dequeue the
      internal triggers queue and will start a run.
    
    \item \emph{Triggered events}: The events corresponding to the
     event fired by (external or internal) triggers.

    \item \emph{The discard trigger event} (i.e.,
      |noTriggeredTransitionsEnabled|):  This event will move the system
      from the |FIRING TRIGGERED| state to the |FIRING UNTRIGGERED| state 
      in the case where no triggered transition events are enabled.
    
    \item \emph{Untriggered events}:  These are all the untriggered 
    events in the systems.

    \item \emph{The completion event} (i.e.,
      |noUntriggeredTransitionsEnabled|): This event will move the
      system from |FIRING UNTRIGGERED| state to the |READY TO DE-QUEUE|
      state in the case where no untriggered transition events are
      enabled.
    \end{itemize}
  \end{itemize}
\end{itemize}
Events from different categories have different roles in our reasoning
about responsive properties.

We now present a theorem and a corollary related to (relative)
deadlock-freeness properties for a different set of events.
\begin{theorem}[Internal System Events are Relative Deadlock-Free]
  \label{thm:Internal-DLF}
  Under the condition that %
  |iQ /= {} !or uc = FALSE !or dt /= {}|, %
  the internal system events are deadlock-free, i.e., there must be
  one internal system event enabled.
\end{theorem}
\begin{proof}
  This is based on the generation of our \EventB model (according to
  the basis structure as seen in Figure~\ref{fig:basis}).  In
  particular, we consider the different cases corresponding to the
  different ``states'', i.e., |READY TO DE-QUEUE|, |FIRING TRIGGERED|,
  and |FIRING UNTRIGGERED|.
  \begin{itemize}
  \item When the system is in the |READY TO DE-QUEUE| state, we know
    that %
    |uc = TRUE & dt = {}|.  %
    According to our assumption, we then have |iQ /= {}|, hence
    |dequeueInternalTriggered| event is enabled.
    
  \item When the system is in the |FIRING TRIGGERED| state,
    either one of the triggered events is enabled or the
    |noTriggeredTransitionsEnabled| event is enabled.
    
  \item Similarly, when the system is in the |FIRING UNTRIGGERED|
    state, either one of the untriggered events is enabled or the
    |noUntriggeredTransitionsEnabled| event is enabled.  
  \end{itemize}
  \qed
\end{proof}
    
\begin{corollary}[System Events are Relative Deadlock-Free]
  Under the condition that %
  |eQ /= {} !or iQ /= {} !or uc = FALSE !or dt /= {}|, %
  the system events are deadlock-free, i.e., there must be
  one system event enabled.
\end{corollary}
\begin{proof}
  This is based on the generation of our \EventB model (according to
  the basis structure as seen in Figure~\ref{fig:basis}) and
  Theorem~\ref{thm:Internal-DLF}.
  \begin{itemize}
  \item In the case where |iQ /= {} !or uc = FALSE !or dt /= {}|,
    according to Theorem~\ref{thm:Internal-DLF}, one of the internal
    events is enabled.
    
  \item Otherwise, i.e., |iQ = {} & uc = TRUE & dt = {}|, according to
    our assumption, |eQ /= {}|. In this case, the
    |dequeueExternalTriggered| event is enabled.
  \end{itemize}
  \qed
\end{proof}

In order to reason about any liveness properties for an event system,
we have to make assumptions about how often events will be fired.
Here, we assume that all the system events are strongly fair.
\begin{assumption}[Fair System Events]
  \label{asm:SF}
  % \SonChange{%
  %   We assume that all system events are strongly fair, i.e., if a
  %   system event is enabled infinitely often, eventually, it will be
  %   eventually fired.%
  % }{%
  \hl{%
    We assume that all internal system event \mbox{\EventBInline{e}}
    are strongly fair, i.e., \mbox{\EventBInline{SF(e)}}; and the
    dequeue external trigger event is weakly fair,
    i.e., \mbox{\EventBInline{WF(dequeueExternalTriggered)}}.%
  }
\end{assumption}
This assumption will ensure that the system will response no matter
how often the external triggers are raised by the external events.

\subsection{Termination of Responses for External Triggers}
\label{sec:contr-rema-resp}

We first define the notion of event convergence and event
anticipation.
\begin{definition}[Event Convergence]
  \label{def:conv}
  A set of events is said to be convergent if they all \emph{decrease}
  a variant according to some well-founded order.
\end{definition}

\begin{definition}[Event Anticipation]
  \label{def:anticipated}
  Given a set of convergent events with respect to a variant,
  another set of events is anticipated with respect to the
  same variant if they \emph{do not increase} the variant.
\end{definition}
\hl{%
  Note that the anticipated events augment the set of convergent
  events and respect the variant used to prove the
  convergence property.% 
}

We start first by stating the main theorem about termination of
responses for external triggers: it is always the case that the system
will comeback to the |READY TO DE-QUEUE| state and |iQ = {}|, i.e.,
the system is ready to dequeue an external trigger (if any).  This is
stated as the following theorem.
\begin{theorem}[Termination of Internal System Events]
  \label{thm:finite-internal-events}
  Given that the internal events are convergent and the external
  events are anticipated, the system's internal queue is always
  eventually empty and the system transitions to the \emph{Ready to
    De-queue} state, i.e.,
  \begin{center}
    |GF(iQ = {} & uc = TRUE & dt = {})|~.
  \end{center}
\end{theorem}
\begin{proof}
  Assuming that the properties is not satisfied, i.e., eventually,
  it is always the case that %
  |iQ /= {} !or  uc = FALSE !or dt /=  {}|.  %
  This can be formalised as follows.
  \begin{center}
    |FG(iQ /= {} !or uc = FALSE !or dt /= {})|~.    
  \end{center}
  Observe that in the states satisfying this condition, the
  |dequeueExternalTriggered| event is disabled.  Furthermore,
  according to Theorem~\ref{thm:Internal-DLF}, the internal events
  will always be deadlock-free, and as a result, at least one of the
  internal event is enabled infinitely often\hl{, hence under the
    Assumption~\mbox{\ref{asm:SF}}, this event occure infinitely
    often}. According to Definition~\ref{def:conv}, the variant will
  be decreased infinitely which violate the condition that the variant
  is defined on an well-founded order.  Here, the external events are
  anticipated to ensure that the variant does not increase, hence does
  not interfere with the convergence of the internal events.%
  \qed
\end{proof}
\hl{%
  Note that Theorem~\mbox{\ref{thm:finite-internal-events}} relies on
  convergence of internal events and anticipation of external events,
  which we will prove later.%
}

\begin{theorem}[Responsiveness to External Triggers]
  \label{thm:resp-ext-trg}
  If an external trigger is raised, then eventually, it will be dequeued.
  \begin{center}
    |G([externalTrigger.t] => F([dequeueExternalTriggered.t]))|~,
  \end{center}
  where we use |externalTrigger.t|
  (resp. |[dequeueExternalTriggered.t]|) to denote the occurrence of
  |externalTrigger| (resp. |dequeueExternalTriggered|) with parameter |t|.
\end{theorem}
\begin{proof}
  Assuming that |[externalTrigger.t]| hence |t !: content(eQ)|, i.e.,
  |eQ /= {}|.  According to Theorem~\ref{thm:finite-internal-events},
  we have that |dequeueExternalTriggered| is enabled infinitely often.
  Since |dequeueExternalTriggered| is weakly-fair
  (Assumption~\ref{asm:SF}), it is taken infinitely. We have two
  cases.
  \begin{itemize}
  \item If |t = head(eQ)|, it means we have
    |[dequeueExternalTriggered.t]|
    
  \item If |t /= head(eQ)|, an occurrence of |dequeueExternalTriggered|
    will move |t| closer to be the head of the external queue |eQ| and
    eventually it will become the head of the queue and subsequently
    be processed eventually.
  \end{itemize}
  \qed
\end{proof}

\subsubsection{Proof of Convergence and Anticipation}
\label{sec:proof-convergence}
The responsiveness to external triggers presented in
Theorem~\ref{thm:resp-ext-trg} relies on
Theorem~\ref{thm:finite-internal-events}, which in turn relies on the
proof of convergence for internal system events and anticipation for
external events.  These proof will need to be done for each individual
SCXML state-chart as they do not hold a priori.  We present a
systematic approach to reason about the proof of convergence and
anticipation relying on lexicographic order as follow.

A variant in \EventB can be a natural number (bounded bellow by |0|)
or a finite set (bounded below by the empty set |{}|).  Moreover, for
a set of events, the variants $V_1, V_2, ...$ are combined into a
lexicographic variant, i.e., $(V_1, V_2, ...)$ with $V_1$ has a higher
priority than $V_2$, etc. An event is said to decrease this
lexicographic variant if it either decreases $V_1$ or if it keeps
$V_1$ the same and decreases $V_2$, so on and so forth.  Lexicographic
variants are supported in the latest Rodin (version 3.5).

Our generic approach to construct a lexicographic variant is according
to the following order and the rule for each variant.
\begin{enumerate}
\item $V_{externalTrigger}$ = |dt /\ ExternalTrigger|.  This variant
  is used to prove the convergence for any externally triggered events
  (i.e., triggered event by some external trigger). These event remove
  the external trigger from |dt| hence ``decrease'' |dt| to the empty
  set.

\item \label{variant:sm} Variants based on the state machine to prove
  the convergence of internally triggered events and untriggered
  events.  This depends on the SCXML diagram and we will illustrate
  this on the example later.
  
\item $V_{dequeueInternalTriggered}$ = |length(iQ)|.  This variant is
  used to prove the convergence of the |dequeueInternalTriggered| event.
  Since this event remove the head of the |iQ|, it decreases the
  length of the |iQ| trivially.  While other events might increase
  |iQ|, by raising new internal triggers. However, these events should
  have been proved to converge using higher-priority variants.
  
\item $V_{noTriggeredTransitionsEnabled}$ = |dt|.  This variant is used to
  prove the convergence of the |noTriggeredTransitionsEnabled| event.
  The event discard the trigger in |dt| hence ``decrease'' |dt| to the
  empty set.
  
\item $V_{noUntriggeredTranstitionsEnabled}$ = |{uc, TRUE}|.  This variant is
  used to prove the convergence of the
  |noUntriggeredTransitionsEnabled| event.  The event changes |uc|
  flag from |FALSE| to |TRUE|, hence ``decrease'' the variant from
  |{FALSE, TRUE}| to |{TRUE}|.
\end{enumerate}
\hl{%
  Note that except for the variant related to the internally triggered
  events and untriggered events, i.e., \mbox{(\ref{variant:sm})}, all
  other variants, i.e., $V_{externalTrigger}$,
  $V_{dequeueInternalTriggered}$, $V_{noTriggeredTransitionsEnabled}$,
  and $V_{noUntriggeredTranstitionsEnabled}$ are generic according to
  the underlying run-to-completion semantics.%
}

The external events are anticipated according to the above variants
trivially since they only modify the external queue |eQ|.  Note that
we do not attempt to prove the convergence of any future events
here. Instead, we assume that these future events will be proved to be
convergent later.

% Order of the variants
We now discuss the specific variants for the Drone example based on
the actual state-chart as showed in Figure~\ref{fig:drone4}.  The
variants are for the internally triggered events and untriggered
events. The lexicographic order of the variant use to prove the
convergence of the events depending (1) on the \emph{nested structure}
of the state-chart and (2) on the \emph{order of the transitions} with
the same state-chart.  For example, the variant for proving the
convergence for the transition from |SHUTDOWN| to |OFF| will have a
higher priority than the one for the transition from |TAKEOFF| to
|FLY|, and this variant subsequently has higher priority than the one
for the transition from |CLIMB| to |HOVER|.  Furthermore, the variant
for proving the convergence for the transition from |TAKEOFF| to |FLY|
has higher priority than the one for the transition from |FLY| to
|DESCEND|.

% Shape of the variants
The translation of SCXML state-chart into UML-B/\EventB represents each
state by a Boolean variable, |TRUE| if the system in that state and
|FALSE| otherwise. As a result, the variant for proving the convergence
of an event going out of a state |S| can be |{S, FALSE}|: the
transition ``decreases'' the value of the variant from |{TRUE, FALSE}|
to |FALSE|.

Based on the above analysis, the variants that we used for proving the
convergence of the internally triggered events and untriggered events
in the Drone example are.
\begin{itemize}
\item |{SHUTDOWN, FALSE}|

\item |{TAKEOFF, FALSE}|

\item |{FLY, FALSE}|

\item |{BATTERYOK, FALSE}|

\item |{CLIMB, FALSE}|

\item |{CLIMB2, FALSE}|
\end{itemize}

This variant proof are available in the Rodin archive at
\url{https://tinyurl.com/ISSE-Drone}.


\subsection{Correct Responses to External Triggers}
\label{sec:corr-resp-extern}
In the previous section, we illustrate the reasoning about the
responsiveness of the system to external triggers.  However, we also
need to prove that the response to the external triggers is correct.
In our conference paper~\cite{detect2020}, we illustrate the use of
ProB model checker to reason about such a property.  Here, we show how
we can prove such a property.

%\SonInlineComment{We should probably move this to the discussion}
% While \EventB refinements have also been shown to preserve some liveness properties under certain conditions~\cite{hoang2016ltl}, there are not yet efficient supporting tools for the technique.%
% \SonComment{We probably need to move this to related work}
% Instead, we can express the property in \LTL  and use the \PROB\footnote{ProB is an animator, constraint solver and model checker for the B-Method. https://www3.hhu.de/stups/prob} model checker to verify it.

Once again, we assume that the system events are strongly fair as in
Assumption~\ref{asm:SF}.  In general, our correct-response properties
will have the following form:
\begin{center}
  |G([external_trigger_event] => F{predicate})|~,
\end{center}
where the predicate concerns variables |v| that the system maintains,
and may refer to old values |old(v)| that existed when the external
trigger occurred.  The translator generates a separate `branch'
refinement for each \LTL property to be verified.  In this special
refinement, history variables are added to record the value at the
state when the external trigger occurs, of any variables that are
referenced as `old' values.

We illustrate the method with an example of a temporal property that
we expect to hold in the drone \SCXML system.  The liveness property
that we wish to verify is that, after an external trigger event
|decreaseCharge|, the battery charge value should decrease in value, i.e.,
\begin{center}
  |G ([ExternalTriggerEvent_decreaseCharge] => F {charge < old(charge)})|~.
\end{center}
As discussed in~\cite{detect2020}, this above property is too strong
and does not hold for the SCXML drone model.  We have to weaken the
property to state that the expected behaviour is only achieved if the
external trigger is raised at the right time, specified as %
|{BATTERYOK=TRUE & charge>20}|, %
and there are no external conflict triggers, here |off|, in processing
or that has been raised, i.e., %
|off /: dt \/ content{eQ}|.  %
The property can be formalised as follows

\begin{center}
  |G([ExternalTriggerEvent_decreaseCharge] & |\\
  |{BATTERYOK=TRUE & charge>20 & off/:dt & off /: content(eQ)}|\\
  |=> F {charge < old(charge)})|~.
\end{center}
In order to prove the above property, we first recall the definition
of \textbf{unless}
property~\cite{DBLP:books/daglib/0067338,hudon16:_unit_b_method}.
\begin{definition}[Unless Properties]
An \emph{unless property} of the
following form
\begin{center}
  |P| \textbf{unless} |Q|
\end{center}
means that if |P| holds then it will hold continuously unless |Q| hold.
\end{definition}
We restate the \emph{Unless rule} (Theorem 1
in~\cite{hudon16:_unit_b_method}) here.
\begin{theorem}{Unless rule}:
  \label{thm:unless-rule}
  An event system satisfies the unless property |P| \textbf{unless}
  |Q|, if for every event, if it starts in a state satisfying %
  |P & !not Q|, %
  it will reach a state satisfying %
  |P !or Q|.  %
\end{theorem}

Coming back to our example, we first prove that the \EventB model
satisfies the following important \textbf{unless} property.
\begin{theorem}
  \label{thm:unless}
  The drone system satisfies the following \textbf{unless}
  property.

  \begin{center}
    |BATTERYOK = TRUE & charge > 20 & decreaseCharge !: content(eQ) & off /: dt &|\\
    |(! i  !: dom(eQ) . eQ(i) = off => (#j !: dom(eQ) . eQ(j) = decreaseCharge & j < i))|\\
    \textbf{unless} \\
    |BATTERYOK = TRUE & charge > 20 & dt = {decreaseCharge}|\\
  \end{center}
\end{theorem}
\begin{proof}[Sketch]
 The proof relies on Theorem~\ref{thm:unless-rule}, i.e.,
 reasoning per event.  The encoding of the proof obligations are
 available from the Rodin archive. We informally explain why this
 property holds for different class of events below.
 \begin{itemize}
 \item \emph{External events}: The external events raise a new external
   trigger and append the new trigger to |eQ|. Given that
   |decreaseCharge| is already in |eQ|, even if the new trigger is
   |off|, this trigger cannot over take |decreaseCharge| in the queue,
   i.e., it is alway behind |decreaseCharge|. These external events
   therefore maintains the left-hand side of the \textbf{unless} property.
   
 \item \emph{Dequeue external trigger}: If the dequeued external
   trigger is |decreaseCharge|, we will have |dt = {decreaseCharge}|,
   hence we establish the right-hand side of the unless property.
   If it is not |decreaseCharge|, it also cannot be |off| (as any
   |off| trigger in |eQ| has to be behind a |decreaseCharge|
   trigger). As a result, the condition that |off| is behind
   |decreaseCharge| in the remaining queue is maintained and
   |off| cannot be in |dt| after the event execution.
   
 \item \emph{Internal system events}: For the internal systems events,
   we separate them into those that are outside the |BATTERY| state
   and those that are inside of the |BATTERY| state.
   \begin{itemize}
   \item For those that are outside the |BATTERY| state, they maintain
     the left handside of the unless property trivially (by leaving
     external queue unchanged and does not alter the relevant state,
     i.e., |BATTERY| or the |charge|).
     
   \item For the self-transition which is triggered by
     |decreaseCharge| inside the |BATTERY| state, the proof of the
     unless property is trivial, since we assume the negation of the
     right-hand side of the unless property, including that that
     |decreaseCharge| is not in |dt|. For the transition from
     |BATTERYOK| to |BATTERYLOW|, the proof of the unless property is
     also trivial, since we assume the left-hand side of the unless
     property, including that |charge > 20|.
   \end{itemize}
 \end{itemize}
 \qed
\end{proof}
Theorem~\ref{thm:unless} means that the system will continuously
satisfy the following conditions:
\begin{itemize}
\item in the |BATTERYOK| state~,
  
\item |charge| is more than |20|~,
  
\item |decreaseCharge| is in the external queue |eQ|~.
  
\item |off| is not in |dt|~.
  
\item if |off| is in the external queue |eQ| then it is behind a
  |decreaseCharge| trigger~.
\end{itemize}
\textbf{unless} it reaches a state satisfying the following conditions:
\begin{itemize}
\item in the |BATTERYOK| state~,
  
\item |charge| is more than |20|~,
  
\item |decreaseCharge| is in |dt|.
\end{itemize}

Coming back to the proof for our correct-response property and assume
that the system is at the right time and the
|[ExternalTriggerEvent_decreaseCharge]| event happens.  Notice that at
that particular moment, the left-hand side of the progress property in
Theorem~\ref{thm:unless} is also satisfied.  According to
Theorem~\ref{thm:resp-ext-trg}, eventually,
|[dequeueExternalTriggered.decreaseCharge]| is fired (and %
|dt = {decreaseCharge}|), %
i.e., |decreaseCharge| is dequeued from the |eQ| into |dt|. And at
that time, according to Theorem~\ref{thm:unless}, we also have
|BATTERYOK = TRUE| and |charge > 20|.  That ensures the triggered
transition event for |decreaseCharge| is enabled (and it is the only
internal event enabled) and will be eventually taken, hence decrease
the |charge|'s value accordingly.


% \begin{center}
%   |G([ExternalTriggerEvent_decreaseCharge] & {BATTERYOK=TRUE & charge>=10 &|
%     |off/:dt\/content(eQ)}|
%   |=> ({BATTERYOK=TRUE & charge>=10 & off /: dt & decreaseCharge :
%     content(eQ) & (! i . eQ(i) = off => (#j . eQ(j) = decreaseCharge &
%     j < i))} U {decreaseCharge =
%   head(eQ)}|
% \end{center}

% To specify a liveness property to be verified, a special \LTL element is added to the \SCXML model with attributes, property (a string of the above form)  and refinement (an integer indicating the refinement level at which the property should be verified).

% A text file is automatically generated containing the \LTL property to be checked. 
% In this generated version, an assumption of strong fairness is added for all other events in the model.
% Without this assumption, the system may never achieve the expected response to a trigger. 
% Therefore it corresponds to a requirement that the system can always make satisfactory progress and not become live locked.
%This assumption is stronger than necessary since some events will not affect the outcome, but is easier to generate and is sufficient for our verification aim. 
% For simplicity we omit this assumption from the remaining examples.
% \begin{center}
%   |SF[e1] & SF[e2]... => G([external_trigger _event] => F[predicate])|
% \end{center}
% This property can be added into the ProB model checker LTL formula text field.

% However, we could not verify this property.
% The counter example traces that \PROB provided gave us a better understanding of the reasons why. 
% The property as stated is too strong (i.e., not true) for our model; there are additional conditions that need to be considered and added as part of the antecedent.
% \begin{itemize}
% \item
% Our model represented the trigger queues abstractly as sets which meant that the |decreaseCharge| trigger may never be dequeued.
% The standalone version of \PROB allows strong fairness to be specified for particular parameter values but this does not work in the Rodin plug-in for \PROB. 
% In any case, a more accurate (concrete) representation of the queue fixes the problem and improves our model.
% \item 
% The charge is not always decreased in response to the |decreaseCharge| trigger.
% The controller only monitors battery charge while in the |BATTERYOK| state and discards the trigger in other states.
% Also, the controller stops decreasing charge when it approaches 0. 
% To cater for this we added a pre-condition %
% |BATTERYOK = TRUE ∧ charge > 20| %
% to the \LTL property.
% \item
% Even if this pre-condition is true when the trigger is raised, another trigger (e.g., |off|) may already be in the queue and take the controller out of |BATTERYOK| before the |decreaseCharge| trigger is dequeued.
% Again we strengthen the pre-condition |off ∉ dt ∪ eQ| of the \LTL expression to avoid this situation.
% \end{itemize}
% After making these changes the final form of the \LTL property, which \PROB was able to exhaustively check and confirm was as follows:
% \begin{center}
% 	|G([ExternalTriggerEvent_decreaseCharge] & {BATTERYOK=TRUE & charge>=10 &|
% 		|off/:SCXML_dt\/SCXML_eq} => F {charge < old(charge)})|~.
% \end{center}

%%% Local Variables:
%%% mode: latex
%%% TeX-master: "../main"
%%% End:
