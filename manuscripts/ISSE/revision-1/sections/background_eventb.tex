% !TEX root = ../main.tex

\subsection{Event-B}
\label{sec:eventb}

\EventB~\cite{abrial10:_model_event_b,hoang13:_introd_event_b_model_method} is a formal method for system
design.  It uses \emph{refinement} to introduce system details gradually into the
formal model.  An \EventB model contains two parts: \emph{contexts} and \emph{machines}. 
Contexts contain \emph{carrier sets}, \emph{constants}, and \emph{axioms} constraining 
the carrier sets and constants.  Machines contain \emph{variables} |v|, \emph{invariants} |I(v)| 
constraining the variables, and \emph{events}. An event consists of a guard 
denoting its enabled-condition and an action defining the value of variables after the event is executed.  
In general, an event |e| has the form: |any t where G(t, v) then S(t, v) end| where |t| 
are the event parameters, |G(t, v)| is the guard of the event, and |S(t, v)| is the action of the event.
% \begin{center}
  % |any t where G(t, v) then S(t, v) end|
%& \inlineevent{\Be}{}{\Bt}{G(\Bt,\Bv)}{}{S(\Bt,\Bv)}
% \end{center}
%In the case where the event has no parameters, we use the following form
%\begin{center}
%  |when G(v) then S(v) end|
%%& \inlineevent{\Be}{}{}{G(\Bv)}{}{S(\Bv)}~,
%\end{center}
% and when the event has no parameters and no guard, we use
% \begin{center}
%   |begin S(v) end|
% %& \inlineevent{\Be}{}{}{}{}{S(\Bv)}~.
% \end{center}
%The action of an event comprises of one or more assignments, each of them has one of the following forms: (1) |v := E(t, v)|, (2) |v :: E(t, v)|, and (3) |v :∣ P(t, v)|.  Assignments of form (1) are deterministic, assign the value of expression |E(t, v)| to |v|.  Assignments of forms (2) and (3) are non-deterministic. (2) assigns any value from the set |E(t,v)| to |v|, while (3) assigns any value satisfied predicate |P(t,v)| to |v|.
%A machine in \EventB corresponds to a transition system
%where \emph{variables} represent the states and \emph{events} specify
%the transitions.  Note that invariants |I(v)| are inductive, i.e., they must be \emph{maintained} by all events. This is more strict than general safety properties which hold for all reachable states of the \EventB machine.  
% This is also the difference between verifying the consistency of \EventB machines using theorem proving and model checking (e.g., \PROB) techniques: model checkers explore all reachable states of the system while interpreting the invariants as safety properties.  

Machines can be refined by adding more details.  Refinement can be done by extending the machine 
to include additional variables (\emph{superposition refinement}) representing new features of 
the system, or by replacing some (abstract) variables by new (concrete) variables (\emph{data refinement}).  
% More information about \EventB can be found
% in~\cite{hoang13:_introd_event_b_model_method}.
Refinement in \EventB is reasoned on an event basis.  
A (concrete) event |f| refines an (abstract) event |e| if whenever |f| is enabled then |e| is also enabled (guard strengthening), and the action of |f| is the same or equivalent to |e| (where equivalence is given by some relationship defined in the invariants). 
New events are said to refine `skip' (an implicit abstract event that did nothing), and therefore do not alter abstract variables.
 More information about \EventB refinement can be
found in~\cite{abrial10:_model_event_b}.
\EventB is supported by the Rodin Platform (Rodin\footnote{An extensible toolkit which includes 
facilities for modelling, verifying the consistency of models using theorem proving and model 
checking techniques, and validating models with simulation-based approaches.})~\cite{abrial10:_rodin}.

Proof obligations are generated to ensure the consistency of
\mbox{\EventB} models.  An important proof obligation in
\mbox{\EventB} is invariant preservation to prove that safety
properties (encoded as invariants of the models) will not be
violated for any reachable states.  In this paper, we also make use
of other proof obligations in \mbox{\EventB} such as (relative)
deadlock-freeness and (conditional) event convergence to construct
our proof of liveness properties under some fairness
assumptions.%


For the trace semantics corresponding to \mbox{\EventB} machines and
the interpretation of LTL properties over traces, we refer the readers
to \mbox{\cite{hoang2016ltl}}.  Here, we recall the notation for
fairness assumptions underlying event-based formalisms such as
\mbox{\EventB~\cite{lamport1977proving,hudon16:_unit_b_method}}. Given
an event \mbox{\EventBInline{e}}, a weak-fairness assumption
\mbox{\EventBInline{WF(e)}} states that if \mbox{\EventBInline{e}}
is enabled continually, then it must occur infinitely often.
Similarly, a strong-fairness assumption \mbox{\EventBInline{SF(e)}}
states that if \mbox{\EventBInline{e}} is enabled infinitely often,
then it must occur infinitely often. Formally,

\begin{center}
  \EventBInline{WF(a)  <=> (FG enabled(e) => GF [e])}, and

  \EventBInline{SF(a)  <=> (GF enabled(e) => GF [e])},
\end{center}
% \hl{%
  where \mbox{\EventBInline{G}} and \mbox{\EventBInline{F}} are the
  temporal operators denoting \emph{globally}, and \emph{finally},
  respectively; and \mbox{\EventBInline{enabled(e)}} denotes that
  event \mbox{\EventBInline{e}} is enabled and
  \mbox{\EventBInline{[e]}} denotes an occurrence of event
  \mbox{\EventBInline{e}}.
% }
% In \EventB the run to completion pseudocode of Listing~\ref{lst:scxml-r2c} could be represented (somewhat abstractly) as shown in Listing~\ref{lst:eventb-r2c}.
% \begin{lstlisting}[caption={Run to completion pseudocode in \EventB},label={lst:eventb-r2c}, language=Event-B, escapechar=|, frame=single, float=t]
%  FireUntriggered // Fire  enabled un-triggered transitions
% when
%     UC = FALSE // Has not yet completed firing un-triggered transitions
%     untriggered() /= {}
% then
%     execute(untriggered()) // Execute enabled un-triggered transitions
% end
% UntriggeredCompleted // Un-triggered transitions are completed
% when
%     UC = FALSE // Has not yet completed firing un-triggered transitions
%     untriggered() = {} // No more enabled un-triggered transitions
% then
%     UC = TRUE // Complete firing un-triggered transitions
% end
% FireInternallyTriggered // Fire an internal trigger
% when
%     UC = TRUE // Complete firing un-triggered transitions
%     IQ /= {} // The internal triggers queue is non-empty
% then
%     execute(IQ.dequeue) // Execute and dequeue from the internal triggers queue
%     UC := FALSE // Re-enable firing of un-triggered transitions
% end
% FireExternallyTriggered // Fire an external trigger
% when
%     UC = TRUE // Complete firing un-triggered transitions
%     IQ = {} // The internal trigger queue is empty
%     EQ /= {} // The external trigger queue is non-empty
% then
%     execute(EQ.dequeue)  // Execute and dequeue from the external triggers queue
%     UC := FALSE // Re-enable firing of un-triggered transitions
% end
% \end{lstlisting}	
% Here, |IQ| and |EQ| are queues of internally and externally, raised triggers, |untriggered| selects a set of currently enabled un-triggered transitions, |dequeue| retrieves the next trigger from the given queue and selects the set of transitions that become enabled by it and |execute| fires the given set of transitions. 
% Note that this is an abstract representation where each event (|FireUntriggered|, |FireInternallyTriggered|, and |FireExternallyTriggered|) would be specialised to select a particular set of transitions that can be fired in parallel and |execute()| would be replaced by actions that encode the state changes made by those transitions.
% Representing the condition \textbf{untriggered\_enabled} (Line 3 in Listing~\ref{lst:scxml-r2c}) is cumbersome since we would need to write a conjunction of all the possible un-triggered guards. Instead we introduce a dummy un-triggered event that is only fired when no other selection of un-triggered transitions are available and sets a boolean flag, |UC|, to indicate that none of the real un-triggered events was fired and a trigger needs to be consumed.
 
% Note that causing all of the
% transitions to simultaneously and atomically fire for each event is a
% further semantic choice.  Transitions associated with |FireUntriggered|,
% |FireInternallyTriggered|, and |FireExternallyTriggered| might
% just as well fire separately in a non-deterministic order, or by use
% of a per-transition priority, fire in a predetermined order.  Each
% choice has realistic exemplars in the physical world, and to some
% degree, the choice is arbitrary.  The argument in favour of the
% parallel atomic transitions chosen here is pragmatic: the resulting
% representation in \EventB is more terse.

%%% Local Variables:
%%% mode: latex
%%% TeX-master: "../main"
%%% End:
