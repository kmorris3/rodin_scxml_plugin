% !TEX root = ../main.tex

%\section{Related Work}
%\label{sec:relatedWork}

The work we will present here includes three refinement rules.
\begin{enumerate}
\item  \emph{Rule A:} Guard conditions on a transition can be strengthened (but not weakened); 
this can be done by adding textual guards to the transition, or
changing the source of the transition to a nested state.
\item \emph{Rule B:} Transitions can have additional actions, provided they do not
  modify variables appearing in the abstraction; this can be 
  accomplished by adding textual action to the transition 
  or by changing the target to nested state.
\item \emph{Rule C:} A state-chart can be embedded within a state of another
  state-chart -- sometimes called hierarchical composition or
  hierarchical refinement.
\end{enumerate}
Via the translation explained in Section~\ref{sec:translation}, these
rules rely on the usual \EventB proof obligations to ensure that they
do indeed yield refinements in the \EventB semantics.  If an \EventB
model |B| can be shown (via the construction rules of the \EventB
language as well as the proof obligations) to refine another \EventB
model |A|, then we know that every behavior of |B| is also a behavior
of |A|. This definition yields a useful principle of preservation of
safety -- if we can show that a bad thing never happens in |A|, then
we can add detail via refinements in |B|, knowing that the bad thing
will continue to never happen in |B|. That is, \EventB refinements
preserve safety properties in the sense adopted by Lamport
~\cite{lamport1977proving}. This makes refinement a useful technique
in developing safety-critical systems: one can analyze a simpler
abstract model for critical safety properties and then add detail to
the model via refinements, secure in the knowledge that the safety
properties will be preserved. While \EventB refinements have also been
shown to preserve some liveness properties under certain
conditions~\cite{hoang2016ltl}, there are not yet efficient supporting
tools for the technique. Instead, we can express the property in \LTL
and use the \PROB\footnote{ProB is an animator, constraint solver and
  model checker for the B-Method. https://www3.hhu.de/stups/prob}
model checker to verify it, as we have shown in previous
work~\cite{detect2020}.  In this paper, we outline a proof of liveness
properties that relies on reasoning about deadlock-freeness and event
convergence.


Although the autonomous drone example in this paper is based on the
example described in~\cite{Syriani_2019}, the definition of refinement
used in that work is quite different from our own. This forces some
differences in our refinement rules and consequently the way the
example is developed.  In~\cite{Syriani_2019} ``refinement'' is a
transformation of the model which preserves reachability of a state
with respect to sequences of inputs. However, this also allows the
possibility of introducing new behaviors in the concrete model that
the abstraction does not exhibit (more details are in
Section~\ref{sec:descr-sample-appl}). While this notion of refinement
seems useful in certain contexts, unlike refinement in \EventB it does
not guarantee preservation of safety properties. Therefore it should
be considered less suited to development of safety-critical systems.

%%% Local Variables:
%%% mode: latex
%%% TeX-master: "../main"
%%% End: