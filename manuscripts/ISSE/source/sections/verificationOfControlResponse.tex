% !TEX root = ../main.tex
\section{Verification of Control Responses}
\label{sec:verificationResponses}

A model that has been proven to satisfy some invariant (e.g. safety) properties, may still not behave in a useful way.
Therefore, as well as verifying invariant properties, we would like to verify the system's responsiveness.  
That is, we want to ensure that the controller responds to external triggers to make appropriate modifications to the system variables. 
These kind of live responses are difficult to prove via invariant
preservation since they are temporal properties.  In this section, we
present our approach to very the responsive properties of the system.


% Different set of events
In our \EventB model, the events can be separated into the following
categories.
\begin{itemize}
\item \emph{Extenal events}: These events raise external triggers
  including |futureRaiseExternalTrigger|, |ExternalTriggerEvent_toTakeoff|,
  |ExternalTriggerEvent_decreaseCharge|, etc.
  
\item \emph{System events}: Events other than external events are
  called system events. They are the event of the system responding to
  the exernal triggers (creating different runs).  These events can be
  seen in Figure~\ref{fig:basis} and are categorised as follow.
  \begin{itemize}
  \item \emph{Future system events}: These events might internal
    triggers, i.e.,, |futureTriggeredTransitionSet|, and
    |futureUntriggeredTransitionSet|.  The purpose of these events is
    for future introduction of more system details via refinement.
    
  \item \emph{Dequeue External Trigger} (i.e.,
    |dequeueExternalTrigger|): This event dequeue the head of the
    external triggers queue (if any) and will start a run.

  \item \emph{Internal system events}: These events belong to the
    internal of the system to accommodate the runs for external
    extriggers.  These events can be seen in different groups as in
    Figure~\ref{fig:basis}.
    \begin{itemize}
    \item \emph{Dequeue Internal Trigger} (i.e.,
      |dequeueInternalTrigger|). This event dequeue the head of the
      internal triggers queue (if any) and will start a run.
    
    \item \emph{Triggered}. Triggered events corresponding to the
      dequeued (external or internal) trigger.

    \item \emph{Discard Trigger}, i.e.,
      |noTriggeredTransitionsEnabled|.  This event will move system
      from the \emph{Firing Triggered} state to the \emph{Firing
        Untriggered} state in the case where no triggered events are
      enabled.
    
    \item \emph{Untriggered}.  Untriggered events in the systems.

    \item \emph{Completion event}, i.e.,
      |noUntriggeredTransitionsEnabled|. This event will move the
      system from \emph{Firing Untriggered} state to the \emph{Ready
        to De-queue} state in the case where no untriggered events are
      enabled.
    \end{itemize}
  \end{itemize}
\end{itemize}

\begin{theorem}[Internal Events are Relative Deadlock-Free]
  \label{thm:Internal-DLF}
  Under the condition that %
  |iQ /= {} !or uc = FALSE !or dt /= {}|, %
  the internal events are deadlock-free, i.e., there must be
  one internal event is enabled.
\end{theorem}
\begin{proof}
  This is based on the generation of our \EventB model (according to
  the basis structure as seen in Figure~\ref{fig:basis}).  In
  particular, we consider the different cases corresponding to the
  different ``states'', i.e., \emph{Ready to De-queue}, \emph{Firing
    Triggered}, and \emph{Firing Untriggered}.
  \begin{itemize}
  \item When the system is in the \emph{Ready to De-queue} state, %
    |uc = TRUE & dt = {}|.  %
    ccording to our assumption, we have |iQ /= {}|, hence
    |dequeueInternalTrigger| event is enabled.
    
  \item When the system is in the \emph{Firing Triggered} state,
    either one of the triggered events is enabled or the
    |noTriggeredTransitionsEnabled| event is enabled.
    
  \item Similarly, when the system is in the \emph{Firing Untriggered}
      state, either one of the untriggered events is enabled or the
      |noUntriggeredTransitionsEnabled| event is enabled.  
  \end{itemize}
\end{proof}
    
\begin{corollary}[System Events are Relative Deadlock-Free]
  Under the condition that %
  |eQ /= {} !or iQ /= {} !or uc = FALSE !or dt /= {}|, %
  the system events are deadlock-free, i.e., there must be
  one system event is enabled.
\end{corollary}
\begin{proof}
  This is based on the generation of our \EventB model (according to
  the basis structure as seen in Figure~\ref{fig:basis}) and
  Theorem~\ref{thm:Internal-DLF}.
  \begin{itemize}
  \item In the case where |iQ /= {} !or uc = FALSE !or dt /= {}|,
    according to Theorem~\ref{thm:Internal-DLF}, one of the internal
    event is enabled.
    
  \item Otherwise, i.e., |iQ = {} & uc = TRUE & dt = {}|, according to
    our assumption, |eQ /= {}|. In this case, the
    |dequeueExternalTrigger| event is enabled.
\end{itemize}
\end{proof}

In order to reason about any liveness/responsiveness properties, we
have to make assumptions about how often an event will be invoked.
Here, we assume that all the system events are strongly fair.
\begin{assumption}[Strongly Fair System Events]
  \label{asm:SF}
  We assume that the systems event, i.e., |dequeueExternalTrigger|
  and \emph{internal events} are strongly fair, i.e., if a system
  event is enabled infinitely often, eventually, it will be eventually
  invoked.
\end{assumption}

\ColinInlineComment{We have some updates to this section... can we relax SF now we have queues? MAYBE WEAK FAIRNESS IS ENOUGH}
\SonInlineComment{If the model are deterministic, WF would be enough.}

\subsection{Termination of Responses for External Triggers}
\label{sec:contr-rema-resp}

We start first by proving that it is always the case that the system
will comeback to the \emph{Ready to De-queue} state and |iQ = {}|,
i.e., it is ready to dequeue an external trigger (if any).  This is
stated as the following theorem.
\begin{theorem}[Responsive to De-queue External Trigger]
  The system is always eventually empty the internal queue and come to
  the \emph{Ready to De-queue} state.  This is formalised as
  \begin{center}
    |GF(iQ = {} & uc = TRUE & dt = {})|~.
  \end{center}
\end{theorem}
\begin{proof}
  Assuming that the properties is not satisified, i.e., eventually,
  it is always the case that |iQ /= {} !or  uc = FALSE !or dt /=
  {})|. This can be formalised as follows.
  \begin{center}
    |FG(iQ /= {} !or uc = FALSE !or dt /= {})|~.    
  \end{center}
  According to Theorem~\ref{thm:Internal-DLF}, the internal events
  there for will always deadlock-free, and as a result, at least one
  of the event is enabled infinitely often.
\end{proof}


In this sections, that is for every dequeued external trigger,
eventually, the systems will completely process this external trigger
and return to the state where a new external trigger (if any) can be
processed.  Formally, the property can be specified as follows. For
all external trigger |t|, i.e., |t !: ExternalTriggers|,
\begin{center}
  |G(dt = {t} => F(iQ = {} & uc = TRUE & dt = {})|~,
\end{center}
where $G$ denotes the \emph{globally} temporary operator and $F$
denotes the \emph{finally} temporal operator.  Notice that when |uc = TRUE & dt = {}|, the system is in the |READY TO DE-QUEUE| state
(Figure~\ref{fig:basis}).  Given |iQ = {}|, the system will stay in
this |READY TO DE-QUEUE| state until an external trigger is dequeued.
\begin{center}
  |G(t = head(eQ) => F(iQ = {} & uc = TRUE & dt = {})|~,
\end{center}
Subsequently, we can have
\begin{center}
  |G(t !: eQ => F(iQ = {} & uc = TRUE & dt = {})|~,
\end{center}
\SonComment{Number these and create theorems}

\begin{theorem}
  For all external trigger |t !: ExternalTriggers|, we have
  \begin{center}
    |G(dt = {t} => F(iQ = {} & uc = TRUE & dt = {})|~.
  \end{center}
\end{theorem}

\begin{theorem}
  Under the condition |!not(iQ = {} & uc = TRUE & dt = {})|, the system is
  deadlock-free, i.e., there is one internal events
  \SonComment{What are internal events} that is enabled.
\end{theorem}
\begin{proof}
  Construction of the model.
\end{proof}

\begin{corollary}
  Under the condition |!not(iQ = {} & uc = TRUE & dt = {})|, assuming that
  the internal events are strongly fair, eventually, one of the
  internal events is taken.
  \begin{center}
    |G(!not(iQ = {} & uc = TRUE & dt = {})) => F([iE])|~.
  \end{center}
\end{corollary}


\begin{theorem}
  \begin{center}
    |G(!not(iQ = {} & uc = TRUE & dt = {})) => F(!not(e(iE))|~.
  \end{center}
\end{theorem}


\ColinInlineComment{We also need to show termination i.e. iQ=empty  and  uc=TRUE... to do this we need to stop future triggers (FT=empty and disable Future UntriggeredTransitions)  and stop external triggers from taking over! i.e. we need a separate machine that gets rid of all the non-determistic stuff thats there for future refinement. This is ok because we only verify this for the current level of the model.. models (hence refinements) are not guaranteed to have these termination properties}   


\subsection{Correct Responses to External Triggers}
\label{sec:corr-resp-extern}

While \EventB refinements have also been shown to preserve some liveness properties under certain conditions~\cite{hoang2016ltl}, there are not yet efficient supporting tools for the technique.%
\SonComment{We probably need to move this to related work}
Instead, we can express the property in \LTL  and use the \PROB\footnote{ProB is an animator, constraint solver and model checker for the B-Method. https://www3.hhu.de/stups/prob} model checker to verify it.

In general, our liveness properties will have the following form:
\begin{center}
  |G([external_trigger_event] => F{predicate})|~,
\end{center}
where the predicate concerns variables |v| that the system maintains, and may refer to old values |old(v)| that existed when the external trigger occurred.
To specify a liveness property to be verified, a special \LTL element is added to the \SCXML model with attributes, property (a string of the above form)  and refinement (an integer indicating the refinement level at which the property should be verified).
The translator generates a separate `branch' refinement for each \LTL property to be verified. 
In this special refinement, history variables are added to record the value at the state when the external trigger occurs, of any variables that are referenced as `old' values.
A text file is automatically generated containing the \LTL property to be checked. 
In this generated version, an assumption of strong fairness is added for all other events in the model.
Without this assumption, the system may never achieve the expected response to a trigger. 
Therefore it corresponds to a requirement that the system can always make satisfactory progress and not become live locked.
%This assumption is stronger than necessary since some events will not affect the outcome, but is easier to generate and is sufficient for our verification aim. 
For simplicity we omit this assumption from the remaining examples.
\begin{center}
  |SF[e1] & SF[e2]... => G([external_trigger _event] => F[predicate])|
\end{center}
This property can be added into the ProB model checker LTL formula text field.

We illustrate the method with an example of a temporal property that we expect to hold in the drone \SCXML system. 
The liveness property that we wish to  verify is that, after an external trigger event |decreaseCharge|, the battery charge value should  decrease in value.
\begin{center}
  |G ([ExternalTriggerEvent_decreaseCharge] => F {charge < old(charge)})|~.
\end{center}
However, we could not verify this property.
The counter example traces that \PROB provided gave us a better understanding of the reasons why. 
The property as stated is too strong (i.e. not true) for our model; there are additional conditions that need to be considered and added as part of the antecedent.
\begin{itemize}
\item
Our model represented the trigger queues abstractly as sets which meant that the |decreaseCharge| trigger may never be dequeued.
The standalone version of \PROB allows strong fairness to be specified for particular parameter values but this does not work in the Rodin plug-in for \PROB. 
In any case, a more accurate (concrete) representation of the queue fixes the problem and improves our model.
\item 
The charge is not always decreased in response to the |decreaseCharge| trigger.
The controller only monitors battery charge while in the |BATTERYOK| state and discards the trigger in other states.
Also, the controller stops decreasing charge when it approaches 0. 
To cater for this we added a pre-condition |BATTERYOK = TRUE ∧ charge ≥10| to the \LTL property.
\item
Even if this pre-condition is true when the trigger is raised, another trigger (e.g. |off|) may already be in the queue and take the controller out of |BATTERYOK| before the |decreaseCharge| trigger is dequeued.
Again we strengthen the pre-condition |off ∉ dt ∪ eQ| of the \LTL expression to avoid this situation.
\end{itemize}
After making these changes the final form of the \LTL property, which \PROB was able to exhaustively check and confirm was as follows:
\begin{center}
	|G([ExternalTriggerEvent_decreaseCharge] & {BATTERYOK=TRUE & charge>=10 &|
		|off/:SCXML_dt\/SCXML_eq} => F {charge < old(charge)})|~.
\end{center}


%%% Local Variables:
%%% mode: latex
%%% TeX-master: "../main"
%%% End:
