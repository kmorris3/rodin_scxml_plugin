% !TEX root = ../main.tex
\section{Verification of Control Responses}
\label{sec:verificationResponses}

A model that has been proven to satisfy some invariant (e.g. safety) properties, may still not behave in a useful way.
Therefore, as well as verifying invariant properties, we would like to verify the system's responsiveness.  
That is, we want to ensure that the controller responds to external triggers to make appropriate modifications to the system variables. 
These kind of live responses are difficult to prove via invariant
preservation since they are temporal properties.  In this section, we
present our approach to very the responsive properties of the system.


% Different set of events
In our \EventB model, the events can be separated into the following
categories
\begin{itemize}
\item \emph{Extenal events}: These events raise external triggers,
  e.g. |ExternalTriggerEvent_toTakeoff|,
  |ExternalTriggerEvent_decreaseCharge| etc.
  
\item \emph{Future events}: These events might raise external triggers or
  internal triggers, i.e., |SCXML_futureRaiseExternalTrigger|,
  |SCXML_futureTriggeredTransitionSet|, and
  |SCXML_futureUntriggeredTransitionSet|.  The purpose of these events
  is for future introduction of more system details via refinement.
  
\item \emph{Dequeue External Trigger} (i.e.,
  |SCXML_dequeueExternalTrigger|): This event dequeue the head of the
  external triggers queue (if any) and will start a run. This event
  can be seen in Figure~\ref{fig:basis}.

\item \emph{Internal events}: These events belong to the internal of
  the system to accommodate the runs for external extriggers.  These
  events can be seen in different groups as in Figure~\ref{fig:basis}.
  \begin{itemize}
  \item \emph{Dequeue Internal Trigger} (i.e.,
    |SCXML_dequeueInternalTrigger|)
    
  \item \emph{Fire Triggered}. Triggered events corresponding
    to the dequeued (external or internal) trigger.

  \item \emph{No Triggered Enabled}, i.e., |SCXML_noTriggeredTransitionsEnabled|. 
    This event will move system from the \emph{Firing Triggered} state to
    the \emph{Firing Untriggered} state in the case where no triggered
    events are enabled.
    
  \item \emph{Fire Untriggered}.  Untriggered events in the systems.

  \item \emph{Completion event}, i.e.,
    |SCXML_noUntriggeredTransitionsEnabled|. This event will move the
    system from \emph{Firing Untriggered} state to the \emph{Ready to
      Dequeue} state in the case where no untriggered events are enabled.
\end{itemize}
\end{itemize}
\SonInlineComment{We only need to ignore the futureRaisedTriggers events}



\ColinInlineComment{We have some updates to this section... can we relax SF now we have queues? MAYBE WEAK FAIRNESS IS ENOUGH}
\SonInlineComment{What fairness assumption that we are going to use here.}

\subsection{Termination of Responses for External Triggers}
\label{sec:contr-rema-resp}

In this sections, we discuss our verification for the termination of
responses to external triggers, that is for every dequeued external
trigger, eventually, the systems will completely process this external
trigger and return to the state where a new external trigger (if any)
can be processed.  Formally, the property can be specified as
follows. For all external trigger |t|, i.e., |t !: ExternalTriggers|, 
\begin{center}
  |G(dt = {t} => F(iQ = {} & uc = TRUE & dt = {})|~,
\end{center}
where $G$ denotes the \emph{globally} temporary operator and $F$
denotes the \emph{finally} temporal operator.  Notice that when |uc = TRUE & dt = {}|, the system is in the |READY TO DE-QUEUE| state
(Figure~\ref{fig:basis}).  Given |iQ = {}|, the system will stay in
this |READY TO DE-QUEUE| state until an external trigger is dequeued.
\begin{center}
  |G(t = head(eQ) => F(iQ = {} & uc = TRUE & dt = {})|~,
\end{center}
Subsequently, we can have
\begin{center}
  |G(t !: eQ => F(iQ = {} & uc = TRUE & dt = {})|~,
\end{center}
\SonComment{Number these and create theorems}

\begin{theorem}
  For all external trigger |t !: ExternalTriggers|, we have
  \begin{center}
    |G(dt = {t} => F(iQ = {} & uc = TRUE & dt = {})|~.
  \end{center}
\end{theorem}

\begin{theorem}
  Under the condition |!not(iQ = {} & uc = TRUE & dt = {})|, the system is
  deadlock-free, i.e., there is one internal events
  \SonComment{What are internal events} that is enabled.
\end{theorem}
\begin{proof}
  Construction of the model.
\end{proof}

\begin{corollary}
  Under the condition |!not(iQ = {} & uc = TRUE & dt = {})|, assuming that
  the internal events are strongly fair, eventually, one of the
  internal events is taken.
  \begin{center}
    |G(!not(iQ = {} & uc = TRUE & dt = {})) => F([iE])|~.
  \end{center}
\end{corollary}


\begin{theorem}
  \begin{center}
    |G(!not(iQ = {} & uc = TRUE & dt = {})) => F(!not(e(iE))|~.
  \end{center}
\end{theorem}


\ColinInlineComment{We also need to show termination i.e. iQ=empty  and  uc=TRUE... to do this we need to stop future triggers (FT=empty and disable Future UntriggeredTransitions)  and stop external triggers from taking over! i.e. we need a separate machine that gets rid of all the non-determistic stuff thats there for future refinement. This is ok because we only verify this for the current level of the model.. models (hence refinements) are not guaranteed to have these termination properties}   


\subsection{Correct Responses to External Triggers}
\label{sec:corr-resp-extern}

While \EventB refinements have also been shown to preserve some liveness properties under certain conditions~\cite{hoang2016ltl}, there are not yet efficient supporting tools for the technique.%
\SonComment{We probably need to move this to related work}
Instead, we can express the property in \LTL  and use the \PROB\footnote{ProB is an animator, constraint solver and model checker for the B-Method. https://www3.hhu.de/stups/prob} model checker to verify it.

In general, our liveness properties will have the following form:
\begin{center}
  |G([external_trigger_event] => F{predicate})|~,
\end{center}
where the predicate concerns variables |v| that the system maintains, and may refer to old values |old(v)| that existed when the external trigger occurred.
To specify a liveness property to be verified, a special \LTL element is added to the \SCXML model with attributes, property (a string of the above form)  and refinement (an integer indicating the refinement level at which the property should be verified).
The translator generates a separate `branch' refinement for each \LTL property to be verified. 
In this special refinement, history variables are added to record the value at the state when the external trigger occurs, of any variables that are referenced as `old' values.
A text file is automatically generated containing the \LTL property to be checked. 
In this generated version, an assumption of strong fairness is added for all other events in the model.
Without this assumption, the system may never achieve the expected response to a trigger. 
Therefore it corresponds to a requirement that the system can always make satisfactory progress and not become live locked.
%This assumption is stronger than necessary since some events will not affect the outcome, but is easier to generate and is sufficient for our verification aim. 
For simplicity we omit this assumption from the remaining examples.
\begin{center}
  |SF[e1] & SF[e2]... => G([external_trigger _event] => F[predicate])|
\end{center}
This property can be added into the ProB model checker LTL formula text field.

We illustrate the method with an example of a temporal property that we expect to hold in the drone \SCXML system. 
The liveness property that we wish to  verify is that, after an external trigger event |decreaseCharge|, the battery charge value should  decrease in value.
\begin{center}
  |G ([ExternalTriggerEvent_decreaseCharge] => F {charge < old(charge)})|~.
\end{center}
However, we could not verify this property.
The counter example traces that \PROB provided gave us a better understanding of the reasons why. 
The property as stated is too strong (i.e. not true) for our model; there are additional conditions that need to be considered and added as part of the antecedent.
\begin{itemize}
\item
Our model represented the trigger queues abstractly as sets which meant that the |decreaseCharge| trigger may never be dequeued.
The standalone version of \PROB allows strong fairness to be specified for particular parameter values but this does not work in the Rodin plug-in for \PROB. 
In any case, a more accurate (concrete) representation of the queue fixes the problem and improves our model.
\item 
The charge is not always decreased in response to the |decreaseCharge| trigger.
The controller only monitors battery charge while in the |BATTERYOK| state and discards the trigger in other states.
Also, the controller stops decreasing charge when it approaches 0. 
To cater for this we added a pre-condition |BATTERYOK = TRUE ∧ charge ≥10| to the \LTL property.
\item
Even if this pre-condition is true when the trigger is raised, another trigger (e.g. |off|) may already be in the queue and take the controller out of |BATTERYOK| before the |decreaseCharge| trigger is dequeued.
Again we strengthen the pre-condition |off ∉ dt ∪ eQ| of the \LTL expression to avoid this situation.
\end{itemize}
After making these changes the final form of the \LTL property, which \PROB was able to exhaustively check and confirm was as follows:
\begin{center}
	|G([ExternalTriggerEvent_decreaseCharge] & {BATTERYOK=TRUE & charge>=10 &|
		|off/:SCXML_dt\/SCXML_eq} => F {charge < old(charge)})|~.
\end{center}


%%% Local Variables:
%%% mode: latex
%%% TeX-master: "../main"
%%% End:
