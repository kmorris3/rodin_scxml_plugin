\documentclass{response}

\papertitle{Refinement of Statecharts with Run-to-Completion Semantics}
\paperauthor{K. Moris, C. Snook, T.S.Hoang, R. Armstrong, and M. Butler}
\usepackage{chgtrk}
\newCTcontributor{Karla}
\newCTcontributor{Colin}
\newCTcontributor{Rob}
\newCTcontributor{Son}
\newCTcontributor{Michael}

\begin{document}

%%%%% Reviewer 1 (Begin) %%%%%
% Overall evaluation: 1 (weak accept)

% ----------- Overall evaluation -----------
% The paper deals with an evergreen topic of verifying state-based
% models, namely statecharts. The authors extended the well-know
% Event-B toolset to be able to support the 'run-to-completion'
% semantics of modern statechart languages. Besides the theoretical
% results, the authors also provide the algorithms in the tool. The
% paper also contains the necessary amount of figures and examples.
% The related work is also comprehensive enough, it proves that the
% authors are familiar with the literature.

\begin{comment}{Reviewer \#1}
Contribution:
Authors present an encoding scheme to be able to handle the
abstraction (and refinement) of statechart models. The contribution is
mainly focused onto the special encoding and the implementation. The
contribution is enough for a workshop paper and it could be a nice
piece of work to be shown in the workshop.
However, the approach seems not to scale well, so I would suggest the
authors evaluate their work using more complex (industrial)
verification problems.
\end{comment}

\begin{response}
  \SonInlineComment{Future work? Which complex problem?}
\end{response}

% Writing: The paper is easy to read, easy to follow. Except for minor
% typos, it has good quality.

% My overall opinion is that I suggest this paper be presented at the
% workshop.

\begin{comment}{Reviewer \#1}
  Typos and suggestions:

  an intermediate refinement level by translation into Event-B --> an
  intermediate refinement level by A translation into Event-B

  Trigger events are queued when they are raised and then one is
  dequeued --> Trigger events are queued when they are raised, and
  then one is dequeued

  This is repeated until no transitions are enabled and then the next
  trigger --> This is repeated until no transitions are enabled, and
  then the next trigger

  The ASIC starts by initialising the buzzer, this involves sending
  --> The ASIC starts by initialising the buzzer; this involves
  sending

  In practice we wish to leverage --> In practice, we wish to leverage
  for the messages send between the system components --> for the
  messages sent between the system components

  right hand region --> right-hand region
\end{comment}
\begin{response}
  Done. Thank you very much.
  \SonInlineComment[Colin]{Can you check if this is done}
\end{response}
%%%%% Reviewer 1 (End) %%%%%

%%%%% Reviewer 2 (Begin) %%%%%
% Overall evaluation: -1 (weak reject)

% ----------- Overall evaluation -----------
% Overview:
% =========
% The paper describes an approach to represent Statecharts semantics
% in iUML-B, a diagrammatic notation for Event-B.  The focus in this
% work is supporting hierarchical refinement, that is, describing
% nested behavior on a previously "leaf-level" state.   The
% Statecharts semantics represented are similar to UML Statecharts.

% Section 2 describes background.  Section 2.1 briefly describes the
% run-to-completion semantics of SCXML (StateCharts XML); Section 2.2
% describes Event B, focusing on refinement, and Section 2.3 describes
% iUML-B State machines, which have a direct translation into Event B
% specifications.  Although the iUML-B state machines look much like
% Statecharts, they do not share the run-to-completion or transition
% triggering semantics.

% Section 3 describes an example model of an intrusion detection
% system initialization sequence.  The model is constructed first as a
% sequential state machine that describes the initialization phases,
% then is refined twice: the first refinement introduces the
% communication SPI bus, and  second is a hierarchical refinement that
% describes in more detail the behavior of each initialization phase.
% Throughout the refinements, we would like to preserve an invariant
% that the initialization phase is in 'go' only if the SPI bus is
% idle.

% Section 4 describes refinements for this notion of StateCharts and
% their relation to Event-B.  The goal is to have a notion of proof
% that is preserved  by hierarchical refinement: we can strengthen
% transition firing conditions, add nested statecharts, and allowing
% additional triggers for internal actions (which are viewed as
% non-deterministic in child context where they may fire).  It
% provides illustrations of how the interplay between micro/macrosteps
% leads to difficulties in terms of Event-B refinement: a strengthened
% transition guard may lead to a system in which fewer microsteps are
% required, weakening the guard on new macrosteps.  To prevent such
% weakenings, the system uses a non-deterministic semantics, in which
% enabled transitions may fire, but do not need to fire, and it is
% always possible that no transitions fire.  An event is generated for
% each possible combination of transitions that may occur.

% In order to support this translation, the SCXML definition is
% extended to support refinement, invariant, and guard (to allow
% refinement of transition labels at lower levels of abstraction).

% Section 5 describes the translation in more detail. This explanation
% is at the level of a prose description, other than for the pieces of
% the translation related to triggering events.  I suspect that these
% aspects have been dealt with in previous papers, but in this case
% the authors should have provided citations to the work containing
% the definitions.  While it is possible to get an outline of the
% translation scheme, it is difficult to understand it in full
% generality.

% Section 6 describes proofs of the intrusion detection system using
% Rodin (the Event-B prover).  In order to prove the property under
% future refinements, one must add a series of invariants to states
% that limit the behavior of future refinements.  In the case of this
% model, a handful of such invariants are required.

% Section 7 concludes.


% Review:
% =======
% The paper is well written and describes interesting work towards
% creating a Statecharts semantics that both supports
% run-to-completion semantics and hierarchical refinement.  For the
% most part, the writing is clear and easy to understand.

\begin{comment}{Reviewer \#2}
  Nevertheless, there are some significant problems with
  the work from my perspective, starting from motivation. There have
  been many, many StateCharts semantics proposed over 20 years and any
  proposal for new semantics much have a clear reason for existing.
  The current approach, while having novelty in that it both supports
  hierarchical refinement and run-to-completion behavior, needs a
  stronger motivation in terms of solving design challenges faced by
  real engineers.
\end{comment}


\begin{comment}{Reviewer \#2}
  The current semantics is one of non-determinism as to which events
  will fire in a given microstep.  While this allows refinement, it
  does not admit a deterministic execution strategy, which means that
  it is not currently useful for generation of implementations.  In
  addition, it appears that it is possible that events can be "lost"
  and the system may choose not to execute them.  The authors must
  provide additional justification that this is a sensible semantics;
  I suspect that it will be surprising to most users who are familiar
  with StateCharts.  When adding transitions at lower layers, it was
  not clear to me how these new transitions fit into the cross-product
  translation used to determine which set of transitions would fire.
  Perhaps this is handled layer-by-layer, but then each layer would
  need a scheduler and this was not discussed.   
\end{comment}

\begin{comment}{Reviewer \#2}
  The goal appears to be to try to create something that resembles
  StateCharts but is bound to the semantics of Event-B.  In this case,
  the authors should also justify why Event-B is a suitable language
  for representing Statecharts behavior.  In this case, it appeared
  that the target language made some aspects of behavior difficult to
  formalize.  
\end{comment}

\begin{comment}{Reviewer \#2}
  In addition, I would have liked a more formal description of the
  translation, or at least a reference to a more formal description of
  the translation than the prose description of different model
  aspects.  
\end{comment}

%%%%% Reviewer 2 (End) %%%%%

%%%%% Reviewer 3 (Begin) %%%%%

% Overall evaluation: 0 (borderline paper)

% ----------- Overall evaluation -----------
% This is an interesting subject and approach. The presentation is not
% that convincing though, it is slightly hard to follow the details.

\begin{comment}{Reviewer \#3}
  Page 1:

  * Line -2, it says:

  "While functional properties (usually) can be tested, safety,
  security and reliability properties
  (usually) must be proved formally."

  "Can" ... and ... "must" ... why?
\end{comment}

\begin{comment}{Reviewer \#3}
  Page 2:

  * Top: The first paragraph can be expressed more clearly. That is:
  what is the approach taken? Is UML state charts mapped to iUML-B or
  not?

  * Second paragraph appears complicated writing.
\end{comment}


\begin{comment}{Reviewer \#3}
  Page 3:

* Listing 1: I think I would name that variable "run2completion" differently.
\end{comment}

\begin{comment}{Reviewer \#3}
  Page 4:

  * Mid, it is not clear what this means:

  "The diagrammatic models are contained within an Event-B machine"

  and this as well:

 "while Event-B events are expected to already exist to represent the transitions."
\end{comment}

\begin{comment}{Reviewer \#3}
  Page 5:

  * It would seem more natural to have one state variable ranging over
  an enumerated type of states, rather than having a Boolean flag for
  each state.

  * Section 3: be more clear about what this system does. Is it just a
  buzz that goes off when a sensor senses something?

  * Mid:

  "At this abstraction the spi_done triggered, which ..."
  ->
  " At this abstraction there is only one event, spi_done, which ..."

  * Line -9: "on an Idle state" -> "in an Idle state".

* Line -6: "while SPI subsystem ..." -> "while the SPI subsystem ..."
\end{comment}

\begin{comment}{Reviewer \#3}
  Page 6:

  * Figure 2: this looks like a way of modeling event refinement,
  which seems different from state refinement. Perhaps comment on
  this.

  * Line -7: "This safety property is introduced ..." : where, when?

* Figure 2(b): "last_byte_send" -> "last_byte_sent".
\end{comment}

\begin{comment}{Reviewer \#3}
  Page 7:

  * Mid: "Ancillary data, with corresponding actions to alter it, can
  ..." - something is wrong with this sentence.

* Line -4: I don't understand: "Note that external triggers are always unguarded ...".
\end{comment}

\begin{comment}{Reviewer \#3}
  Page 8:

* The next couple of pages is a lot of English without some structure to hang it up on.
\end{comment}

\begin{comment}{Reviewer \#3}
  Page 9:

* The three bullets (refinement, invariant, guard): there is mention of the parent state. Not sure I understand.
\end{comment}

\begin{comment}{Reviewer \#3}
  Page 10:

* In general, the explanation of the example could be improved a lot.
\end{comment}

\begin{comment}{Reviewer \#3}
  Page 12:

* I had a hard time reading this.
\end{comment}

\begin{comment}{Reviewer \#3}
  Page 14:

* Mid: "SCXML future..TransitionSet" - is this correct?
\end{comment}
%%%%% Reviewer 3 (End) %%%%%

\end{document}
