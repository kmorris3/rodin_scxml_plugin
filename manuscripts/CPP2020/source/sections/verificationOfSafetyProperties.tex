% !TEX root = ../main.tex


\section{Verification of Safety Properties}

In a statechart model we naturally wish to verify properties P that are expected to hold true in a particular state S.
Hence, all of the safety properties that we consider are of the form: S=TRUE => P, where the antecedent is implied by the containment of P within S.

There are two kinds of properties that we might want to verify in a statechart;
1) properties concerning the values of auxiliary data which is being maintained by the system and 2) constraints on the state of a parallel statechart region.

SCXML models represent components that react to received triggers and cannot be perfectly synchronised with changes to the monitored properties. 
Hence, if naively expressed, P may be temporarily violated until the system reacts by leaving the state S in which the property is expected to hold.
To cater for this we express P in a modified form P' that allows time for the reaction to take place.

There are two forms of reaction that can be used to exit S; a) an untriggered transition, or b) a transition that is triggered by an internally raised trigger.
For a), the modified property P' is \emph{untriggered transitions are not complete} or P, and for b) P' is \emph{trigger t is in the internal queue or dequeued} or P (where t is the particular trigger needed to react to the violation of P).

For properties about the value of auxiliary data, untriggered transitions appear to be more suited because, in this case, there is unlikely to be a natural place to raise an internal trigger when the appropriate conditions arise.
For properties about the state of a parallel region either reaction could be used depending on whether the system detects the violation in the state that contains P of the state that P refers to.

We illustrate an example where some auxiliary data is monitored by one statechart  and another statechart utilises this by referring to the state of the monitor. 
Hence the reaction consists of an untriggered transition in the monitoring statechart that sends an internal trigger to the other statechart when it leaves the desired monitoring state.







