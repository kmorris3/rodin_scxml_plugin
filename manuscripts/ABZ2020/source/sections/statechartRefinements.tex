% !TEX root = ../main.tex

\section{Statechart Refinement}
\label{sec:scref}

Our system includes three refinement rules.

\todo{Geoff: these need more detail probably.}
\begin{enumerate}
\item Guard conditions on a transition can be strengthened; 
this can done by adding textual guards to the transition, or
changing the source of the transition to a nested state.
\item Transitions can have additional actions, provided they do not
  modify variables appearing in the abstraction; this can be 
  acoomplished by adding texual action to the transition 
  or by changing the target to nested state.
\item A statechart can be embedded within a state of another
  statechart -- sometimes called hierarchical composition or
  hierarchical refinement.
\end{enumerate}

Via the translation explained in Section~\ref{sec:translation}, these rules
rely on the usual Event-B proof obligations to ensure that they do
indeed yield refinements in the Event-B semantics.

If an Event-B model |B| can be shown (via the construction rules of
the Event-B language as well as the proof obligations) to refine
another Event-B model |A|, then we know that every behavior of |B| is
also a behavior of |A|. This definition yields a useful principle of
preservation of safety -- if we can show that a bad thing never
happens in |A|, then we can add detail via refinements in |B|, knowing
that the bad thing will continue to never happen in |B|. That is,
Event-B refinements preserve safety properties in the sense
of~\cite{lamport1977proving}. This makes refinement a useful technique
in developing safety-critical systems: one can analyze a simpler
abstract model for critical safety properties and then add detail to
the model via refinements, secure in the knowledge that the safety
properties will be preserved.

Event-B refinements have also been shown to preserve some liveness
properties, under certain conditions~\cite{hoang2016ltl}.

Although the autonomous drone example in this paper is based on the
example described in~\cite{Syriani_2019}, the definition of refinement
used in that work is quite different from our own. This forces some
differences in our refinement rules and consequently the way the
example is developed. For example, the transition triggered by |toLand| 
(see ~\ref{fig:drone4} transition from |TAKEOFF| to |DESCEND|) 
adds new behaviour which enables the drone to go directly from 
|TAKEOFF| to |DESCEND| with going through |FLY|.
In~\cite{Syriani_2019} ``refinement'' is a
transformation of the model which preserves reachability of a state
with respect to sequences of inputs. This guarantees that a refinement
in their system will include all the behaviors of the abstraction,
while posssibly adding others. While this notion of refinement seems
useful in certain contexts, unlike refinement in Event-B it does not
guarantee preservation of safety properties. Therefore it should be
considered less suited to development of safety-critical systems.

\todo{Geoff: point out specific differences in the example (e.g., toLand).}
