% !TEX root = ../main.tex


\section{Introduction}

Possible outline:
\begin{itemize}
\item Motivation
\item Background and Previous Work
\item Description of Sample Application
\item Statechart Refinements
\item Verification of Safety Properties 
\item Results
\item Conclusion 
\end{itemize}

\todo{Rob is working on the intro}
Ideas on content
We prove properties in early refinements - because easier to see - (so far fully automatic - try proving battery inv in later refinement)
	although we put everything in the same model we envisage gradual construction	
	we could also build a Sirius tool that can show views at each refinement.

We simulate the translation with the scenario checker. It automatically runs to completion (events are annotated as internal and prioritised)
This shows the main intended R2C behaviour for a refinement level.
Other behaviours are possible depending on future refinement - e.g. a completion can occur without taking enabled transitions if in future we might strengthen a guard.
	these can be explored by single event stepping without the scenario checker big step.

proof forces us to finalise transitions..
	.. i.e. to remove non-determinism that allows for some future refinement.
		(in usual event-b it is not built in to the 'engine' and is only checked when the next refinement is provided..
		.. we only have this finalisation because we needed to model the run to completion queueing mechanisms, hence new transitions modify old variables (queues))
	we have to limit what can be done in later refinements as some things could violate what is proved.
	e.g. a response transition must come next (to make the invariant before completion) so we cannot strengthen its guards any more.
	e.g. similarly a trigger raised as a response must be used in the way that preserves the invariant, so future transitions may not consume it.

Also section on how we arrived at the new translation.
	adding the invariant about only one dequeued trigger at a time - otherwise proof did not discharge about the future triggered transitions
		(because consuming another dequeued trigger would discard the one we are interested in)
	the drone model made us rethink the basis.. we had to dequeue triggers.. (why?)
