% !TEX root = ../main.tex


\section{Description of Sample Application}

To illustrate the development and analysis process of a design using the previously described 
statechart semantics, we will discuss a quadrotor helicopter or quadrotor application similar to 
the one presented by Syriani et al.~\cite{Syriani_2019}.The application will focus on the incremental 
design of some of the drone's required functionality.
The model constructed following statechart refinement rules that are proven within the Rodin tool.
The structure of the statechart for this model at each subsequent abstraction level restrict farther 
development of the model to refinements that obey the rules. This will allow us to prove properties 
of the model in a very strategic fashion, as properties proven of early abstraction levels 
are preserve in later refinements.

The first abstraction of the model shown in figure~\ref{fig:drone1} captures the basic 
functionality of the drone. The model initial state is |OFF| and as a result of the |on| and 
|toTakeoff| external triggers it transitions to the |START| and |OPERATIONAL| respectively. 
The drone reacts to the |off| external trigger by shutting down and a subsequent transition to |OFF|.
Within the |OPERATIONAL| state the drone will transition to |FLY|, |DESCEND| or |LANDED| 
after the internal triggers |toFly|, |toLand| or |landed|. In this abstraction these internal 
triggers are raised non-deterministically in the system by functionality not currently defined.
As additional details are incorporated into the model in later refinements some of that non-determinism is 
removed and replaced transitions with actions that raised the previously defined internal triggers.

\begin{figure}[!h]
	\vspace{-.4cm}
	\centering
	\includegraphics[width=0.95\textwidth]{figures/Picture1.png}
	\caption{Statechart of drone application. Abstract level including only generic behavior }
	\label{fig:drone1}
	\vspace{-.4cm}
\end{figure} 

Figure~\ref{fig:drone2} shows the first refinement of the model, as we refine the parent state |TAKEOFF|
by introducing child states and new model variables, similar to the rule 1 and 2 defined by Syriani et al.
As part of this refinement we introduced an untriggered transition responsible for 
raising the |toFly| internal trigger, and therefore removed some of the non-determinisms in the abstraction.

\begin{figure}[!h]
	\centering
	\includegraphics[width=0.95\textwidth]{figures/Picture2.png}
	\caption{Statechart of drone application. Refinement level introducing details for TakeOff}
	\label{fig:drone2}
\end{figure} 

The second refinement, figure~\ref{fig:drone3} extends the capabilities within |OPERATIONAL| by making it a parallel
state that controls flying and battery related functionality. This is same as rule 4 defined by Syriani et al.
Our statechart semantics support transition refinement when constrained to adding guards and/or 
actions to previously defined transitions. These new strengthening of guards, or additional 
actions are expression in term on new model variables that contribute implementation details to the model.

\begin{figure}[!h]
	\centering
	\includegraphics[width=0.95\textwidth]{figures/Picture3.png}
	\caption{Statechart of drone application.Refinement level for descending capabilities}
	\label{fig:drone3}
\end{figure} 

Figure~\ref{fig:drone4} shows

\begin{figure}[!h]
	\centering
	\includegraphics[width=0.95\textwidth]{figures/Picture4.png}
	\caption{Statechart of drone application. Refinement level for battery consumption functionality}
	\label{fig:drone4}
\end{figure} 
% \emph{Describe the Drone case study including refinements and things we would verify}
