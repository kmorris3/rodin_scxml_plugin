% !TEX root = ../SCXMLREF.tex

\section{Tooling}
\label{sec:tooling}

A tool to automatically translate \SCXML models into \iUMLB has been produced. The tool is based on the Eclipse Modelling Framework (EMF) and uses a an \SCXML metamodel provided by Sirius~\cite{siriuswebsite} which has good support for extensibility. The tooling for \iUMLB and \EventB already contains EMF metamodels and provides a generic translator framework which has been specialised for the \SCXML to \iUMLB translation.  
 
The following syntax extensions are added to \SCXML models to support modelling features needed in \iUMLB/\EventB. These extensions are prefixed with `iumlb:' in order to distinguish them from the scxml XML parser. (So that they are ignored by \SCXML simulation tools). 
%They are loaded by EMF as generic feature maps (‘Any’ for contained elements and ‘AnyAttribute’ for attributes).
\begin{itemize}
	\item \textbf{iumlb:refinement} - an integer attribute representing the refinement level at which the parent element should be introduced.
	\item \textbf{iumlb:invariant} - an element that generates an invariant in \iUMLB. This provides a way to add invariants to states so that important properties concerning the synchronisation of state with ancilliary data and other statemachines can be expressed.
	\item \textbf{iumlb:guard} - an element that generates a transition guard in \iUMLB. 
	This provides a way to add new guard conditions to transitions over several refinement as well as providing an element with attributes such as derived (for \EventB theroems), name and comment.
	\item \textbf{iumlb:predicate} - a string attribute used for the predicate of a guard or invariant.
	\item \ldots other attributes useful for \iUMLB elements: name, derived, type, comment.
\end{itemize}

Hierarchical nested state charts are translated into similarly structured \iUMLB state-machines. The generated \iUMLB model contains refinements that add nested state-machines as indicated in the  \SCXML state-chart by the \textbf{iumlb:refinement} attributes annotated on state elements. \iUMLB transitions are generated for each \SCXML transition and linked to \EventB events that represent each of the possible synchronisations that could involve that transition.



