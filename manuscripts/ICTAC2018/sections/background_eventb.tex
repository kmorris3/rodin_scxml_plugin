% !TEX root = ../SCXMLREF.tex

\subsection{Event-B}
\label{sec:eventb}

\ColinCommented{we can copy from a previous paper}
Overview of Event-B... semantics, refinement, proof obligations, tools..

\ldots

In Event-B the run to completion pseudocode of Listing~\ref{lst:scxml-r2c} could be represented (somewhat abstractly) as

\begin{Bcode}
	$%
	\event{FireUntriggered}{}{}{UC=FALSE}{}{execute(untriggered())}
	\event{FireInternallyTriggered}{}{}{UC=TRUE\\IQ\neq\emptyset}{}{execute(IQ.dequeue)\\UC:=FALSE}
	\event{FireExternallyTriggered}{}{}{UC=TRUE\\IQ=\emptyset\\EQ\neq\emptyset}{}{execute(EQ.dequeue)\\UC:=FALSE}
	$
%	\Bvspace[2ex]
\end{Bcode}

Note that this is an abstract representation where each event would be specialised to select a particular set of transitions that can be fired in parallel and execute would be replaced by actions that encode the state changes made by those transitions.
Representing the condition ‘untriggered enabled’ is cumbersome since we would need to write a conjunction of all the possible untriggered guards. Instead we introduce a dummy untriggered event that is only fired when no other selection of untriggered transitions are available and sets a boolean flag, UC, to indicate that none of the real untriggered events was fired and a trigger needs to be consumed.