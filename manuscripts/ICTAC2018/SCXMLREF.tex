\documentclass[runningheads,a4paper]{llncs}

% Set TOC depth to 3
%\setcounter{tocdepth}{3}

% Package for change tracking
%\usepackage[disabled]{chgtrk}
\usepackage{chgtrk}
\newCTcontributor{Karla}
\newCTcontributor{Colin}
\newCTcontributor{Rob}
\newCTcontributor{Son}
\newCTcontributor{Michael}

% Package for typesetting AMS Symbols
\usepackage{amssymb}

% Package for typesetting abbreviations
\usepackage{abbrev-SCXMLREF}


% Package for typesetting diagrams using TikZ
\usepackage{tikz}
\usetikzlibrary{positioning}
%\usepackage{pgf-picture}

% Package for typesetting requirements
%\usepackage[compact]{reqdoc}

% Package for typesetting Event-B mathematical symbols
\usepackage{bsymb}

% Package for typesetting SCXMLREF example models in Event-B
%\usepackage{eventB-SCXMLREF}

% Package for AMS Math
\usepackage{amsmath}

% Package for typesetting URLs
\usepackage{url}
\urldef{\mailsa}\path|{cfs, mjb}@ecs.soton.ac.uk|
\urldef{\mailsa}\path|{knmorri, rob}@sandia.gov|

% Package for highlight TODOs
%\usepackage[disable]{hltodonotes}
%\usepackage[]{hltodonotes}

\newcommand{\keywords}[1]{\par\addvspace\baselineskip
\noindent\keywordname\enspace\ignorespaces#1}

% Package for fancy references
\usepackage{varioref}

% Package for including figures
\usepackage{graphicx}
\usepackage{subcaption}

% Package for sub-floats (e.g. figures)
% \usepackage{subfig}

% Package for including standalone source files
\usepackage{standalone}

% Package for listings (e.g. JavaScript)
%\usepackage{eventBlistings}

\usepackage{hyperref}
\hypersetup{
  colorlinks=true,
  linkcolor = blue,
  urlcolor=cyan!50!black,
  citecolor=cyan,
}

\usepackage{examplep}
% Package for typesetting Event-B, load this package after all other packages
\usepackage[colour]{lstEventB}

% Listing for XML
\definecolor{dkgreen}{rgb}{0,0.6,0}
\definecolor{gray}{rgb}{0.5,0.5,0.5}
\definecolor{mauve}{rgb}{0.58,0,0.82}
\definecolor{gray}{rgb}{0.4,0.4,0.4}
\definecolor{darkblue}{rgb}{0.0,0.0,0.6}
\definecolor{lightblue}{rgb}{0.0,0.0,0.9}
\definecolor{cyan}{rgb}{0.0,0.6,0.6}
\definecolor{darkred}{rgb}{0.6,0.0,0.0}

\lstset{
  basicstyle=\ttfamily\footnotesize,
  columns=fullflexible,
  showstringspaces=false,
  numbers=left,                   % where to put the line-numbers
  numberstyle=\tiny\color{gray},  % the style that is used for the line-numbers
  stepnumber=1,
  numbersep=5pt,                  % how far the line-numbers are from the code
  backgroundcolor=\color{white},      % choose the background color. You must add \usepackage{color}
  showspaces=false,               % show spaces adding particular underscores
  showstringspaces=false,         % underline spaces within strings
  showtabs=false,                 % show tabs within strings adding particular underscores
  frame=none,                   % adds a frame around the code
  rulecolor=\color{black},        % if not set, the frame-color may be changed on line-breaks within not-black text (e.g. commens (green here))
  tabsize=2,                      % sets default tabsize to 2 spaces
  captionpos=b,                   % sets the caption-position to bottom
  breaklines=true,                % sets automatic line breaking
  breakatwhitespace=false,        % sets if automatic breaks should only happen at whitespace
  title=\lstname,                   % show the filename of files included with \lstinputlisting;
                                  % also try caption instead of title  
  commentstyle=\color{gray}\upshape
}

\lstdefinelanguage{XML}
{
  morestring=[s][\color{mauve}]{"}{"},
  morestring=[s][\color{black}]{>}{<},
  morecomment=[s]{<?}{?>},
  morecomment=[s][\color{dkgreen}]{<!--}{-->},
  stringstyle=\color{black},
  identifierstyle=\color{lightblue},
  keywordstyle=\color{red},
  morekeywords={xmlns,xsi,noNamespaceSchemaLocation,type,id,x,y,source,target,version,tool,transRef,roleRef,objective,eventually}% list your attributes here
}
% Listing for XML


\begin{document}

\mainmatter  % start of an individual contribution

% first the title is needed
\title{Refinement of SCXML state-charts via translation to Event-B}

% a short form should be given in case it is too long for the running head
\titlerunning{Refinement of SCXML}

% the name(s) of the author(s) follow(s) next
%
\author{C. Snook \inst{1} %\textsuperscript{https://orcid.org/0000-0002-0210-0983} 
\and K.Morris    \inst{2} 
\and R.Armstrong \inst{2}
\and T.S.Hoang   \inst{1}
\and M.Butler    \inst{1} 
}

%  \authorrunning{} has to be set for the shorter version of the authors' names;
% otherwise a warning will be rendered in the running heads. When processed by
% EasyChair, this command is mandatory: a document without \authorrunning
% will be rejected by EasyChair

\authorrunning{C. Snook, K.Morris et al.}

% Institutes for affiliations are also joined by \and,
\institute{
	University of Southampton,
	Southampton, United Kingdom\\
	\email{\{cfs,t.s.hoang,mjb\}@soton.ac.uk}\\
	\and
	Sandia National Laboratories, 
	Livermore, California, U.S.A.\\
	\email{\{knmorri,rob\}@sandia.gov}
}

\maketitle

% Reset all abbreviations
\resetabbrev

% !TEX root = ../SCXMLREF.tex
\begin{abstract}

State-chart modelling notations, such as State Chart eXtensible Markup Language (SCXML), with so-called `run to completion' semantics and simulation tools for validation, are popular with engineers for designing machines. However, they do not support refinement and they lack formal verification methods and tools. Properties concerning the synchronisation between different parts of a machine may be difficult to verify for all scenarios.  Event-B, on the other hand, is based on refinement from an initial abstraction and is designed to make formal verification by automatic theorem provers feasible, obviating the need for instantiation and testing. We would like to combine the best of both approaches by incorporating a notion of refinement, similar to that of Event-B, into SCXML and leveraging Event-B's tool support for proof. We describe some the pitfalls in translating 'run to completion' models to Event-B refinements, and suggest a solution and propose extensions to the SCXML syntax to describe refinements. We illustrate the approach using our prototype translation tools and show by example, how a synchronisation property between parallel state-charts can be automatically proven at an incomplete refinement level by translation into Event-B. 

\keywords SCXML, State-charts, Event-B, iUML-B, refinement
\end{abstract}

%%% Local Variables: 
%%% mode: latex
%%% TeX-master: "RailGround"
%%% End: 


% !TEX root = ../SCXMLREF.tex

\section{Introduction}
\label{sec:introduction}
This is the introduction...


% !TEX root = ../SCXMLREF.tex

\section{Background}
\label{sec:background}

% !TEX root = ../main.tex

\subsection{SCXML}
\label{sec:scxml}

\SCXML is a modelling language based on Harel state-charts with facilities for adding data elements that are modified by transition actions and used in conditions for their firing~\cite{scxmlwebsite}. 
\SCXML follows a `run to completion' semantics, where trigger events\footnote{In \SCXML the triggers are called `events', however, we refer to them as `triggers' to avoid confusion with \EventB} may be needed to enable transitions.
Trigger events are queued when they are raised, and then one is de-queued and consumed by firing all the transitions that it enables, followed by firing the un-triggered transitions that become enabled due to the change of state caused by the initial transition firing.
This is repeated until no transitions are enabled, and then the next trigger is de-queued and consumed.
Note that the enabledness of transitions is calculated batch-wise at each step, not after each and every transition.
Hence the set of parallel transitions that are enabled by a trigger is calculated and then only those are fired, irrespective of whether firing one may disable or enable another.
Similarly, the set of parallel untriggered transitions to be fired is calculated at each iteration before any is fired.
There are two kinds of triggers: internal triggers are raised by transitions and external triggers are raised by the environment (non-deterministicly for the purpose of our analysis). 
An external trigger may only be consumed when the internal trigger queue has been emptied.
We chose \SCXML as our source language because it is relatively simple compared to some run to completion modelling languages yet has a well defined action language and simulation tool support.

\ColinInlineComment{I have added the pseudocode back in because i like it (we removed it for space for detect) }

Listing~\ref{lst:scxml-r2c} shows a pseudocode representation of the run to completion semantics as defined within the latest W3C recommendation document~\cite{scxmlwebsite}. Here IQ and EQ are the triggers present in the internal and external queues respectively. We adopt the commonly used terminology where a single transition is called a \emph{micro-step} and a complete run (between de-queueing external triggers) is referred to as a \emph{macro-step}.

 \begin{lstlisting}[caption=Pseudocode for 'run to completion',label={lst:scxml-r2c}, frame=single]
 while running:
 	while completion = false
 		if untriggered_enabled
 			execute(untriggered())
 		elseif IQ /= {}
 			execute(internal(IQ.dequeue)) 
 		else
 			completion = true
 		endif
 	endwhile
 	if EQ /= {}
 		execute(EQ.dequeue) 
 		completion = false
 	endif
 endwhile 
 \end{lstlisting}



%%% Local Variables: 
%%% mode: latex
%%% TeX-master: "../main.tex"
%%% End: 

% !TEX root = ../SCXMLREF.tex

\subsection{Event-B}
\label{sec:eventb}

\eventB~\cite{abrial10:_model_event_b} is a formal method for system
development.  Main features of \eventB include the use of
\emph{refinement} to introduce system details gradually into the
formal model.  An \eventB model contains two parts: \emph{contexts} and \emph{machines}. Contexts contain \emph{carrier sets}, \emph{constants}, and \emph{axioms} constraining the carrier sets and constants.  Machines contain \emph{variables} \Bv, \emph{invariants} $I(\Bv)$ constraining the variables, and \emph{events}. An event comprises a guard denoting its enabled-condition and an action describing how the variables are modified when the event is executed.  In general, an event \Be has the following form, where \Bt are the event parameters, $G(\Bt, \Bv)$ is the guard of the event, and $S(\Bt, \Bv)$ is the action of the event.
\begin{align}
& \inlineevent{\Be}{}{\Bt}{G(\Bt,\Bv)}{}{S(\Bt,\Bv)}
\end{align}
In the case where the event has no parameters, we use the following form
\begin{align}
& \inlineevent{\Be}{}{}{G(\Bv)}{}{S(\Bv)}~,
\end{align}
and when the event has no parameters and guard, we use
\begin{align}
& \inlineevent{\Be}{}{}{}{}{S(\Bv)}~.
\end{align}
The action of an event comprises of one or more assignments, each of them has one of the following forms.
\begin{align}
& \Bv \bcmeq E(\Bt, \Bv) \label{eq:bcmeq}\\
& \Bv \bcmin E(\Bt, \Bv) \label{eq:bcmin}\\
& \Bv \bcmsuch P(\Bt, \Bv) \label{eq:bcmsuch}
\end{align}
Assignments of the form \eqref{eq:bcmeq} are deterministic, assign value of expression $E(\Bt, \Bv)$ to \Bv.  Assignments of the forms \eqref{eq:bcmin} and \eqref{eq:bcmsuch} are non-deterministic. \eqref{eq:bcmin} assigns any value from the set $E(\Bt,\Bv)$ to \Bv, while \eqref{eq:bcmsuch} assigns any value satisified predicate $P(\Bt,\Bv)$ to \Bv.
Note that invariants $I(\Bv)$ are inductive, i.e., they must be \emph{maintained} by all events. This is more strict than general safety properties which hold for all reachable states of the \EventB machine.  This is also the difference between verifying the consistency of \EventB machines using theorem proving and model checking (e.g., \PROB) techniques: model checkers explore all reachable states of the system while interpreting the invariants as safety properties.

A machine in \eventB corresponds to a transition system
where \emph{variables} represent the states and \emph{events} specify
the transitions.    More information about \eventB can be found in~\cite{hoang13:_introd_event_b_model_method}.  \eventB is supported by the
\Rodin~\cite{abrial10:_rodin}, an extensible toolkit which includes
facilities for modelling, verifying the consistency of models
using theorem proving and model checking techniques, and validating
models with simulation-based approaches.

In Event-B the run to completion pseudocode of Listing~\ref{lst:scxml-r2c} could be represented (somewhat abstractly) as

\begin{Bcode}
	$%
	\event{FireUntriggered}{}{}{UC=FALSE}{}{execute(untriggered())}
	\event{FireInternallyTriggered}{}{}{UC=TRUE\\IQ\neq\emptyset}{}{execute(IQ.dequeue)\\UC:=FALSE}
	\event{FireExternallyTriggered}{}{}{UC=TRUE\\IQ=\emptyset\\EQ\neq\emptyset}{}{execute(EQ.dequeue)\\UC:=FALSE}
	$
%	\Bvspace[2ex]
\end{Bcode}

Note that this is an abstract representation where each event (FireUntriggered, FireInternallyTriggered, and FireExternallyTriggered) would be specialised to select a particular set of transitions that can be fired in parallel and \emph{execute()} would be replaced by actions that encode the state changes made by those transitions.
Representing the condition \textbf{untriggered\_enabled} (line 3 in Listing~\ref{lst:scxml-r2c}) is cumbersome since we would need to write a conjunction of all the possible untriggered guards. Instead we introduce a dummy untriggered event that is only fired when no other selection of untriggered transitions are available and sets a boolean flag, UC, to indicate that none of the real untriggered events was fired and a trigger needs to be consumed.
% !TEX root = ../main.tex

% \subsection{UML-B State-machines}
% \label{sec:iumlb}

\paragraph{UML-B State-machines} provides a diagrammatic modelling notation for \EventB in the form of state-machines and class diagrams~\cite{said15:umlbSosym,snook14:iumlbStatem,snook06umlbTosem}. 
The diagrammatic models relate to an \EventB machine and generate or contribute to parts of it. 
For example a state-machine will automatically generate the \EventB data elements (sets, constants, axioms, variables, and invariants) to implement the states. 
Transitions contribute further guards and actions representing their state change, to the events that they elaborate.  
State-machines are typically refined by adding nested state-machines to states.
% Figure~\ref{fig:iumlb-sm} shows an example of a simple state-machine with two states.
% \begin{figure}[!h]
% 	\vspace{-.5cm}
% 	\centering
% 	\includegraphics[width=0.6\textwidth]{figures/iumlb-SM}
% 	\caption{An example \UMLB state-machine}
% 	\label{fig:iumlb-sm}
% 	\vspace{-.5cm}
% \end{figure}

Each state is encoded as a boolean variable and the current state is indicated by one of the boolean variables being set to |TRUE|. 
An invariant ensures that only one state is set to |TRUE| at a time.
%The state-machine, is initialised by setting one state variable to |TRUE| and all others to |FALSE|.
Events change the values of state variables to move the |TRUE| value according to the transitions in the state-machine.  
% The \EventB translation%
% %
% \footnote{%
%   Here, $\mathrm{partition(S, T1, T2, \ldots)}$ means the set $S$ is partitioned into disjoint (sub-)sets $T1, T2, \ldots$.
% that cover $S$} 
% of the state-machine in Figure~\ref{fig:iumlb-sm} can be seen in Listing~\ref{lst:eventb-sm}.
% \UMLB also provides the option of an alternative translation with a single state variable ranging over an enumerated type of states, however, the boolean representation of each state is more natural for a user to reference in \SCXML guards and actions.
	
While the \UMLB translation deals with the basic data formalisation of state-machines it differs 
significantly from the semantics discussed in this manuscript. 
\UMLB adopts \EventB's simple guarded action semantics and does not have a concept of triggers and run-to-completion.
Here we make use of \UMLB's state-machine translation but provide a completely different semantic by generating a behaviour into the underlying \EventB events that are linked to the generated \UMLB transitions.
% \begin{lstlisting}[caption={Translation of the state-machine in Fig.~\ref{fig:iumlb-sm}},label={lst:eventb-sm}, language=Event-B, escapechar=|, frame=single]
% variables S1 S2
% invariants 
% 	TRUE !: {S1, S2} => partition({TRUE}, {S1}/\{TRUE}, {S2}/\{TRUE})
% events
%     INITIALISATION: begin S1, S2 := TRUE, FALSE end
%     e: when S1 = TRUE then S1, S2 ≔ FALSE, TRUE  end
%     f: when S2 = TRUE then S2 := FALSE end
% end
% \end{lstlisting}
%%% Local Variables:
%%% mode: latex
%%% TeX-master: "../main"
%%% End:



% !TEX root = ../SCXMLREF.tex


\section{Intrusion Detection System}
\label{sec:secbot}

An \IDS is used to illustrate the use of refinement in \statecharts and how it is supported by \EventB verification tools.
The \IDS is designed using an \ASIC which connects to a buzzer and a sensor over a \SPI bus. The system is controlled via the \ASIC on the \SPI bus. At power-up, the \ASIC sends commands over the \SPI bus to initialise the sensor and the buzzer. After waiting for 50 milliseconds the \ASIC enters its main routine, which makes the buzzer respond to the sensor. In the early design phase the \statechart model of this system may be limited to the \ASIC that captures the initialisation of the peripherals and the 50 ms wait. In the interest of simplicity, we elide all details of the main routine.

A \statechart model of this system is shown in Fig.~\ref{fig:ASIC}. The \ASIC starts by initialising the buzzer, this involves sending a message over the \SPI bus. These messages constitute an implementation detail that we elide at this abstraction level. Once the message is sent (which will be indicated by some event saying that the \SPI system is done), the \ASIC moves on to initialise the sensor. After that the \ASIC moves into a waiting state for 50 ms, and finally moves into the state which represents normal operation. At this abstraction the \textbf{spi\_done} triggered, which signals completion by the \SPI system, is an internal trigger that can be fired at any time.

\begin{figure*}[t!]
    % \begin{centering}
	    \begin{subfigure}[t]{0.3\textwidth}
	        \begin{centering}
	        \includegraphics[height=3in]{figures/ASIC}
	        \caption{\ASIC component high level abstraction}
	        \label{fig:ASIC}
	        \end{centering}
	    \end{subfigure}
\qquad
	    \begin{subfigure}[t]{0.5\textwidth}
	        % \begin{centering}
	        \includegraphics[height=3in]{figures/ASIC&SPI_1}
	        \caption{First refinement introducing the abstract model of the \SPI subsystem.}
	        \label{fig:ASIC_SPI_1}
	        % \end{centering}
	    \end{subfigure}
	    \caption{\Statechart diagram for \IDS including the abstract representation of the \ASIC and \SPI components.}
    % \end{centering}
\end{figure*}

In a subsequent level of refinement, shown in Fig.~\ref{fig:ASIC_SPI_1}, the designer adds a parallel state representing the \SPI subsystem. The \SPI subsystem is usually on an \textbf{Idle} state until the \textbf{send\_message} trigger is raised, at which point the \SPI subsystem enters a state \textbf{Sending Message}, which represents sending the message, byte by byte. When the last byte of the message is sent, it raises the \textbf{spi\_done} trigger, allowing the other parallel state to continue, while \SPI subsystem returns to idle. In the current refined model we have incorporated the implementation details for raising \textbf{spi\_done} and introduced a new internal trigger 
\textbf{send\_message}, which is nondeterministic at this point.

The model can be further refined by incorporating more details on how the initialisation states, the wait state, and the \SPI subsystem operate, including how they interact with each other. The \statechart diagram for this refinement level is in Fig.~\ref{fig:ASIC_SPI_2}. The \textbf{Initialise Buzzer} state constructs the \SPI message to send, then raises the \textbf{send\_message} trigger, and then waits.
After \textbf{send\_message} is raised, the \SPI subsystem reacts. It spins for a while in the \textbf{Send Byte} state, looping as many times as it takes to get to the last byte in the message. When the last byte in the message is sent, it goes back to \textbf{Idle} and raises an event which allows the state machine on the left to proceed. The sensor is then initialised in a very similar manner to the buzzer. After both peripherals are initialised, the state machine goes into the \textbf{Wait 50 ms} state, where it increments a counter until it reaches some maximum, then exits.

\begin{figure}[!htbp]
  \begin{centering}
  \includegraphics[width=0.8\textwidth]{figures/ASIC&SPI_2}
  \caption{\Statechart diagram for \IDS including implementation details for the messages send between the system components.}
  \label{fig:ASIC_SPI_2}
  \end{centering}
\end{figure} 

The system described must send messages to complete the initialisation of the buzzer and sensor, but once the main routine is reached (\textbf{Go} state) no more messages should be sent through the \SPI bus. As a result, when the \ASIC is in the \textbf{Go} state the \SPI subsystem must be in the \textbf{Idle} state. This system property must be satisfied by the system model first refinement and any subsequent refinement representations of the system.



%%% Local Variables:
%%% mode: latex
%%% TeX-master: "../SCXMLREF"
%%% End:



% !TEX root = ../SCXMLREF.tex

\section{Design Rationale}
\label{sec:discussion}
 
We consider the kinds of things we would like to do in \SCXML refinements and what properties should be preserved.
In practice, we wish to leverage existing \EventB verification tools and hence adopt a notion of refinement that can be automatically translated into an equivalent \EventB model consisting of a chain of refinements.
We use particular refinement idioms at the \statechart level that correspond to \EventB's superposition refinement and thus have simple proof obligations. 
These refinement idioms are very natural from an engineering perspective (as illustrated by the running example).
Hence we start from the following requirements which allow superposition refinements and guard strengthening in \SCXML models:
\begin{itemize}
	\item The firing conditions of a transition can be strengthened by adding further textual constraints about the state of other variables and state machines in the system.
	\item The firing conditions of a transition can be strengthened by being more specific about the (nested) source state,
	\item Nested \Statecharts can be added in refinements.
	\item \KarlaAdd{Actions that modify ancillary data can be added to transitions.} \KarlaDelete{Ancillary data, with corresponding actions to alter it, can be added to transitions.}
	\item Raise actions can be added to transitions to define how internal triggers are raised. These internal triggers may have already been introduced and used to trigger transitions, in which case they are non-deterministically raised at the abstract levels. \KarlaDelete{(Note that external triggers are always unguarded and cannot be refined).}
	\item \KarlaAdd{External triggers represent inputs to the model. If no restrictions are imposed on the inputs then the events that raise external trigger are always unguarded and cannot be refined.}
	\item Invariants can be added to states to specify properties that hold while in that state.
\end{itemize}
While it would be possible to utilise \EventB's data refinement to perform more substantial \statechart refinements (for example replacing an abstract \statechart with a different one in the refined model), this would lead to complex proof obligations and is impractical when the \SCXML model is a single \Statechart (rather than a chain of refined models).


\begin{sloppypar}
  Adherence to \EventB refinement means that refined transitions
  (hence micro- and macro-steps) should preserve the abstract state
  and new ones should not alter the abstract state.
  % (An alternative approach that we are considering for future work
  % takes the view that the macro-steps need not align and a
  % micro-step may shift from one macro-step to another in a
  % refinement).
  With this approach, there is an inherent difficulty in refining `run
  to completion' semantics where every enabled micro-step must be
  completed before the next macro-step is started.  The problem is
  that, in a refinement, we want to strengthen the conditions for a
  micro-step.  However, by making the micro-steps more constrained we
  may disable them and hence make the completion of enabled ones more
  easily achieved.  This makes the guard for taking the next
  macro-step weaker breaking the notion of refinement.
\end{sloppypar}

While it is always possible to abstract away sufficiently to reach a common semantics (see~\cite{Snook12:FMCO} for example), in this work we wish to explore verification that considers `run to completion' behaviour as closely as possible.    
To simulate the `run to completion' semantics in \EventB, we initially adopted a scheduler approach where `engine' events decide which user transitions should be fired based on their guards. 
Boolean flags were then used to enable these transitions which may fire before the next step of the engine.
The engine implemented the operational semantics of Listing~\ref{lst:scxml-r2c} by deciding when to use internal or external triggers.
To allow for transition guards to be strengthened in later refinements (hence achieving completion earlier) the scheduling engine was allowed to continue without actually firing the transitions.
However, this non-deterministic completion introduced many additional behaviours making simulation difficult.
%We abstract away from the \SCXML rules about executing parallel transitions in document order and adopt non-deterministic firing order of transitions. 
%A consequence is that if transitions that can fire in parallel assign to the same variable as each other the resulting value is non-deterministic.
%\ColinInlineCommented{Son's suggestion to move the non-determinism to the enabling step may improve this scheduler engine approach}

Due to these difficulties with non-deterministic completion we developed an alternative approach where a separate event is generated for each combination of transitions that could possibly be fired together in the same step. 
For example, if T1 and T2 are transitions that could both become enabled at the same scheduler step, four events are needed to cater for the possible combinations: neither, T1, T2 and both (where the combined event is constructed from the conjunction of guards and parallel firing of actions). 
To allow for strengthening of the guards  in refinement we omit the negation of guards
%(if $Gref \implies Gabs$  it does not follow that $\neg Gref \implies \neg Gabs$) 
leaving the choice of lesser combinations, including the empty one, non-deterministically available in case of future refinement.
For example, T1 could fire alone even if T2 is enabled since we cannot add the negation of T2's guard to T1 unless we know that it will never be strengthened. 
This non-determinism in the model accurately reflects the abstract run to completion where we do not yet know whether T2 will be enabled or not in future refinements.
\ColinAdd{The non-determinism is useful to allow abstractions which facilitate verification proofs but must be removed in refinements to reach a design suitable for implementation.}
In future work we intend to add an attribute \emph{finalised} to indicate that no further guard strengthening refinements will be made to a transition, removing non-determinism throughout the refinement chain.


Since there is only ever a single event to be fired in a particular micro-step, the scheduler can be integrated with the events that represent the transition combinations, greatly simplifying the Event-B model.
Instead of explicit events to progress and implement the scheduling engine, an abstract machine is provided with events that can be refined by the translation of the user's \SCXML model into events that represent combinations of transitions that can fire in the same micro-step.
\ColinAdd{Each refinement produces a new set of events representing the (possibly extended) transition combinations that may occur at that level of refinement.} 
This has benefits both for simulation (i.e. execution of the \statechart for validation) which is easier to follow having less translation artefacts and for proof where the obligations are directly associated with particular transition combinations. 
Another benefit is that any parallel assignments to the same variable are rejected by the Event-B static checker.
The disadvantage, of course, is that there could be a combinatorial explosion in the number of events generated.
In practice though, this is unlikely since, to be fired in parallel, transitions must have the same trigger and be located in parallel \statecharts.
A high number of events is also not necessarily a bad thing since they are automatically generated and the main purpose of the \EventB model is for proof which could be simplified by replacing some of the unnecessary sequential steps of the model by a choice.
If the number of combinations is excessive it may indicate poor modelling style which can be reduced by introducing more internal triggers.
So far our examples have required few or no parallel transitions.

The following syntax extensions are added to \SCXML models to support refinement and invariant verification. 
\begin{itemize}
	\item \textbf{refinement} - an integer attribute representing the refinement level at which the parent element should be introduced,
	\item \textbf{invariant} - an invariant property (such as synchronisation of state with ancillary data and other state machines) that holds while in the parent state,
	\item \textbf{guard} - a guard condition of the parent transition (allowing transition conditions to be added at particular refinement levels). 
\end{itemize}
%%% Local Variables: 
%%% mode: latex
%%% TeX-master: "../SCXMLREF.tex"
%%% End: 


% !TEX root = ../SCXMLREF.tex

\section{Tooling}
\label{sec:tooling}

\ColinCommented{Things to cover: EMF metamodel, translation tool, anything else?}
This is the section about our plug-in..

% !TEX root = ../SCXMLREF.tex

\section{\SCXML Translation}
\label{sec:translation}

The translation from \iUMLB to \EventB is based on an abstract `basis' that models the `run to completion' semantics. 
This basis consists of an \EventB \emph{context} and \emph{machine} that are the same for all input models and are refined by the specific output of the translation.  
The basis context, Listing~\ref{lst:BasisContext}, introduces a given set of all possible triggers that is partitioned into internal and external ones, some of which will be introduced in future refinements. 
Refinements partition these trigger sets further to introduce concrete triggers, leaving a new abstract set to represent the remaining triggers yet to be introduced. 
For example, the \IDS model introduces a specific internal trigger, \textbf{spi\_done},  by partitioning |SCXML_FutureInternalTrigger| into the singleton \textbf{\{spi\_done\}} and a new set, |SCXML_FutureInternalTrigger0|, representing the remainder. 
 % as shown in line~\ref{line:refPartition} of Listing~\ref{lst:SecBotCont0}. 

\begin{lstlisting}[caption={Abstract basis context},label={lst:BasisContext}, language=Event-B, escapechar=|, frame=single, basicstyle=\rmfamily\scriptsize]
context
	basis_c 	// (generated for SCXML)
sets
	SCXML_TRIGGER	 // all possible triggers
constants
	SCXML_FutureInternalTrigger	 // all possible internal triggers
	SCXML_FutureExternalTrigger	 // all possible external triggers  
axioms
	partition(SCXML_TRIGGER, SCXML_FutureInternalTrigger, SCXML_FutureExternalTrigger) 
end
\end{lstlisting}	

% \begin{lstlisting}[caption={Context for \IDS abstract model},label={lst:SecBotCont0}, language=Event-B, escapechar=|, frame=single]
% context
% 	IDS_Model_0_ctx //(generated from:/IDS_generated/secbot.scxml)
% extends
% 	basis_c 
% constants
% 	SCXML_FutureInternalTrigger0	
% 	SCXML_FutureExternalTrigger0
% 	spi_done	 	//trigger
% axioms
% 	SCXML_FutureExternalTrigger0=SCXML_FutureExternalTrigger
% 	partition(SCXML_FutureInternalTrigger, SCXML_FutureInternalTrigger0,{spi_done}) |\label{line:refPartition}|
% end
% \end{lstlisting}

The basis machine, part of which is shown in Listing~\ref{lst:BasisMachine}, declares variables that correspond to the triggers present in the queue at any given time, and a flag, |SCXML_uc|, that signals when a run to completion macro-step has been completed (no un-triggered transitions are enabled). 
After initialisation, both trigger queues are empty and |SCXML_uc| is set to |FALSE| so that un-triggered transitions are dealt with. 
The basis machine provides events that describe the generic behaviour of models that follow the run to completion semantics in terms of altering the trigger queues and completion flag.
Since new events introduced in a refinement cannot modify existing variables, all future events generated by translation of the specific \SCXML model, will refine these abstract events.
The abstract event, |SCXML_futureExternalTrigger| represents the raising of an external trigger.    
The abstract event, |SCXML_futureInternalTransitionSet| represents a combination of transitions that are triggered by an internal trigger. 
The guards of this event ensure prior completion of the previous macro-step. 
A similar event, |SCXML_futureExternalTransitionSet| (not shown) represents a combination of transitions that are triggered by an external trigger and has the additional guard that the internal trigger queue is empty.
These two triggered transition events reset the completion flag to ensure that any un-triggered transitions that may have become enabled have a chance to fire next.
The abstract event |SCXML_futureUntriggeredTransitionSet| represents a combination of transitions that are un-triggered and may only be fired when the completion flag is unset (FALSE).
It leaves the completion flag unset in case further combinations of un-triggered transitions are enabled.
All three of these transition events also allow for raising a non-deterministic set of internal triggers.
A final abstract event, |SCXML_completion|, sets the completion flag (TRUE) if it is not already set. At this abstract basis level, this is non-deterministically fired since we do not yet have any detail of what needs to be completed.

\begin{lstfloat}[!tb]
\begin{lstlisting}[caption={Abstract basis machine (part of)}, label={lst:BasisMachine},language=Event-B, escapechar=|, frame=single, basicstyle=\rmfamily\scriptsize]
machine 
	basis_m   // (generated for SCXML)
sees 
	basis_c 
variables
	SCXML_iq	  // internal trigger queue
	SCXML_eq	  // external trigger queue
	SCXML_uc	  // run to completion flag
invariants
	SCXML_iq ⊆ SCXML_FutureInternalTrigger	// internal trigger queue
	SCXML_eq ⊆ SCXML_FutureExternalTrigger	// external trigger queue
	SCXML_iq ∩ SCXML_eq= ∅					// queues are disjoint
	SCXML_uc ∈ BOOL							// completion flag
events

	INITIALISATION: 
	begin
		SCXML_iq := {}		//internal Q is initially empty
		SCXML_eq := {}		//external Q is initially empty
		SCXML_uc := FALSE	//completion is initially FALSE
	end

	SCXML_futureExternalTrigger: 
	any SCXML_raisedTriggers where
		SCXML_raisedTriggers ⊆ SCXML_FutureExternalTrigger 
	then
		SCXML_eq ≔ SCXML_eq ∪ SCXML_raisedTriggers 
	end

	SCXML_futureInternalTransitionSet: 
	any SCXML_it SCXML_raisedTriggers where
		SCXML_it ∈ SCXML_iq 
		SCXML_uc = TRUE 
		SCXML_raisedTriggers ⊆ SCXML_FutureInternalTrigger 
	then
		SCXML_uc ≔ FALSE 
		SCXML_iq ≔ (SCXML_iq ∪ SCXML_raisedTriggers) ∖ {SCXML_it} 
	end

	SCXML_futureUntriggeredTransitionSet: 
	any SCXML_raisedTriggers where
		SCXML_uc = FALSE
		SCXML_raisedTriggers ⊆ SCXML_FutureInternalTrigger
	then
		SCXML_uc ≔ FALSE 
		SCXML_iq ≔ SCXML_iq ∪ SCXML_raisedTriggers 
	end

end
\end{lstlisting}
\end{lstfloat}

The translation of a specific \SCXML model comprises two stages as follows. 
Firstly, all possible combinations of transitions that can fire together are calculated and corresponding events are generated, at appropriate refinement levels, that refine the abstract basis events.  
If these transitions raise internal triggers, a guard, (e.g. |{i1,i2...} <: SCXML_raisedTrigger|, where |i1,i2..| have been added to the internal triggers set), is introduced that defines the raised triggers parameter. 
The subset constraint leaves it open for more raised triggers to be added by later refinements.
For triggered transition combinations, the trigger is specified in a guard (see line~\ref{line:defTrigger} of Listing~\ref{lst:SecBotMach0}) that provides a value for the trigger parameter. 

\begin{lstlisting}[caption={Event-B event corresponding to internal triggered transition to \textbf{Wait50ms} state in refinement level 1 shown in Fig.~\ref{fig:ASIC}}, label={lst:SecBotMach0},language=Event-B, escapechar=|, frame=single, float=t]
spi_done__InitialiseSensor_Wait50ms:	
refines SCXML_futureInternalTransitionSet 
any SCXML_it SCXML_raisedTriggers where
	SCXML_it  ∈ SCXML_iq 
	SCXML_uc = TRUE
	SCXML_raisedTriggers ⊆ SCXML_FutureInternalTrigger
	InitialiseSensor = TRUE
	SCXML_it = spi_done  	//trigger for this transition |\label{line:defTrigger}|
then
	SCXML_uc ≔ FALSE
	SCXML_iq ≔ (SCXML_iq ∪ SCXML_raisedTriggers) ∖ {SCXML_it}
	InitialiseSensor ≔ FALSE
	Wait50ms ≔ TRUE
end
\end{lstlisting}

Secondly, the \SCXML state-chart is translated into a corresponding iUML-B state-machine whose transitions elaborate (i.e. add state change details to) the possible transition combination events that the transition may be involved in. 
A transition may fire in parallel with transitions of parallel nested state-machines that have the same (possibly null) trigger.
Fig.~\ref{fig:iumlb-verif} shows the generated \iUMLB first refinement level corresponding to the \IDS described in Fig.~\ref{fig:ASIC_SPI_1}. 
Table~\ref{tab:translation_rules} provides a summary of the main SCXML to iUML-B/Event-B translation rules.
The iUML-B state-machine is subsequently translated into Event-B using the standard iUML-B translation~\cite{snook14:_b_statem} which provides variables to model the current state and guards and actions to model the state changes that transitions perform..


\begin{EventBNoShortInline}
  \begin{table}[]
	\centering
	\begin{tabular}{@{}p{0.25\linewidth}p{0.4\linewidth}p{0.35\linewidth}@{}}
		\hline
		\textbf{SCXML feature} & \textbf{Generated Event-B} & \textbf{Notes} 
		\\\midrule
		Top level scxml model &
		A refinement chain of Event-B machines each containing an initialisation event and a root level iUML-B state-machine &
		The depth of the refinement chain is found by searching the scxml for the maximum refinement annotation
		\\\hline
%		Invariant owned by the top level scxml & 
%		Invariant owned by an Event-B machine produced from the containing scxml &
%		Added only at the refinement level defined in the invariant (default 0) 
%		\\\hline
		State not owned by a parallel & 
		State owned by the iUML-B state-machine that has been produced from the containing scxml or state &
		A refined state is also added in all of the refinements of the parent iUML-B state-machine 
		\\\hline
		Invariant owned by a state that generates an iUML-B state (i.e. not contained in a parallel). &
		Invariant owned by the iUML-B state that has been produced from the containing scxml state. &
		Added only at the refinement level defined in the invariant (defaults to first level at which containing iUML-B state is introduced)
		\\\hline
		State owned by a parallel element &
		An iUML-B state-machine is added to the state that has been generated from the owner of the parallel &
		The nested iUML-B state-machine is added starting from the refinement level that is annotated on the source state and continuing throughout subsequent refinements.
		\\\hline
		State that contains states &
		A nested iUML-B state-machine, with an initial state, is added to the iUML-B state that has been produced from the source state, if any, or from its containing state if it did not produce an iUML-B state. &
		The nested iUML-B state-machine is added starting from the refinement level that is annotated on the source state and continuing throughout subsequent refinements.
		\\\hline
		Initial Attribute of a top-level scxml model &
		An iUML-B initial state, and a transition from it to the iUML-B state indicated in the scxml initial attribute, are added to the iUML-B state-machine produced from the parent scxml &
		The iUML-B initial state and iUML-B transition are added at all refinement levels. The iUML-B transitions are set to elaborate the Event-B INITIALISATION event for that refinement level.
		\\\hline
		Final &
		An iUML-B state with a transition to a final state are added to the state-machine that has been generated from the containing scxml or state. The transition represents the same events that are linked to the transitions that exit the parent iUML-B state. &
		The iUML-B state, final state and transition are also added as refined elements to all of the refinements of the parent iUML-B state-machine
		\\\hline
		Transition &
		An iUML-B transition is added to the state-machine that has been generated from the containing scxml or state. The iUML-B transition’s source and target are those that have been produced from the corresponding scxml transition’s source and target states. 
		The transition elaborates generated Event-B events according to the rules given in Section ~\ref{sec:translation} &
		The iUML-B transition and elaborated Event-B events are also added as corresponding refined elements in all of the refinements of the parent iUML-B state-machine
%		\\\hline
%		Target attribute of a transition element &
%		Used to determine the transitions target state as described above. &
		\\\hline																	
	\end{tabular}
	\caption{Main SCXML to iUML-B/Event-B Translation Rules}
	\label{tab:translation_rules}
  \end{table}
\end{EventBNoShortInline}

A tool to automatically translate \SCXML models into \iUMLB has been produced. 
The tool is based on the \EMF and uses an \SCXML meta-model provided by Sirius~\cite{siriuswebsite} which has good support for extensibility. 
%The tooling for \iUMLB and \EventB already contains \EMF meta-models and provides a generic translator framework which has been specialised for the \SCXML to \iUMLB translation. 

%%% Local Variables:
%%% mode: latex
%%% TeX-master: "../SCXMLREF"
%%% End:

% !TEX root = ../SCXMLREF.tex

\section{Verification of Intrusion Detection System}
\label{sec:example}

One of our main goals is to express properties in \SCXML intermediate refinements and prove them via translation to \EventB.
In this section we illustrate how this can be done in the \IDS example.

Properties about the synchronisation of parallel state-machines (such as |Go = TRUE => Idle = TRUE|)
% and |ASIC=Wait50ms => SPI=IDLE|
can be difficult to verify for all scenarios via simulation in \SCXML. 
Proof of such properties is a major benefit of translating into Event-B.  
Furthermore,  in order to benefit from the abstraction provided by Event-B, we would like to prove such things at abstract levels before the complication of further details are introduced. 
Typically these further details concern the raising of internal triggers that contribute to the synchronisation we wish to verify. 
Therefore additional constraints, that are an abstraction of the missing details, are needed about triggers in order to perform the proof.
% \begin{figure}[!tbp]
\begin{figure}[!h]
	\vspace{-.4cm}
	\centering
	\includegraphics[width=0.8\textwidth]{figures/iumlb_verif}
	\caption{State invariants to be verified at refinement level 1.}
	\label{fig:iumlb-verif}
	\vspace{-.4cm}
\end{figure}

Fig.~\ref{fig:iumlb-verif} is the generated \iUMLB showing state invariants (textual properties with a star icon inside states) to be verified. Note that the invariants are added to the \SCXML model but are easier to visualise in the \iUMLB with the current tooling.
The main aim is to show the property |Idle=TRUE| holds in state |Go|. 
This is true because after sending the message while in |InitialiseSensor|, no other messages are triggered by the |ASIC|, so the |SPI| subsystem stays in the |Idle| state indefinitely. 
To enable the provers to discharge the proof obligation we work back along the |ASIC|'s sequence of states. 
That is, |Idle = TRUE| is maintained in state |Go| if it holds in state |Wait50ms| and no |send_message| triggers are raised by the entry transition |Wait50ms_Go| nor once the |ASIC| subsystem is in state |Go|. 
To ensure this we add a guard |send_message /: SCXML_raisedTriggers| to |Wait50ms_Go| to prevent any future refinement from raising the trigger |send_message|.
(Currently, this is added verbatim but we envision a `doesn't raise' notation to avoid the user having to reference the translation artefact, |SCXML_raisedTriggers|).
We also need to prevent any future transitions from raising this trigger in the state |Go|.
To automate this for all abstract `future' events, they could be automatically generated and added to satisfy all user invariants concerning the raising of internal triggers regardless of whether they are violated in future levels. 
For example, the guard  |Go = TRUE => send_message /: SCXML_raisedTriggers| needs to be automatically added to the three `basis' events, |SCXML_futureUntriggeredTransitionSet|, |SCXML_futureInternalTransitionSet| and |SCXML_futureExternalTransitionSet| to prove they do not break the property being verified. 
If it is not obeyed by future transitions, guard strengthening proof obligations will fail, making it obvious where the problems lie.
As indicated above, we now need to prove by similar means that |Idle=TRUE| holds in state |Wait50ms|. 
In this case, however, we can only say that |Idle=TRUE| in state |InitialiseSensor| after the \SPI-system finishes sending the message and raises the trigger, |spi_done|. 
Hence the state invariant for |InitialiseSensor| becomes |spi_done∈SCXML_iq => Idle=TRUE|. 
In order to prove this we again need a corresponding state invariant about |send_message| and need to make sure that the |SPI| system will never raise |send_message|.
We also ensure it does not raise |spi_done| until it is finished. 
With these invariants and additional guards the Rodin automatic provers are able to prove all proof obligations and hence verify that the |SPI| system remains in |Idle| after servicing the `Initialise Sensor' message.

In order to prove properties at an abstract level we constrain the behaviour to be added in later refinements. 
For example, we needed to add a guard to specify that a transition does not raise a particular trigger in any future refinement. 
The abstract constraints should not appear in later refinements when the details have been finalised. To do this we could introduce ranges into our refinement attributes.

%%% Local Variables:
%%% mode: latex
%%% TeX-master: "../SCXMLREF"
%%% End:

% !TEX root = ../SCXMLREF.tex

\section{Conclusion}
\label{sec:conclusion}

We have shown how a slightly extended and annotated startchart, with a typical 'run to completion' semantic, can be translated into the \EventB notation for verification of synchronisation properties using the powerful \EventB theorem proving tools.
Furthermore, borrowing from the refinement concepts of \EventB, we introduce a notion of refinement to statecharts and demonstrate how the proof of a property at an abstract level, helps formulate constraints that must apply (and will be verifed to do so) in further refinements.

In future work we will continue to experiment with different examples to explore the alternative translation strategies in more detail. 
In particular, further work on refinement of the micro/macro-step and whether correspondence of macro-steps can be relaxed; whether more complex refinement techniques could be supported (for example, using ranges in refinement annotations) would be useful; supporting/comparing alternative variations of semantics (by generating a different basis/scheduler for the translation).

\RobCommented{Add some stuff on auto-abstract interpretation. (?)}
While Statecharts interpreted in iUML-B provide a way to incorporate refinement in an intuitive way, reversing this to \emph{discover} refinements holds promise. 
Checking a particular Statechart model for heirarchical structures that happen to follow the refinement proof obligations suggests an automatic way to accomplish abstract interpretation on an existing model.  
Such discovered abstraction/refinement relationships might improve the scalability of more complex Statechart models ``for free''.


\section*{Acknowledgment} The authors would like to thank Jason Michnovicz for developing the \IDS example used throughout the manuscript.

\begin{footnotesize} %small} %scriptsize}
\vspace{10 pt}
\par
\noindent
All data supporting this study are openly available from the University of Southampton repository at
http://doi.org/10.????/SOTON/D0???\\   %<GET a new DOI!!>
\end{footnotesize} %small} %scriptsize}


\bibliographystyle{plain}
\bibliography{SCXMLREF}

\end{document}

%%% Local Variables: 
%%% mode: latex
%%% TeX-master: t
%%% End: 
