\section{Summary}
\label{sec:summary}

This short paper provides some insights comparing the two modelling styles for formalising semantics of modelling languages: using \EventB contexts/machines vs using the Theory plugin's theories. In both approaches, the syntactical constraints on the models can be represented, either as axioms in the context or as the definition of the well-definedness operators.  While contexts and context extensions are a natural way to introduce the syntactical elements of a modelling language gradually, essentially they implicitly represent a single model (e.g., a single statechart). Using the Theory plugin, a datatype and the corresponding well-definedness operator represent all valid models (e.g., all well-defined statecharts).  The explicit representation of models as objects from a datatype allows us to write theorems in first-order logic about these well-defined models. We predicted that using Theory will help with stating and reasoning about model relationships such as refinement.